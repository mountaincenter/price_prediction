% parent.tex   v1.0  (2025-05-29)
% ───────────────────────────────────────────
% CHANGELOG:
% - 2025-06-10  event phase1 追加
% - 2025-06-07  range phase5 追加
% - 2025-06-10  open_price phase5 追加
% - 2025-06-06  center_shift phase3 追加
% - 2025-05-29  open_price ディレクトリ分割に合わせて新規作成
%
% Header-rules:
% • 未来日・過去日を入れない(Created は初回作成日だけ)
% • 修正があれば最上段に “- YYYY-MM-DD  概要” を追記
% • LaTeX では行頭 % を docstring の代替に使う
% • 内容を本文中や他コメントへ重複させない
% ───────────────────────────────────────────

\documentclass[dvipdfmx,openany,oneside]{jsbook}
\usepackage{amsmath,amssymb,tabularx,booktabs,graphicx}
\renewcommand{\arraystretch}{1.2}

\begin{document}

%-------------------------------------------------------------------------------
% basic_form.tex   v1.1  (2025-05-28)
%───────────────────────────────────────────────────────────────────────────────
% CHANGELOG:  最新→過去(降順)で追加してください
% - 2025-05-29  数式・表すべてを flushleft/flalign* ベースに統一
%-------------------------------------------------------------------------------

%=== 基本形:日中レンジを伴うモデル ==========================================
\section*{1. 基本形:日中レンジを伴うモデル}\nopagebreak[4]
%────────────────────────────────────
\begin{flushleft}
\begin{flalign*}
&\text{ベース値}\quad
  B_{t-1}=
    \begin{cases}
      Cl_{t-1} & (\text{デイトレ中心})\\[4pt]
      \dfrac{H_{t-1}+L_{t-1}}{2} & (\text{振れの大きい銘柄})
    \end{cases} &&\\[10pt]
%
&\text{中心シフト量}\quad
  \alpha_t=\kappa(\sigma_t)\,S_t,\qquad 0\le|S_t|\le1 &&\\[6pt]
%
&\text{日中中心値}\quad
  C_t=B_{t-1}(1+\alpha_t)\,\beta_{\text{event},t} &&\\[10pt]
%
&\text{半レンジ}\quad
  m_t=\sigma_t\,\beta_{\text{vol},t} &&\\[6pt]
%
&\text{高値}\quad
  H_t=C_t+m_t &&\\[4pt]
&\text{安値}\quad
  L_t=C_t-m_t &&\\[10pt]
%
&\text{始値}\quad
  O_t=C_t+\gamma_t\sigma_t &&\\[4pt]
&\text{終値}\quad
  Cl_t=C_t-\gamma_t\sigma_t &&
\end{flalign*}
\end{flushleft}

\subsection*{主要変数・係数(基本形)}
\begin{flushleft}
\begin{minipage}{0.85\textwidth}
\begin{tabularx}{\textwidth}{@{}>{\hfil$\displaystyle}l<{$\hfil}@{\quad}X@{}}
\toprule
記号 & 定義・役割 \\
\midrule
Cl_{t-1} & 前日終値 \\
H_{t-1},L_{t-1} & 前日高値・安値 \\
B_{t-1} & 前日リファレンス値 \\
\sigma_{t} & 当日ボラティリティ推定 \\
\kappa(\sigma) & ボラ依存シフトスケール \\
S_{t} & direction\_score \\
\beta_{\text{event},t} & 曜日・決算などバイアス係数 \\
\beta_{\text{vol},t} & 幅倍率(\(\sigma\) 拡大率) \\
\gamma_{t} & モメンタム偏位係数 \\
\bottomrule
\end{tabularx}
\end{minipage}
\end{flushleft}
\bigskip
%===============================================================================
   % 1 章:基本形
\clearpage

%-------------------------------------------------------------------------------
% event/phase0.tex   v1.4  (2025-06-02)
%-------------------------------------------------------------------------------
% CHANGELOG  -- newest -> oldest
% - 2025-06-02  v1.4 : section 階層見直し・ASCII 化・beta^{(3)} 表記へ統一
% - 2025-06-02  v1.3 : CHANGELOG 復元・整形を明記(rules.md 準拠)
% - 2025-05-31  v1.2 : fixed tabularx preamble to 3 columns (l X l)
% - 2025-05-31  v1.1 : \beta_event,i,t = \beta_weekday \times \beta_earn \times \beta_market
% - 2025-05-31  v1.0 : \beta_event,t = 1.0 fallback
%-------------------------------------------------------------------------------

%=== Phase 0 : イベント係数 基本定義 ============================================
\section*{event / Phase 0 : 基本定義}\nopagebreak[4]
%────────────────────────────────────
\begin{flushleft}
\begin{flalign*}
&\text{イベント係数(銘柄 }i\text{)}\quad
  \boxed{%
    \beta_{\text{event},i,t}
      =\beta_{\text{weekday},i,t}^{(3)}\,
       \beta_{\text{earn},i,t}\,
       \beta_{\text{market},i,t}
  } &&\\[6pt]
\end{flalign*}
\end{flushleft}

\subsection*{因子の役割}
\begin{flushleft}
\begin{minipage}{0.92\textwidth}
\begin{tabularx}{\textwidth}{@{}>{\hfil$\displaystyle}l<{$\hfil}@{\quad}X@{\quad}l@{}}
\toprule
因子 & 定義・データソース & 既定レンジ \\
\midrule
\beta_{\text{weekday},i,t}^{(3)} & 曜日+祝日+平滑済み最終係数 & 0.8--1.2 \\
\beta_{\text{earn},i,t}          & 決算ラグ・内容反映係数        & 0.8--1.5 \\
\beta_{\text{market},i,t}        & 指標相関係数                 & 0.8--1.2 \\
\bottomrule
\end{tabularx}
\end{minipage}
\end{flushleft}

\subsection*{備考}
\begin{flushleft}
\begin{itemize}
  \item 欠損時は 1.0 にフォールバック。  
  \item 係数更新は weekday / earn / market サブディレクトリで実施。  
\end{itemize}
\end{flushleft}
\bigskip
%===============================================================================
       % Phase-0:EWMA ギャップ
\clearpage

%-------------------------------------------------------------------------------
% event/earn/phase1.tex   v1.1  (2025-06-02)
%-------------------------------------------------------------------------------
% CHANGELOG  -- newest -> oldest
% - 2025-06-02  v1.1 : U+2212→ASCII "-", beta^{(1)} 表記, header tidy
% - 2025-05-31  v1.0 : 初版(決算 day±1 固定係数)
%-------------------------------------------------------------------------------

%=== Phase 1 : 決算係数(day±1) ==============================================
\section*{event / earn / Phase 1}\nopagebreak[4]
%────────────────────────────────────
\subsection*{ステップ・目的}
\begin{flushleft}
\begin{enumerate}
  \item \textbf{決算カレンダーでラグ判定}\;
        day -1(前日)/day 0(当日)/day +1(翌営業日)を抽出。
  \item \textbf{係数決定}\;
        \[
          \beta_{\text{earn},i,t}^{(1)}=
          \begin{cases}
            1.15 & (\text{day\,-1})\\
            1.20 & (\text{day\,0})\\
            1.10 & (\text{day\,+1})\\
            1.00 & (\text{otherwise})
          \end{cases}
        \]
  \item \textbf{イベント係数更新}\;
        \[
          \beta_{\text{event},i,t}^{(1)}
            =\beta_{\text{event},i,t}^{\text{prev}}
             \,\beta_{\text{earn},i,t}^{(1)},
          \quad 0.80 \le \beta_{\text{event},i,t}^{(1)} \le 1.50
        \]
\end{enumerate}
\end{flushleft}

\subsection*{追加変数・係数}
\begin{flushleft}
\begin{minipage}{0.90\textwidth}
\begin{tabularx}{\textwidth}{@{}>{\hfil$\displaystyle}l<{$\hfil}@{\quad}X@{}}
\toprule
記号 & 定義・役割 \\
\midrule
i & 銘柄コード \\
\beta_{\text{earn},i,t}^{(1)} & day±1 固定決算係数 \\
\beta_{\text{event},i,t}^{\text{prev}} & 直前フェーズ(weekday 等)出力 \\
\beta_{\text{event},i,t}^{(1)} & earn 系フェーズ 1 出力 \\
\bottomrule
\end{tabularx}
\end{minipage}
\end{flushleft}
\bigskip
%===============================================================================
       % Phase-1:IQR スケーリング
\clearpage

%-------------------------------------------------------------------------------
% center_shift/sigma/phase2.tex   v1.3  (2025-06-02)
%-------------------------------------------------------------------------------
% CHANGELOG  -- new entry on top
% - 2025-06-02  v1.3 : add section/subsection, hints, ASCII-only
%-------------------------------------------------------------------------------

%=== center_shift =============================================================
\section*{center\_shift}\nopagebreak[4]

%--- sigma ---------------------------------------------------------------------
\subsection*{sigma}\nopagebreak[4]

%--- Phase 2 : 自己適応 λ_shift 更新 -------------------------------------------
\subsubsection*{Phase 2:自己適応 $\lambda_{\text{shift}}$ 更新}\nopagebreak[4]
%────────────────────────────────────
\paragraph{ステップ/目的}
\begin{flushleft}
\begin{enumerate}
  \item \textbf{誤差系列}\;
        \(e_{t-k}=\Delta Cl_{t-k}^{2}-\sigma_{t-k}^{2}\)
  \item \textbf{局所 MSE}\;
        \(\mathrm{MSE}_t=\dfrac{1}{30}\sum_{k=1}^{30}e_{t-k}^{2}\)
  \item \textbf{勾配近似}\;
        \(g_t\approx-\dfrac{2}{30}\sum_{k=1}^{30}
          e_{t-k}\,\sigma_{t-k}^{2}\)
  \item \textbf{$\lambda_{\text{shift}}$ 更新}\;
        \(\lambda_{\text{shift},t}
          =\operatorname{clip}\bigl(
            \lambda_{\text{shift},t-1}-\eta g_t,\,
            0.90,\,0.98\bigr)\)
  \item \textbf{翌日へ反映}\;
        上式の \(\lambda_{\text{shift},t}\) で  
        \(\sigma_{t+1}^{2}\) を再計算
\end{enumerate}
\end{flushleft}

\subsubsection*{変数のポイント}
\begin{flushleft}
\begin{itemize}
  \item \(\lambda_{\text{shift}}\) は [0.90, 0.98] に制限
  \item \(|g_t|\le10\) でクリップし暴走を防止
\end{itemize}
\end{flushleft}

\subsubsection*{実装ヒント}
\begin{flushleft}
学習率 \(\eta=0.01\) が無難。  
ウォームアップ期間 (30~d) は固定 \(\lambda_{\text{shift}}=0.94\)。
\end{flushleft}

\subsubsection*{追加変数・係数}
\begin{flushleft}
\begin{minipage}{0.90\textwidth}
\begin{tabularx}{\textwidth}{@{}>{\hfil$\displaystyle}l<{$\hfil}@{\quad}X@{}}
\toprule
記号 & 定義・役割 \\
\midrule
\lambda_{\text{shift},t-1} & 前日 EWMA 定数 \\
\lambda_{\text{shift},t}   & 更新後 EWMA 定数 \\
g_t & 勾配近似 \\
\eta & 学習率 (0.01) \\
e_{t-k} & 誤差 \\
\mathrm{MSE}_t & 30~d MSE \\
\bottomrule
\end{tabularx}
\end{minipage}
\end{flushleft}
\bigskip
%===============================================================================
       % Phase-2:σ 比補正
\clearpage

%-------------------------------------------------------------------------------
% event/weekday/phase3.tex   v1.1  (2025-06-02)
%-------------------------------------------------------------------------------
% CHANGELOG  -- newest -> oldest
% - 2025-06-02  v1.1 : beta^{(3)} 表記・集約ロジック明確化
% - 2025-05-31  v1.0 : weekday 系サブフェーズ集約
%-------------------------------------------------------------------------------

%=== Phase 3 : weekday 系集約 ===================================================
\section*{event / weekday / Phase 3}\nopagebreak[4]
%────────────────────────────────────
\subsection*{ステップ・目的}
\begin{flushleft}
\begin{enumerate}
  \item \textbf{holiday 側最終係数を取り込み}\;
        \verb|%-------------------------------------------------------------------------------
% event/weekday/phase3.tex   v1.1  (2025-06-02)
%-------------------------------------------------------------------------------
% CHANGELOG  -- newest -> oldest
% - 2025-06-02  v1.1 : beta^{(3)} 表記・集約ロジック明確化
% - 2025-05-31  v1.0 : weekday 系サブフェーズ集約
%-------------------------------------------------------------------------------

%=== Phase 3 : weekday 系集約 ===================================================
\section*{event / weekday / Phase 3}\nopagebreak[4]
%────────────────────────────────────
\subsection*{ステップ・目的}
\begin{flushleft}
\begin{enumerate}
  \item \textbf{holiday 側最終係数を取り込み}\;
        \verb|%-------------------------------------------------------------------------------
% event/weekday/phase3.tex   v1.1  (2025-06-02)
%-------------------------------------------------------------------------------
% CHANGELOG  -- newest -> oldest
% - 2025-06-02  v1.1 : beta^{(3)} 表記・集約ロジック明確化
% - 2025-05-31  v1.0 : weekday 系サブフェーズ集約
%-------------------------------------------------------------------------------

%=== Phase 3 : weekday 系集約 ===================================================
\section*{event / weekday / Phase 3}\nopagebreak[4]
%────────────────────────────────────
\subsection*{ステップ・目的}
\begin{flushleft}
\begin{enumerate}
  \item \textbf{holiday 側最終係数を取り込み}\;
        \verb|\input{event/weekday/holiday/phase3}| で  
        \(\tilde\beta_{\text{weekday},i,t}\) を取得。
  \item \textbf{最終 weekday 係数を宣言}\;
        \[
          \boxed{\beta_{\text{weekday},i,t}^{(3)}
          =\tilde\beta_{\text{weekday},i,t}}
        \]
  \item \textbf{イベント係数パイプラインへ出力}\;
        event/phase0.tex が  
        \(\beta_{\text{weekday},i,t}^{(3)}\) を利用。
\end{enumerate}
\end{flushleft}

\subsection*{追加変数・係数}
\begin{flushleft}
\begin{minipage}{0.88\textwidth}
\begin{tabularx}{\textwidth}{@{}>{\hfil$\displaystyle}l<{$\hfil}@{\quad}X@{}}
\toprule
記号 & 定義・役割 \\
\midrule
\tilde\beta_{\text{weekday},i,t} & holiday/phase3 出力係数 \\
\beta_{\text{weekday},i,t}^{(3)} & weekday 系最終係数 (本フェーズ) \\
\bottomrule
\end{tabularx}
\end{minipage}
\end{flushleft}
\bigskip
%===============================================================================
| で  
        \(\tilde\beta_{\text{weekday},i,t}\) を取得。
  \item \textbf{最終 weekday 係数を宣言}\;
        \[
          \boxed{\beta_{\text{weekday},i,t}^{(3)}
          =\tilde\beta_{\text{weekday},i,t}}
        \]
  \item \textbf{イベント係数パイプラインへ出力}\;
        event/phase0.tex が  
        \(\beta_{\text{weekday},i,t}^{(3)}\) を利用。
\end{enumerate}
\end{flushleft}

\subsection*{追加変数・係数}
\begin{flushleft}
\begin{minipage}{0.88\textwidth}
\begin{tabularx}{\textwidth}{@{}>{\hfil$\displaystyle}l<{$\hfil}@{\quad}X@{}}
\toprule
記号 & 定義・役割 \\
\midrule
\tilde\beta_{\text{weekday},i,t} & holiday/phase3 出力係数 \\
\beta_{\text{weekday},i,t}^{(3)} & weekday 系最終係数 (本フェーズ) \\
\bottomrule
\end{tabularx}
\end{minipage}
\end{flushleft}
\bigskip
%===============================================================================
| で  
        \(\tilde\beta_{\text{weekday},i,t}\) を取得。
  \item \textbf{最終 weekday 係数を宣言}\;
        \[
          \boxed{\beta_{\text{weekday},i,t}^{(3)}
          =\tilde\beta_{\text{weekday},i,t}}
        \]
  \item \textbf{イベント係数パイプラインへ出力}\;
        event/phase0.tex が  
        \(\beta_{\text{weekday},i,t}^{(3)}\) を利用。
\end{enumerate}
\end{flushleft}

\subsection*{追加変数・係数}
\begin{flushleft}
\begin{minipage}{0.88\textwidth}
\begin{tabularx}{\textwidth}{@{}>{\hfil$\displaystyle}l<{$\hfil}@{\quad}X@{}}
\toprule
記号 & 定義・役割 \\
\midrule
\tilde\beta_{\text{weekday},i,t} & holiday/phase3 出力係数 \\
\beta_{\text{weekday},i,t}^{(3)} & weekday 系最終係数 (本フェーズ) \\
\bottomrule
\end{tabularx}
\end{minipage}
\end{flushleft}
\bigskip
%===============================================================================
       % Phase-3:Proxy Board Gap
\clearpage

%-------------------------------------------------------------------------------
% center_shift/phase4.tex   v1.0  (2025-06-06)
%-------------------------------------------------------------------------------
% CHANGELOG  -- new entry on top (latest -> oldest)
% - 2025-06-06  v1.0 : 初版
%-------------------------------------------------------------------------------

%=== center_shift =============================================================
\section*{center\_shift}\nopagebreak[4]

%=== Phase 4 : \eta / \lambda の深掘り ======================================
\subsection*{Phase 4:$\eta$ と $\lambda$ の深掘り}\nopagebreak[4]
%────────────────────────────────────
\paragraph{ステップ/目的}
\begin{flushleft}
\begin{enumerate}
  \item \textbf{学習率}
        \(\eta\) は $\lambda_{\text{shift}}$ 更新の歩幅を制御
  \item \textbf{勾配近似}
        \(g_t\approx-\dfrac{2}{30}\sum_{k=1}^{30}e_{t-k}\,\sigma_{t-k}^2\)
  \item \textbf{$\lambda_{\text{shift}}$ 更新}
        \(\lambda_{\text{shift},t}
          =\operatorname{clip}\bigl(\lambda_{\text{shift},t-1}
          -\eta\,g_t,\,0.90,\,0.98\bigr)\)
  \item \textbf{ウォームアップ}
        初期 30~d は固定 $\lambda_{\text{shift}}=0.94$ で安定化
\end{enumerate}
\end{flushleft}

\subsubsection*{変数のポイント}
\begin{flushleft}
\begin{itemize}
  \item 大きすぎる $\eta$ は \(\lambda_{\text{shift}}\) を振動させる
  \item 小さすぎる $\eta$ では収束が遅延
  \item 更新範囲 [0.90, 0.98] を超えないよう \(\operatorname{clip}\)
  \item $|g_t|>10$ なら勾配をクリップし安定化
\end{itemize}
\end{flushleft}

\subsubsection*{実装ヒント}
\begin{flushleft}
\begin{itemize}
  \item 経験的に $\eta=0.01$ が妥当な上限値
  \item 週次で $\eta$ の微調整を試し、予測 MAE を観察
  \item 勾配計算には 30~d の誤差系列を用意
\end{itemize}
\end{flushleft}

\subsubsection*{追加変数・係数}
\begin{flushleft}
\begin{minipage}{0.90\textwidth}
\begin{tabularx}{\textwidth}{@{}>{\hfil$\displaystyle}l<{$\hfil}@{\quad}X@{}}
\toprule
記号 & 定義・役割 \\
\midrule
\eta & 学習率 \\
\lambda_{\text{shift},t} & 更新後 EWMA 定数 \\
\lambda_{\text{shift},t-1} & 前日 EWMA 定数 \\
\sigma_t^2 & 分散推定値 \\
\operatorname{clip} & 範囲制限関数 \\
\end{tabularx}
\end{minipage}
\end{flushleft}
\bigskip
%==============================================================================

\clearpage

%-------------------------------------------------------------------------------
% event/earn/phase5.tex   v1.1  (2025-06-02)
%-------------------------------------------------------------------------------
% CHANGELOG  -- newest -> oldest
% - 2025-06-02  v1.1 : ASCII 統一, beta^{final} 表記, clip 修正
% - 2025-05-31  v1.0 : 初版(Bayes 縮小)
%-------------------------------------------------------------------------------

%=== Phase 5 : w_profit ベイズ縮小 =============================================
\section*{event / earn / Phase 5}\nopagebreak[4]
%────────────────────────────────────
\subsection*{ステップ・目的}
\begin{flushleft}
\begin{enumerate}
  \item \textbf{サンプル数取得}\;
        \( n_i=\text{count\_earnings}(i,\text{last 3Y}) \)

  \item \textbf{セクター平均重み}\;
        \( \bar w_{\text{profit},s}=\operatorname{mean}(w_{\text{profit},j}) \)

  \item \textbf{Bayes 縮小}\;
        \[
          \tilde w_{\text{profit},i}
            =\frac{n_i}{n_i+\tau}\,w_{\text{profit},i}
             +\frac{\tau}{n_i+\tau}\,\bar w_{\text{profit},s},
          \quad \tau = 10
        \]
        \( \tilde w_{\text{profit},i}=\operatorname{clip}(\tilde w_{\text{profit},i},0.50,0.90) \)

  \item \textbf{サプライズ率再計算} → $\beta_{\text{earn},i,t}^{(5)}$ を取得。

  \item \textbf{イベント係数最終更新}\;
        \[
          \beta_{\text{event},i,t}^{\text{final}}
            =\beta_{\text{event},i,t}^{(4)}\,
             \beta_{\text{earn},i,t}^{(5)}
        \]
\end{enumerate}
\end{flushleft}

\subsection*{追加変数・係数}
\begin{flushleft}
\begin{minipage}{0.92\textwidth}
\begin{tabularx}{\textwidth}{@{}>{\hfil$\displaystyle}l<{$\hfil}@{\quad}X@{}}
\toprule
記号 & 定義・役割 \\
\midrule
n_i & 過去 3 年の決算サンプル数 \\
\tau & 縮小ハイパーパラメータ (10) \\
\bar w_{\text{profit},s} & セクター平均利益重み \\
\beta_{\text{event},i,t}^{\text{final}} & earn 系最終係数 \\
\bottomrule
\end{tabularx}
\end{minipage}
\end{flushleft}
\bigskip
%===============================================================================

\clearpage

%-------------------------------------------------------------------------------
% event/phase0.tex   v1.4  (2025-06-02)
%-------------------------------------------------------------------------------
% CHANGELOG  -- newest -> oldest
% - 2025-06-02  v1.4 : section 階層見直し・ASCII 化・beta^{(3)} 表記へ統一
% - 2025-06-02  v1.3 : CHANGELOG 復元・整形を明記(rules.md 準拠)
% - 2025-05-31  v1.2 : fixed tabularx preamble to 3 columns (l X l)
% - 2025-05-31  v1.1 : \beta_event,i,t = \beta_weekday \times \beta_earn \times \beta_market
% - 2025-05-31  v1.0 : \beta_event,t = 1.0 fallback
%-------------------------------------------------------------------------------

%=== Phase 0 : イベント係数 基本定義 ============================================
\section*{event / Phase 0 : 基本定義}\nopagebreak[4]
%────────────────────────────────────
\begin{flushleft}
\begin{flalign*}
&\text{イベント係数(銘柄 }i\text{)}\quad
  \boxed{%
    \beta_{\text{event},i,t}
      =\beta_{\text{weekday},i,t}^{(3)}\,
       \beta_{\text{earn},i,t}\,
       \beta_{\text{market},i,t}
  } &&\\[6pt]
\end{flalign*}
\end{flushleft}

\subsection*{因子の役割}
\begin{flushleft}
\begin{minipage}{0.92\textwidth}
\begin{tabularx}{\textwidth}{@{}>{\hfil$\displaystyle}l<{$\hfil}@{\quad}X@{\quad}l@{}}
\toprule
因子 & 定義・データソース & 既定レンジ \\
\midrule
\beta_{\text{weekday},i,t}^{(3)} & 曜日+祝日+平滑済み最終係数 & 0.8--1.2 \\
\beta_{\text{earn},i,t}          & 決算ラグ・内容反映係数        & 0.8--1.5 \\
\beta_{\text{market},i,t}        & 指標相関係数                 & 0.8--1.2 \\
\bottomrule
\end{tabularx}
\end{minipage}
\end{flushleft}

\subsection*{備考}
\begin{flushleft}
\begin{itemize}
  \item 欠損時は 1.0 にフォールバック。  
  \item 係数更新は weekday / earn / market サブディレクトリで実施。  
\end{itemize}
\end{flushleft}
\bigskip
%===============================================================================

\clearpage

%-------------------------------------------------------------------------------
% event/earn/phase1.tex   v1.1  (2025-06-02)
%-------------------------------------------------------------------------------
% CHANGELOG  -- newest -> oldest
% - 2025-06-02  v1.1 : U+2212→ASCII "-", beta^{(1)} 表記, header tidy
% - 2025-05-31  v1.0 : 初版(決算 day±1 固定係数)
%-------------------------------------------------------------------------------

%=== Phase 1 : 決算係数(day±1) ==============================================
\section*{event / earn / Phase 1}\nopagebreak[4]
%────────────────────────────────────
\subsection*{ステップ・目的}
\begin{flushleft}
\begin{enumerate}
  \item \textbf{決算カレンダーでラグ判定}\;
        day -1(前日)/day 0(当日)/day +1(翌営業日)を抽出。
  \item \textbf{係数決定}\;
        \[
          \beta_{\text{earn},i,t}^{(1)}=
          \begin{cases}
            1.15 & (\text{day\,-1})\\
            1.20 & (\text{day\,0})\\
            1.10 & (\text{day\,+1})\\
            1.00 & (\text{otherwise})
          \end{cases}
        \]
  \item \textbf{イベント係数更新}\;
        \[
          \beta_{\text{event},i,t}^{(1)}
            =\beta_{\text{event},i,t}^{\text{prev}}
             \,\beta_{\text{earn},i,t}^{(1)},
          \quad 0.80 \le \beta_{\text{event},i,t}^{(1)} \le 1.50
        \]
\end{enumerate}
\end{flushleft}

\subsection*{追加変数・係数}
\begin{flushleft}
\begin{minipage}{0.90\textwidth}
\begin{tabularx}{\textwidth}{@{}>{\hfil$\displaystyle}l<{$\hfil}@{\quad}X@{}}
\toprule
記号 & 定義・役割 \\
\midrule
i & 銘柄コード \\
\beta_{\text{earn},i,t}^{(1)} & day±1 固定決算係数 \\
\beta_{\text{event},i,t}^{\text{prev}} & 直前フェーズ(weekday 等)出力 \\
\beta_{\text{event},i,t}^{(1)} & earn 系フェーズ 1 出力 \\
\bottomrule
\end{tabularx}
\end{minipage}
\end{flushleft}
\bigskip
%===============================================================================
 % κ(σ) Phase 1:段階定数モデル
\clearpage

%-------------------------------------------------------------------------------
% event/earn/phase1.tex   v1.1  (2025-06-02)
%-------------------------------------------------------------------------------
% CHANGELOG  -- newest -> oldest
% - 2025-06-02  v1.1 : U+2212→ASCII "-", beta^{(1)} 表記, header tidy
% - 2025-05-31  v1.0 : 初版(決算 day±1 固定係数)
%-------------------------------------------------------------------------------

%=== Phase 1 : 決算係数(day±1) ==============================================
\section*{event / earn / Phase 1}\nopagebreak[4]
%────────────────────────────────────
\subsection*{ステップ・目的}
\begin{flushleft}
\begin{enumerate}
  \item \textbf{決算カレンダーでラグ判定}\;
        day -1(前日)/day 0(当日)/day +1(翌営業日)を抽出。
  \item \textbf{係数決定}\;
        \[
          \beta_{\text{earn},i,t}^{(1)}=
          \begin{cases}
            1.15 & (\text{day\,-1})\\
            1.20 & (\text{day\,0})\\
            1.10 & (\text{day\,+1})\\
            1.00 & (\text{otherwise})
          \end{cases}
        \]
  \item \textbf{イベント係数更新}\;
        \[
          \beta_{\text{event},i,t}^{(1)}
            =\beta_{\text{event},i,t}^{\text{prev}}
             \,\beta_{\text{earn},i,t}^{(1)},
          \quad 0.80 \le \beta_{\text{event},i,t}^{(1)} \le 1.50
        \]
\end{enumerate}
\end{flushleft}

\subsection*{追加変数・係数}
\begin{flushleft}
\begin{minipage}{0.90\textwidth}
\begin{tabularx}{\textwidth}{@{}>{\hfil$\displaystyle}l<{$\hfil}@{\quad}X@{}}
\toprule
記号 & 定義・役割 \\
\midrule
i & 銘柄コード \\
\beta_{\text{earn},i,t}^{(1)} & day±1 固定決算係数 \\
\beta_{\text{event},i,t}^{\text{prev}} & 直前フェーズ(weekday 等)出力 \\
\beta_{\text{event},i,t}^{(1)} & earn 系フェーズ 1 出力 \\
\bottomrule
\end{tabularx}
\end{minipage}
\end{flushleft}
\bigskip
%===============================================================================
 % σ_t Phase 1:EWMA14 推定
\clearpage

%-------------------------------------------------------------------------------
% center_shift/sigma/phase2.tex   v1.3  (2025-06-02)
%-------------------------------------------------------------------------------
% CHANGELOG  -- new entry on top
% - 2025-06-02  v1.3 : add section/subsection, hints, ASCII-only
%-------------------------------------------------------------------------------

%=== center_shift =============================================================
\section*{center\_shift}\nopagebreak[4]

%--- sigma ---------------------------------------------------------------------
\subsection*{sigma}\nopagebreak[4]

%--- Phase 2 : 自己適応 λ_shift 更新 -------------------------------------------
\subsubsection*{Phase 2:自己適応 $\lambda_{\text{shift}}$ 更新}\nopagebreak[4]
%────────────────────────────────────
\paragraph{ステップ/目的}
\begin{flushleft}
\begin{enumerate}
  \item \textbf{誤差系列}\;
        \(e_{t-k}=\Delta Cl_{t-k}^{2}-\sigma_{t-k}^{2}\)
  \item \textbf{局所 MSE}\;
        \(\mathrm{MSE}_t=\dfrac{1}{30}\sum_{k=1}^{30}e_{t-k}^{2}\)
  \item \textbf{勾配近似}\;
        \(g_t\approx-\dfrac{2}{30}\sum_{k=1}^{30}
          e_{t-k}\,\sigma_{t-k}^{2}\)
  \item \textbf{$\lambda_{\text{shift}}$ 更新}\;
        \(\lambda_{\text{shift},t}
          =\operatorname{clip}\bigl(
            \lambda_{\text{shift},t-1}-\eta g_t,\,
            0.90,\,0.98\bigr)\)
  \item \textbf{翌日へ反映}\;
        上式の \(\lambda_{\text{shift},t}\) で  
        \(\sigma_{t+1}^{2}\) を再計算
\end{enumerate}
\end{flushleft}

\subsubsection*{変数のポイント}
\begin{flushleft}
\begin{itemize}
  \item \(\lambda_{\text{shift}}\) は [0.90, 0.98] に制限
  \item \(|g_t|\le10\) でクリップし暴走を防止
\end{itemize}
\end{flushleft}

\subsubsection*{実装ヒント}
\begin{flushleft}
学習率 \(\eta=0.01\) が無難。  
ウォームアップ期間 (30~d) は固定 \(\lambda_{\text{shift}}=0.94\)。
\end{flushleft}

\subsubsection*{追加変数・係数}
\begin{flushleft}
\begin{minipage}{0.90\textwidth}
\begin{tabularx}{\textwidth}{@{}>{\hfil$\displaystyle}l<{$\hfil}@{\quad}X@{}}
\toprule
記号 & 定義・役割 \\
\midrule
\lambda_{\text{shift},t-1} & 前日 EWMA 定数 \\
\lambda_{\text{shift},t}   & 更新後 EWMA 定数 \\
g_t & 勾配近似 \\
\eta & 学習率 (0.01) \\
e_{t-k} & 誤差 \\
\mathrm{MSE}_t & 30~d MSE \\
\bottomrule
\end{tabularx}
\end{minipage}
\end{flushleft}
\bigskip
%===============================================================================
 % σ_t Phase 2:IQR スケーリング
\clearpage

%-------------------------------------------------------------------------------
% event/earn/phase1.tex   v1.1  (2025-06-02)
%-------------------------------------------------------------------------------
% CHANGELOG  -- newest -> oldest
% - 2025-06-02  v1.1 : U+2212→ASCII "-", beta^{(1)} 表記, header tidy
% - 2025-05-31  v1.0 : 初版(決算 day±1 固定係数)
%-------------------------------------------------------------------------------

%=== Phase 1 : 決算係数(day±1) ==============================================
\section*{event / earn / Phase 1}\nopagebreak[4]
%────────────────────────────────────
\subsection*{ステップ・目的}
\begin{flushleft}
\begin{enumerate}
  \item \textbf{決算カレンダーでラグ判定}\;
        day -1(前日)/day 0(当日)/day +1(翌営業日)を抽出。
  \item \textbf{係数決定}\;
        \[
          \beta_{\text{earn},i,t}^{(1)}=
          \begin{cases}
            1.15 & (\text{day\,-1})\\
            1.20 & (\text{day\,0})\\
            1.10 & (\text{day\,+1})\\
            1.00 & (\text{otherwise})
          \end{cases}
        \]
  \item \textbf{イベント係数更新}\;
        \[
          \beta_{\text{event},i,t}^{(1)}
            =\beta_{\text{event},i,t}^{\text{prev}}
             \,\beta_{\text{earn},i,t}^{(1)},
          \quad 0.80 \le \beta_{\text{event},i,t}^{(1)} \le 1.50
        \]
\end{enumerate}
\end{flushleft}

\subsection*{追加変数・係数}
\begin{flushleft}
\begin{minipage}{0.90\textwidth}
\begin{tabularx}{\textwidth}{@{}>{\hfil$\displaystyle}l<{$\hfil}@{\quad}X@{}}
\toprule
記号 & 定義・役割 \\
\midrule
i & 銘柄コード \\
\beta_{\text{earn},i,t}^{(1)} & day±1 固定決算係数 \\
\beta_{\text{event},i,t}^{\text{prev}} & 直前フェーズ(weekday 等)出力 \\
\beta_{\text{event},i,t}^{(1)} & earn 系フェーズ 1 出力 \\
\bottomrule
\end{tabularx}
\end{minipage}
\end{flushleft}
\bigskip
%===============================================================================

\clearpage

%-------------------------------------------------------------------------------
% center_shift/sigma/phase2.tex   v1.3  (2025-06-02)
%-------------------------------------------------------------------------------
% CHANGELOG  -- new entry on top
% - 2025-06-02  v1.3 : add section/subsection, hints, ASCII-only
%-------------------------------------------------------------------------------

%=== center_shift =============================================================
\section*{center\_shift}\nopagebreak[4]

%--- sigma ---------------------------------------------------------------------
\subsection*{sigma}\nopagebreak[4]

%--- Phase 2 : 自己適応 λ_shift 更新 -------------------------------------------
\subsubsection*{Phase 2:自己適応 $\lambda_{\text{shift}}$ 更新}\nopagebreak[4]
%────────────────────────────────────
\paragraph{ステップ/目的}
\begin{flushleft}
\begin{enumerate}
  \item \textbf{誤差系列}\;
        \(e_{t-k}=\Delta Cl_{t-k}^{2}-\sigma_{t-k}^{2}\)
  \item \textbf{局所 MSE}\;
        \(\mathrm{MSE}_t=\dfrac{1}{30}\sum_{k=1}^{30}e_{t-k}^{2}\)
  \item \textbf{勾配近似}\;
        \(g_t\approx-\dfrac{2}{30}\sum_{k=1}^{30}
          e_{t-k}\,\sigma_{t-k}^{2}\)
  \item \textbf{$\lambda_{\text{shift}}$ 更新}\;
        \(\lambda_{\text{shift},t}
          =\operatorname{clip}\bigl(
            \lambda_{\text{shift},t-1}-\eta g_t,\,
            0.90,\,0.98\bigr)\)
  \item \textbf{翌日へ反映}\;
        上式の \(\lambda_{\text{shift},t}\) で  
        \(\sigma_{t+1}^{2}\) を再計算
\end{enumerate}
\end{flushleft}

\subsubsection*{変数のポイント}
\begin{flushleft}
\begin{itemize}
  \item \(\lambda_{\text{shift}}\) は [0.90, 0.98] に制限
  \item \(|g_t|\le10\) でクリップし暴走を防止
\end{itemize}
\end{flushleft}

\subsubsection*{実装ヒント}
\begin{flushleft}
学習率 \(\eta=0.01\) が無難。  
ウォームアップ期間 (30~d) は固定 \(\lambda_{\text{shift}}=0.94\)。
\end{flushleft}

\subsubsection*{追加変数・係数}
\begin{flushleft}
\begin{minipage}{0.90\textwidth}
\begin{tabularx}{\textwidth}{@{}>{\hfil$\displaystyle}l<{$\hfil}@{\quad}X@{}}
\toprule
記号 & 定義・役割 \\
\midrule
\lambda_{\text{shift},t-1} & 前日 EWMA 定数 \\
\lambda_{\text{shift},t}   & 更新後 EWMA 定数 \\
g_t & 勾配近似 \\
\eta & 学習率 (0.01) \\
e_{t-k} & 誤差 \\
\mathrm{MSE}_t & 30~d MSE \\
\bottomrule
\end{tabularx}
\end{minipage}
\end{flushleft}
\bigskip
%===============================================================================
       % Phase-2:σ 比補正
\clearpage

%-------------------------------------------------------------------------------
% event/weekday/phase3.tex   v1.1  (2025-06-02)
%-------------------------------------------------------------------------------
% CHANGELOG  -- newest -> oldest
% - 2025-06-02  v1.1 : beta^{(3)} 表記・集約ロジック明確化
% - 2025-05-31  v1.0 : weekday 系サブフェーズ集約
%-------------------------------------------------------------------------------

%=== Phase 3 : weekday 系集約 ===================================================
\section*{event / weekday / Phase 3}\nopagebreak[4]
%────────────────────────────────────
\subsection*{ステップ・目的}
\begin{flushleft}
\begin{enumerate}
  \item \textbf{holiday 側最終係数を取り込み}\;
        \verb|%-------------------------------------------------------------------------------
% event/weekday/phase3.tex   v1.1  (2025-06-02)
%-------------------------------------------------------------------------------
% CHANGELOG  -- newest -> oldest
% - 2025-06-02  v1.1 : beta^{(3)} 表記・集約ロジック明確化
% - 2025-05-31  v1.0 : weekday 系サブフェーズ集約
%-------------------------------------------------------------------------------

%=== Phase 3 : weekday 系集約 ===================================================
\section*{event / weekday / Phase 3}\nopagebreak[4]
%────────────────────────────────────
\subsection*{ステップ・目的}
\begin{flushleft}
\begin{enumerate}
  \item \textbf{holiday 側最終係数を取り込み}\;
        \verb|%-------------------------------------------------------------------------------
% event/weekday/phase3.tex   v1.1  (2025-06-02)
%-------------------------------------------------------------------------------
% CHANGELOG  -- newest -> oldest
% - 2025-06-02  v1.1 : beta^{(3)} 表記・集約ロジック明確化
% - 2025-05-31  v1.0 : weekday 系サブフェーズ集約
%-------------------------------------------------------------------------------

%=== Phase 3 : weekday 系集約 ===================================================
\section*{event / weekday / Phase 3}\nopagebreak[4]
%────────────────────────────────────
\subsection*{ステップ・目的}
\begin{flushleft}
\begin{enumerate}
  \item \textbf{holiday 側最終係数を取り込み}\;
        \verb|\input{event/weekday/holiday/phase3}| で  
        \(\tilde\beta_{\text{weekday},i,t}\) を取得。
  \item \textbf{最終 weekday 係数を宣言}\;
        \[
          \boxed{\beta_{\text{weekday},i,t}^{(3)}
          =\tilde\beta_{\text{weekday},i,t}}
        \]
  \item \textbf{イベント係数パイプラインへ出力}\;
        event/phase0.tex が  
        \(\beta_{\text{weekday},i,t}^{(3)}\) を利用。
\end{enumerate}
\end{flushleft}

\subsection*{追加変数・係数}
\begin{flushleft}
\begin{minipage}{0.88\textwidth}
\begin{tabularx}{\textwidth}{@{}>{\hfil$\displaystyle}l<{$\hfil}@{\quad}X@{}}
\toprule
記号 & 定義・役割 \\
\midrule
\tilde\beta_{\text{weekday},i,t} & holiday/phase3 出力係数 \\
\beta_{\text{weekday},i,t}^{(3)} & weekday 系最終係数 (本フェーズ) \\
\bottomrule
\end{tabularx}
\end{minipage}
\end{flushleft}
\bigskip
%===============================================================================
| で  
        \(\tilde\beta_{\text{weekday},i,t}\) を取得。
  \item \textbf{最終 weekday 係数を宣言}\;
        \[
          \boxed{\beta_{\text{weekday},i,t}^{(3)}
          =\tilde\beta_{\text{weekday},i,t}}
        \]
  \item \textbf{イベント係数パイプラインへ出力}\;
        event/phase0.tex が  
        \(\beta_{\text{weekday},i,t}^{(3)}\) を利用。
\end{enumerate}
\end{flushleft}

\subsection*{追加変数・係数}
\begin{flushleft}
\begin{minipage}{0.88\textwidth}
\begin{tabularx}{\textwidth}{@{}>{\hfil$\displaystyle}l<{$\hfil}@{\quad}X@{}}
\toprule
記号 & 定義・役割 \\
\midrule
\tilde\beta_{\text{weekday},i,t} & holiday/phase3 出力係数 \\
\beta_{\text{weekday},i,t}^{(3)} & weekday 系最終係数 (本フェーズ) \\
\bottomrule
\end{tabularx}
\end{minipage}
\end{flushleft}
\bigskip
%===============================================================================
| で  
        \(\tilde\beta_{\text{weekday},i,t}\) を取得。
  \item \textbf{最終 weekday 係数を宣言}\;
        \[
          \boxed{\beta_{\text{weekday},i,t}^{(3)}
          =\tilde\beta_{\text{weekday},i,t}}
        \]
  \item \textbf{イベント係数パイプラインへ出力}\;
        event/phase0.tex が  
        \(\beta_{\text{weekday},i,t}^{(3)}\) を利用。
\end{enumerate}
\end{flushleft}

\subsection*{追加変数・係数}
\begin{flushleft}
\begin{minipage}{0.88\textwidth}
\begin{tabularx}{\textwidth}{@{}>{\hfil$\displaystyle}l<{$\hfil}@{\quad}X@{}}
\toprule
記号 & 定義・役割 \\
\midrule
\tilde\beta_{\text{weekday},i,t} & holiday/phase3 出力係数 \\
\beta_{\text{weekday},i,t}^{(3)} & weekday 系最終係数 (本フェーズ) \\
\bottomrule
\end{tabularx}
\end{minipage}
\end{flushleft}
\bigskip
%===============================================================================
       % Phase-3:\alpha_t 深掘り
\clearpage

%-------------------------------------------------------------------------------
% center_shift/phase4.tex   v1.0  (2025-06-06)
%-------------------------------------------------------------------------------
% CHANGELOG  -- new entry on top (latest -> oldest)
% - 2025-06-06  v1.0 : 初版
%-------------------------------------------------------------------------------

%=== center_shift =============================================================
\section*{center\_shift}\nopagebreak[4]

%=== Phase 4 : \eta / \lambda の深掘り ======================================
\subsection*{Phase 4:$\eta$ と $\lambda$ の深掘り}\nopagebreak[4]
%────────────────────────────────────
\paragraph{ステップ/目的}
\begin{flushleft}
\begin{enumerate}
  \item \textbf{学習率}
        \(\eta\) は $\lambda_{\text{shift}}$ 更新の歩幅を制御
  \item \textbf{勾配近似}
        \(g_t\approx-\dfrac{2}{30}\sum_{k=1}^{30}e_{t-k}\,\sigma_{t-k}^2\)
  \item \textbf{$\lambda_{\text{shift}}$ 更新}
        \(\lambda_{\text{shift},t}
          =\operatorname{clip}\bigl(\lambda_{\text{shift},t-1}
          -\eta\,g_t,\,0.90,\,0.98\bigr)\)
  \item \textbf{ウォームアップ}
        初期 30~d は固定 $\lambda_{\text{shift}}=0.94$ で安定化
\end{enumerate}
\end{flushleft}

\subsubsection*{変数のポイント}
\begin{flushleft}
\begin{itemize}
  \item 大きすぎる $\eta$ は \(\lambda_{\text{shift}}\) を振動させる
  \item 小さすぎる $\eta$ では収束が遅延
  \item 更新範囲 [0.90, 0.98] を超えないよう \(\operatorname{clip}\)
  \item $|g_t|>10$ なら勾配をクリップし安定化
\end{itemize}
\end{flushleft}

\subsubsection*{実装ヒント}
\begin{flushleft}
\begin{itemize}
  \item 経験的に $\eta=0.01$ が妥当な上限値
  \item 週次で $\eta$ の微調整を試し、予測 MAE を観察
  \item 勾配計算には 30~d の誤差系列を用意
\end{itemize}
\end{flushleft}

\subsubsection*{追加変数・係数}
\begin{flushleft}
\begin{minipage}{0.90\textwidth}
\begin{tabularx}{\textwidth}{@{}>{\hfil$\displaystyle}l<{$\hfil}@{\quad}X@{}}
\toprule
記号 & 定義・役割 \\
\midrule
\eta & 学習率 \\
\lambda_{\text{shift},t} & 更新後 EWMA 定数 \\
\lambda_{\text{shift},t-1} & 前日 EWMA 定数 \\
\sigma_t^2 & 分散推定値 \\
\operatorname{clip} & 範囲制限関数 \\
\end{tabularx}
\end{minipage}
\end{flushleft}
\bigskip
%==============================================================================
       % Phase-4:$\eta$/$\lambda$ 深掘り
\clearpage

%-------------------------------------------------------------------------------
% event/phase0.tex   v1.4  (2025-06-02)
%-------------------------------------------------------------------------------
% CHANGELOG  -- newest -> oldest
% - 2025-06-02  v1.4 : section 階層見直し・ASCII 化・beta^{(3)} 表記へ統一
% - 2025-06-02  v1.3 : CHANGELOG 復元・整形を明記(rules.md 準拠)
% - 2025-05-31  v1.2 : fixed tabularx preamble to 3 columns (l X l)
% - 2025-05-31  v1.1 : \beta_event,i,t = \beta_weekday \times \beta_earn \times \beta_market
% - 2025-05-31  v1.0 : \beta_event,t = 1.0 fallback
%-------------------------------------------------------------------------------

%=== Phase 0 : イベント係数 基本定義 ============================================
\section*{event / Phase 0 : 基本定義}\nopagebreak[4]
%────────────────────────────────────
\begin{flushleft}
\begin{flalign*}
&\text{イベント係数(銘柄 }i\text{)}\quad
  \boxed{%
    \beta_{\text{event},i,t}
      =\beta_{\text{weekday},i,t}^{(3)}\,
       \beta_{\text{earn},i,t}\,
       \beta_{\text{market},i,t}
  } &&\\[6pt]
\end{flalign*}
\end{flushleft}

\subsection*{因子の役割}
\begin{flushleft}
\begin{minipage}{0.92\textwidth}
\begin{tabularx}{\textwidth}{@{}>{\hfil$\displaystyle}l<{$\hfil}@{\quad}X@{\quad}l@{}}
\toprule
因子 & 定義・データソース & 既定レンジ \\
\midrule
\beta_{\text{weekday},i,t}^{(3)} & 曜日+祝日+平滑済み最終係数 & 0.8--1.2 \\
\beta_{\text{earn},i,t}          & 決算ラグ・内容反映係数        & 0.8--1.5 \\
\beta_{\text{market},i,t}        & 指標相関係数                 & 0.8--1.2 \\
\bottomrule
\end{tabularx}
\end{minipage}
\end{flushleft}

\subsection*{備考}
\begin{flushleft}
\begin{itemize}
  \item 欠損時は 1.0 にフォールバック。  
  \item 係数更新は weekday / earn / market サブディレクトリで実施。  
\end{itemize}
\end{flushleft}
\bigskip
%===============================================================================
             % Phase-0:半レンジ m_t
\clearpage

%-------------------------------------------------------------------------------
% event/earn/phase1.tex   v1.1  (2025-06-02)
%-------------------------------------------------------------------------------
% CHANGELOG  -- newest -> oldest
% - 2025-06-02  v1.1 : U+2212→ASCII "-", beta^{(1)} 表記, header tidy
% - 2025-05-31  v1.0 : 初版(決算 day±1 固定係数)
%-------------------------------------------------------------------------------

%=== Phase 1 : 決算係数(day±1) ==============================================
\section*{event / earn / Phase 1}\nopagebreak[4]
%────────────────────────────────────
\subsection*{ステップ・目的}
\begin{flushleft}
\begin{enumerate}
  \item \textbf{決算カレンダーでラグ判定}\;
        day -1(前日)/day 0(当日)/day +1(翌営業日)を抽出。
  \item \textbf{係数決定}\;
        \[
          \beta_{\text{earn},i,t}^{(1)}=
          \begin{cases}
            1.15 & (\text{day\,-1})\\
            1.20 & (\text{day\,0})\\
            1.10 & (\text{day\,+1})\\
            1.00 & (\text{otherwise})
          \end{cases}
        \]
  \item \textbf{イベント係数更新}\;
        \[
          \beta_{\text{event},i,t}^{(1)}
            =\beta_{\text{event},i,t}^{\text{prev}}
             \,\beta_{\text{earn},i,t}^{(1)},
          \quad 0.80 \le \beta_{\text{event},i,t}^{(1)} \le 1.50
        \]
\end{enumerate}
\end{flushleft}

\subsection*{追加変数・係数}
\begin{flushleft}
\begin{minipage}{0.90\textwidth}
\begin{tabularx}{\textwidth}{@{}>{\hfil$\displaystyle}l<{$\hfil}@{\quad}X@{}}
\toprule
記号 & 定義・役割 \\
\midrule
i & 銘柄コード \\
\beta_{\text{earn},i,t}^{(1)} & day±1 固定決算係数 \\
\beta_{\text{event},i,t}^{\text{prev}} & 直前フェーズ(weekday 等)出力 \\
\beta_{\text{event},i,t}^{(1)} & earn 系フェーズ 1 出力 \\
\bottomrule
\end{tabularx}
\end{minipage}
\end{flushleft}
\bigskip
%===============================================================================
             % Phase-1:σ 比補正
\clearpage

%-------------------------------------------------------------------------------
% center_shift/sigma/phase2.tex   v1.3  (2025-06-02)
%-------------------------------------------------------------------------------
% CHANGELOG  -- new entry on top
% - 2025-06-02  v1.3 : add section/subsection, hints, ASCII-only
%-------------------------------------------------------------------------------

%=== center_shift =============================================================
\section*{center\_shift}\nopagebreak[4]

%--- sigma ---------------------------------------------------------------------
\subsection*{sigma}\nopagebreak[4]

%--- Phase 2 : 自己適応 λ_shift 更新 -------------------------------------------
\subsubsection*{Phase 2:自己適応 $\lambda_{\text{shift}}$ 更新}\nopagebreak[4]
%────────────────────────────────────
\paragraph{ステップ/目的}
\begin{flushleft}
\begin{enumerate}
  \item \textbf{誤差系列}\;
        \(e_{t-k}=\Delta Cl_{t-k}^{2}-\sigma_{t-k}^{2}\)
  \item \textbf{局所 MSE}\;
        \(\mathrm{MSE}_t=\dfrac{1}{30}\sum_{k=1}^{30}e_{t-k}^{2}\)
  \item \textbf{勾配近似}\;
        \(g_t\approx-\dfrac{2}{30}\sum_{k=1}^{30}
          e_{t-k}\,\sigma_{t-k}^{2}\)
  \item \textbf{$\lambda_{\text{shift}}$ 更新}\;
        \(\lambda_{\text{shift},t}
          =\operatorname{clip}\bigl(
            \lambda_{\text{shift},t-1}-\eta g_t,\,
            0.90,\,0.98\bigr)\)
  \item \textbf{翌日へ反映}\;
        上式の \(\lambda_{\text{shift},t}\) で  
        \(\sigma_{t+1}^{2}\) を再計算
\end{enumerate}
\end{flushleft}

\subsubsection*{変数のポイント}
\begin{flushleft}
\begin{itemize}
  \item \(\lambda_{\text{shift}}\) は [0.90, 0.98] に制限
  \item \(|g_t|\le10\) でクリップし暴走を防止
\end{itemize}
\end{flushleft}

\subsubsection*{実装ヒント}
\begin{flushleft}
学習率 \(\eta=0.01\) が無難。  
ウォームアップ期間 (30~d) は固定 \(\lambda_{\text{shift}}=0.94\)。
\end{flushleft}

\subsubsection*{追加変数・係数}
\begin{flushleft}
\begin{minipage}{0.90\textwidth}
\begin{tabularx}{\textwidth}{@{}>{\hfil$\displaystyle}l<{$\hfil}@{\quad}X@{}}
\toprule
記号 & 定義・役割 \\
\midrule
\lambda_{\text{shift},t-1} & 前日 EWMA 定数 \\
\lambda_{\text{shift},t}   & 更新後 EWMA 定数 \\
g_t & 勾配近似 \\
\eta & 学習率 (0.01) \\
e_{t-k} & 誤差 \\
\mathrm{MSE}_t & 30~d MSE \\
\bottomrule
\end{tabularx}
\end{minipage}
\end{flushleft}
\bigskip
%===============================================================================
             % Phase-2:ボラ依存補正
\clearpage

%-------------------------------------------------------------------------------
% event/weekday/phase3.tex   v1.1  (2025-06-02)
%-------------------------------------------------------------------------------
% CHANGELOG  -- newest -> oldest
% - 2025-06-02  v1.1 : beta^{(3)} 表記・集約ロジック明確化
% - 2025-05-31  v1.0 : weekday 系サブフェーズ集約
%-------------------------------------------------------------------------------

%=== Phase 3 : weekday 系集約 ===================================================
\section*{event / weekday / Phase 3}\nopagebreak[4]
%────────────────────────────────────
\subsection*{ステップ・目的}
\begin{flushleft}
\begin{enumerate}
  \item \textbf{holiday 側最終係数を取り込み}\;
        \verb|%-------------------------------------------------------------------------------
% event/weekday/phase3.tex   v1.1  (2025-06-02)
%-------------------------------------------------------------------------------
% CHANGELOG  -- newest -> oldest
% - 2025-06-02  v1.1 : beta^{(3)} 表記・集約ロジック明確化
% - 2025-05-31  v1.0 : weekday 系サブフェーズ集約
%-------------------------------------------------------------------------------

%=== Phase 3 : weekday 系集約 ===================================================
\section*{event / weekday / Phase 3}\nopagebreak[4]
%────────────────────────────────────
\subsection*{ステップ・目的}
\begin{flushleft}
\begin{enumerate}
  \item \textbf{holiday 側最終係数を取り込み}\;
        \verb|%-------------------------------------------------------------------------------
% event/weekday/phase3.tex   v1.1  (2025-06-02)
%-------------------------------------------------------------------------------
% CHANGELOG  -- newest -> oldest
% - 2025-06-02  v1.1 : beta^{(3)} 表記・集約ロジック明確化
% - 2025-05-31  v1.0 : weekday 系サブフェーズ集約
%-------------------------------------------------------------------------------

%=== Phase 3 : weekday 系集約 ===================================================
\section*{event / weekday / Phase 3}\nopagebreak[4]
%────────────────────────────────────
\subsection*{ステップ・目的}
\begin{flushleft}
\begin{enumerate}
  \item \textbf{holiday 側最終係数を取り込み}\;
        \verb|\input{event/weekday/holiday/phase3}| で  
        \(\tilde\beta_{\text{weekday},i,t}\) を取得。
  \item \textbf{最終 weekday 係数を宣言}\;
        \[
          \boxed{\beta_{\text{weekday},i,t}^{(3)}
          =\tilde\beta_{\text{weekday},i,t}}
        \]
  \item \textbf{イベント係数パイプラインへ出力}\;
        event/phase0.tex が  
        \(\beta_{\text{weekday},i,t}^{(3)}\) を利用。
\end{enumerate}
\end{flushleft}

\subsection*{追加変数・係数}
\begin{flushleft}
\begin{minipage}{0.88\textwidth}
\begin{tabularx}{\textwidth}{@{}>{\hfil$\displaystyle}l<{$\hfil}@{\quad}X@{}}
\toprule
記号 & 定義・役割 \\
\midrule
\tilde\beta_{\text{weekday},i,t} & holiday/phase3 出力係数 \\
\beta_{\text{weekday},i,t}^{(3)} & weekday 系最終係数 (本フェーズ) \\
\bottomrule
\end{tabularx}
\end{minipage}
\end{flushleft}
\bigskip
%===============================================================================
| で  
        \(\tilde\beta_{\text{weekday},i,t}\) を取得。
  \item \textbf{最終 weekday 係数を宣言}\;
        \[
          \boxed{\beta_{\text{weekday},i,t}^{(3)}
          =\tilde\beta_{\text{weekday},i,t}}
        \]
  \item \textbf{イベント係数パイプラインへ出力}\;
        event/phase0.tex が  
        \(\beta_{\text{weekday},i,t}^{(3)}\) を利用。
\end{enumerate}
\end{flushleft}

\subsection*{追加変数・係数}
\begin{flushleft}
\begin{minipage}{0.88\textwidth}
\begin{tabularx}{\textwidth}{@{}>{\hfil$\displaystyle}l<{$\hfil}@{\quad}X@{}}
\toprule
記号 & 定義・役割 \\
\midrule
\tilde\beta_{\text{weekday},i,t} & holiday/phase3 出力係数 \\
\beta_{\text{weekday},i,t}^{(3)} & weekday 系最終係数 (本フェーズ) \\
\bottomrule
\end{tabularx}
\end{minipage}
\end{flushleft}
\bigskip
%===============================================================================
| で  
        \(\tilde\beta_{\text{weekday},i,t}\) を取得。
  \item \textbf{最終 weekday 係数を宣言}\;
        \[
          \boxed{\beta_{\text{weekday},i,t}^{(3)}
          =\tilde\beta_{\text{weekday},i,t}}
        \]
  \item \textbf{イベント係数パイプラインへ出力}\;
        event/phase0.tex が  
        \(\beta_{\text{weekday},i,t}^{(3)}\) を利用。
\end{enumerate}
\end{flushleft}

\subsection*{追加変数・係数}
\begin{flushleft}
\begin{minipage}{0.88\textwidth}
\begin{tabularx}{\textwidth}{@{}>{\hfil$\displaystyle}l<{$\hfil}@{\quad}X@{}}
\toprule
記号 & 定義・役割 \\
\midrule
\tilde\beta_{\text{weekday},i,t} & holiday/phase3 出力係数 \\
\beta_{\text{weekday},i,t}^{(3)} & weekday 系最終係数 (本フェーズ) \\
\bottomrule
\end{tabularx}
\end{minipage}
\end{flushleft}
\bigskip
%===============================================================================
             % Phase-3:イベント/曜日バイアス
\clearpage

%-------------------------------------------------------------------------------
% center_shift/phase4.tex   v1.0  (2025-06-06)
%-------------------------------------------------------------------------------
% CHANGELOG  -- new entry on top (latest -> oldest)
% - 2025-06-06  v1.0 : 初版
%-------------------------------------------------------------------------------

%=== center_shift =============================================================
\section*{center\_shift}\nopagebreak[4]

%=== Phase 4 : \eta / \lambda の深掘り ======================================
\subsection*{Phase 4:$\eta$ と $\lambda$ の深掘り}\nopagebreak[4]
%────────────────────────────────────
\paragraph{ステップ/目的}
\begin{flushleft}
\begin{enumerate}
  \item \textbf{学習率}
        \(\eta\) は $\lambda_{\text{shift}}$ 更新の歩幅を制御
  \item \textbf{勾配近似}
        \(g_t\approx-\dfrac{2}{30}\sum_{k=1}^{30}e_{t-k}\,\sigma_{t-k}^2\)
  \item \textbf{$\lambda_{\text{shift}}$ 更新}
        \(\lambda_{\text{shift},t}
          =\operatorname{clip}\bigl(\lambda_{\text{shift},t-1}
          -\eta\,g_t,\,0.90,\,0.98\bigr)\)
  \item \textbf{ウォームアップ}
        初期 30~d は固定 $\lambda_{\text{shift}}=0.94$ で安定化
\end{enumerate}
\end{flushleft}

\subsubsection*{変数のポイント}
\begin{flushleft}
\begin{itemize}
  \item 大きすぎる $\eta$ は \(\lambda_{\text{shift}}\) を振動させる
  \item 小さすぎる $\eta$ では収束が遅延
  \item 更新範囲 [0.90, 0.98] を超えないよう \(\operatorname{clip}\)
  \item $|g_t|>10$ なら勾配をクリップし安定化
\end{itemize}
\end{flushleft}

\subsubsection*{実装ヒント}
\begin{flushleft}
\begin{itemize}
  \item 経験的に $\eta=0.01$ が妥当な上限値
  \item 週次で $\eta$ の微調整を試し、予測 MAE を観察
  \item 勾配計算には 30~d の誤差系列を用意
\end{itemize}
\end{flushleft}

\subsubsection*{追加変数・係数}
\begin{flushleft}
\begin{minipage}{0.90\textwidth}
\begin{tabularx}{\textwidth}{@{}>{\hfil$\displaystyle}l<{$\hfil}@{\quad}X@{}}
\toprule
記号 & 定義・役割 \\
\midrule
\eta & 学習率 \\
\lambda_{\text{shift},t} & 更新後 EWMA 定数 \\
\lambda_{\text{shift},t-1} & 前日 EWMA 定数 \\
\sigma_t^2 & 分散推定値 \\
\operatorname{clip} & 範囲制限関数 \\
\end{tabularx}
\end{minipage}
\end{flushleft}
\bigskip
%==============================================================================
             % Phase-4:自己適応 λ_vol 更新
\clearpage

%-------------------------------------------------------------------------------
% event/earn/phase5.tex   v1.1  (2025-06-02)
%-------------------------------------------------------------------------------
% CHANGELOG  -- newest -> oldest
% - 2025-06-02  v1.1 : ASCII 統一, beta^{final} 表記, clip 修正
% - 2025-05-31  v1.0 : 初版(Bayes 縮小)
%-------------------------------------------------------------------------------

%=== Phase 5 : w_profit ベイズ縮小 =============================================
\section*{event / earn / Phase 5}\nopagebreak[4]
%────────────────────────────────────
\subsection*{ステップ・目的}
\begin{flushleft}
\begin{enumerate}
  \item \textbf{サンプル数取得}\;
        \( n_i=\text{count\_earnings}(i,\text{last 3Y}) \)

  \item \textbf{セクター平均重み}\;
        \( \bar w_{\text{profit},s}=\operatorname{mean}(w_{\text{profit},j}) \)

  \item \textbf{Bayes 縮小}\;
        \[
          \tilde w_{\text{profit},i}
            =\frac{n_i}{n_i+\tau}\,w_{\text{profit},i}
             +\frac{\tau}{n_i+\tau}\,\bar w_{\text{profit},s},
          \quad \tau = 10
        \]
        \( \tilde w_{\text{profit},i}=\operatorname{clip}(\tilde w_{\text{profit},i},0.50,0.90) \)

  \item \textbf{サプライズ率再計算} → $\beta_{\text{earn},i,t}^{(5)}$ を取得。

  \item \textbf{イベント係数最終更新}\;
        \[
          \beta_{\text{event},i,t}^{\text{final}}
            =\beta_{\text{event},i,t}^{(4)}\,
             \beta_{\text{earn},i,t}^{(5)}
        \]
\end{enumerate}
\end{flushleft}

\subsection*{追加変数・係数}
\begin{flushleft}
\begin{minipage}{0.92\textwidth}
\begin{tabularx}{\textwidth}{@{}>{\hfil$\displaystyle}l<{$\hfil}@{\quad}X@{}}
\toprule
記号 & 定義・役割 \\
\midrule
n_i & 過去 3 年の決算サンプル数 \\
\tau & 縮小ハイパーパラメータ (10) \\
\bar w_{\text{profit},s} & セクター平均利益重み \\
\beta_{\text{event},i,t}^{\text{final}} & earn 系最終係数 \\
\bottomrule
\end{tabularx}
\end{minipage}
\end{flushleft}
\bigskip
%===============================================================================
             % Phase-5:残差補正
\clearpage

%-------------------------------------------------------------------------------
% event/phase0.tex   v1.4  (2025-06-02)
%-------------------------------------------------------------------------------
% CHANGELOG  -- newest -> oldest
% - 2025-06-02  v1.4 : section 階層見直し・ASCII 化・beta^{(3)} 表記へ統一
% - 2025-06-02  v1.3 : CHANGELOG 復元・整形を明記(rules.md 準拠)
% - 2025-05-31  v1.2 : fixed tabularx preamble to 3 columns (l X l)
% - 2025-05-31  v1.1 : \beta_event,i,t = \beta_weekday \times \beta_earn \times \beta_market
% - 2025-05-31  v1.0 : \beta_event,t = 1.0 fallback
%-------------------------------------------------------------------------------

%=== Phase 0 : イベント係数 基本定義 ============================================
\section*{event / Phase 0 : 基本定義}\nopagebreak[4]
%────────────────────────────────────
\begin{flushleft}
\begin{flalign*}
&\text{イベント係数(銘柄 }i\text{)}\quad
  \boxed{%
    \beta_{\text{event},i,t}
      =\beta_{\text{weekday},i,t}^{(3)}\,
       \beta_{\text{earn},i,t}\,
       \beta_{\text{market},i,t}
  } &&\\[6pt]
\end{flalign*}
\end{flushleft}

\subsection*{因子の役割}
\begin{flushleft}
\begin{minipage}{0.92\textwidth}
\begin{tabularx}{\textwidth}{@{}>{\hfil$\displaystyle}l<{$\hfil}@{\quad}X@{\quad}l@{}}
\toprule
因子 & 定義・データソース & 既定レンジ \\
\midrule
\beta_{\text{weekday},i,t}^{(3)} & 曜日+祝日+平滑済み最終係数 & 0.8--1.2 \\
\beta_{\text{earn},i,t}          & 決算ラグ・内容反映係数        & 0.8--1.5 \\
\beta_{\text{market},i,t}        & 指標相関係数                 & 0.8--1.2 \\
\bottomrule
\end{tabularx}
\end{minipage}
\end{flushleft}

\subsection*{備考}
\begin{flushleft}
\begin{itemize}
  \item 欠損時は 1.0 にフォールバック。  
  \item 係数更新は weekday / earn / market サブディレクトリで実施。  
\end{itemize}
\end{flushleft}
\bigskip
%===============================================================================

\clearpage
%-------------------------------------------------------------------------------
% event/earn/phase1.tex   v1.1  (2025-06-02)
%-------------------------------------------------------------------------------
% CHANGELOG  -- newest -> oldest
% - 2025-06-02  v1.1 : U+2212→ASCII "-", beta^{(1)} 表記, header tidy
% - 2025-05-31  v1.0 : 初版(決算 day±1 固定係数)
%-------------------------------------------------------------------------------

%=== Phase 1 : 決算係数(day±1) ==============================================
\section*{event / earn / Phase 1}\nopagebreak[4]
%────────────────────────────────────
\subsection*{ステップ・目的}
\begin{flushleft}
\begin{enumerate}
  \item \textbf{決算カレンダーでラグ判定}\;
        day -1(前日)/day 0(当日)/day +1(翌営業日)を抽出。
  \item \textbf{係数決定}\;
        \[
          \beta_{\text{earn},i,t}^{(1)}=
          \begin{cases}
            1.15 & (\text{day\,-1})\\
            1.20 & (\text{day\,0})\\
            1.10 & (\text{day\,+1})\\
            1.00 & (\text{otherwise})
          \end{cases}
        \]
  \item \textbf{イベント係数更新}\;
        \[
          \beta_{\text{event},i,t}^{(1)}
            =\beta_{\text{event},i,t}^{\text{prev}}
             \,\beta_{\text{earn},i,t}^{(1)},
          \quad 0.80 \le \beta_{\text{event},i,t}^{(1)} \le 1.50
        \]
\end{enumerate}
\end{flushleft}

\subsection*{追加変数・係数}
\begin{flushleft}
\begin{minipage}{0.90\textwidth}
\begin{tabularx}{\textwidth}{@{}>{\hfil$\displaystyle}l<{$\hfil}@{\quad}X@{}}
\toprule
記号 & 定義・役割 \\
\midrule
i & 銘柄コード \\
\beta_{\text{earn},i,t}^{(1)} & day±1 固定決算係数 \\
\beta_{\text{event},i,t}^{\text{prev}} & 直前フェーズ(weekday 等)出力 \\
\beta_{\text{event},i,t}^{(1)} & earn 系フェーズ 1 出力 \\
\bottomrule
\end{tabularx}
\end{minipage}
\end{flushleft}
\bigskip
%===============================================================================

\clearpage

%-------------------------------------------------------------------------------
% event/earn/phase1.tex   v1.1  (2025-06-02)
%-------------------------------------------------------------------------------
% CHANGELOG  -- newest -> oldest
% - 2025-06-02  v1.1 : U+2212→ASCII "-", beta^{(1)} 表記, header tidy
% - 2025-05-31  v1.0 : 初版(決算 day±1 固定係数)
%-------------------------------------------------------------------------------

%=== Phase 1 : 決算係数(day±1) ==============================================
\section*{event / earn / Phase 1}\nopagebreak[4]
%────────────────────────────────────
\subsection*{ステップ・目的}
\begin{flushleft}
\begin{enumerate}
  \item \textbf{決算カレンダーでラグ判定}\;
        day -1(前日)/day 0(当日)/day +1(翌営業日)を抽出。
  \item \textbf{係数決定}\;
        \[
          \beta_{\text{earn},i,t}^{(1)}=
          \begin{cases}
            1.15 & (\text{day\,-1})\\
            1.20 & (\text{day\,0})\\
            1.10 & (\text{day\,+1})\\
            1.00 & (\text{otherwise})
          \end{cases}
        \]
  \item \textbf{イベント係数更新}\;
        \[
          \beta_{\text{event},i,t}^{(1)}
            =\beta_{\text{event},i,t}^{\text{prev}}
             \,\beta_{\text{earn},i,t}^{(1)},
          \quad 0.80 \le \beta_{\text{event},i,t}^{(1)} \le 1.50
        \]
\end{enumerate}
\end{flushleft}

\subsection*{追加変数・係数}
\begin{flushleft}
\begin{minipage}{0.90\textwidth}
\begin{tabularx}{\textwidth}{@{}>{\hfil$\displaystyle}l<{$\hfil}@{\quad}X@{}}
\toprule
記号 & 定義・役割 \\
\midrule
i & 銘柄コード \\
\beta_{\text{earn},i,t}^{(1)} & day±1 固定決算係数 \\
\beta_{\text{event},i,t}^{\text{prev}} & 直前フェーズ(weekday 等)出力 \\
\beta_{\text{event},i,t}^{(1)} & earn 系フェーズ 1 出力 \\
\bottomrule
\end{tabularx}
\end{minipage}
\end{flushleft}
\bigskip
%===============================================================================
       % Phase-1:イベント係数テーブル
\clearpage

%-------------------------------------------------------------------------------
% center_shift/sigma/phase2.tex   v1.3  (2025-06-02)
%-------------------------------------------------------------------------------
% CHANGELOG  -- new entry on top
% - 2025-06-02  v1.3 : add section/subsection, hints, ASCII-only
%-------------------------------------------------------------------------------

%=== center_shift =============================================================
\section*{center\_shift}\nopagebreak[4]

%--- sigma ---------------------------------------------------------------------
\subsection*{sigma}\nopagebreak[4]

%--- Phase 2 : 自己適応 λ_shift 更新 -------------------------------------------
\subsubsection*{Phase 2:自己適応 $\lambda_{\text{shift}}$ 更新}\nopagebreak[4]
%────────────────────────────────────
\paragraph{ステップ/目的}
\begin{flushleft}
\begin{enumerate}
  \item \textbf{誤差系列}\;
        \(e_{t-k}=\Delta Cl_{t-k}^{2}-\sigma_{t-k}^{2}\)
  \item \textbf{局所 MSE}\;
        \(\mathrm{MSE}_t=\dfrac{1}{30}\sum_{k=1}^{30}e_{t-k}^{2}\)
  \item \textbf{勾配近似}\;
        \(g_t\approx-\dfrac{2}{30}\sum_{k=1}^{30}
          e_{t-k}\,\sigma_{t-k}^{2}\)
  \item \textbf{$\lambda_{\text{shift}}$ 更新}\;
        \(\lambda_{\text{shift},t}
          =\operatorname{clip}\bigl(
            \lambda_{\text{shift},t-1}-\eta g_t,\,
            0.90,\,0.98\bigr)\)
  \item \textbf{翌日へ反映}\;
        上式の \(\lambda_{\text{shift},t}\) で  
        \(\sigma_{t+1}^{2}\) を再計算
\end{enumerate}
\end{flushleft}

\subsubsection*{変数のポイント}
\begin{flushleft}
\begin{itemize}
  \item \(\lambda_{\text{shift}}\) は [0.90, 0.98] に制限
  \item \(|g_t|\le10\) でクリップし暴走を防止
\end{itemize}
\end{flushleft}

\subsubsection*{実装ヒント}
\begin{flushleft}
学習率 \(\eta=0.01\) が無難。  
ウォームアップ期間 (30~d) は固定 \(\lambda_{\text{shift}}=0.94\)。
\end{flushleft}

\subsubsection*{追加変数・係数}
\begin{flushleft}
\begin{minipage}{0.90\textwidth}
\begin{tabularx}{\textwidth}{@{}>{\hfil$\displaystyle}l<{$\hfil}@{\quad}X@{}}
\toprule
記号 & 定義・役割 \\
\midrule
\lambda_{\text{shift},t-1} & 前日 EWMA 定数 \\
\lambda_{\text{shift},t}   & 更新後 EWMA 定数 \\
g_t & 勾配近似 \\
\eta & 学習率 (0.01) \\
e_{t-k} & 誤差 \\
\mathrm{MSE}_t & 30~d MSE \\
\bottomrule
\end{tabularx}
\end{minipage}
\end{flushleft}
\bigskip
%===============================================================================
       % Phase-2:イベント係数テーブル
\clearpage

%-------------------------------------------------------------------------------
% event/earn/phase1.tex   v1.1  (2025-06-02)
%-------------------------------------------------------------------------------
% CHANGELOG  -- newest -> oldest
% - 2025-06-02  v1.1 : U+2212→ASCII "-", beta^{(1)} 表記, header tidy
% - 2025-05-31  v1.0 : 初版(決算 day±1 固定係数)
%-------------------------------------------------------------------------------

%=== Phase 1 : 決算係数(day±1) ==============================================
\section*{event / earn / Phase 1}\nopagebreak[4]
%────────────────────────────────────
\subsection*{ステップ・目的}
\begin{flushleft}
\begin{enumerate}
  \item \textbf{決算カレンダーでラグ判定}\;
        day -1(前日)/day 0(当日)/day +1(翌営業日)を抽出。
  \item \textbf{係数決定}\;
        \[
          \beta_{\text{earn},i,t}^{(1)}=
          \begin{cases}
            1.15 & (\text{day\,-1})\\
            1.20 & (\text{day\,0})\\
            1.10 & (\text{day\,+1})\\
            1.00 & (\text{otherwise})
          \end{cases}
        \]
  \item \textbf{イベント係数更新}\;
        \[
          \beta_{\text{event},i,t}^{(1)}
            =\beta_{\text{event},i,t}^{\text{prev}}
             \,\beta_{\text{earn},i,t}^{(1)},
          \quad 0.80 \le \beta_{\text{event},i,t}^{(1)} \le 1.50
        \]
\end{enumerate}
\end{flushleft}

\subsection*{追加変数・係数}
\begin{flushleft}
\begin{minipage}{0.90\textwidth}
\begin{tabularx}{\textwidth}{@{}>{\hfil$\displaystyle}l<{$\hfil}@{\quad}X@{}}
\toprule
記号 & 定義・役割 \\
\midrule
i & 銘柄コード \\
\beta_{\text{earn},i,t}^{(1)} & day±1 固定決算係数 \\
\beta_{\text{event},i,t}^{\text{prev}} & 直前フェーズ(weekday 等)出力 \\
\beta_{\text{event},i,t}^{(1)} & earn 系フェーズ 1 出力 \\
\bottomrule
\end{tabularx}
\end{minipage}
\end{flushleft}
\bigskip
%===============================================================================
 % Phase-1:祝日係数テーブル
\clearpage

%-------------------------------------------------------------------------------
% center_shift/sigma/phase2.tex   v1.3  (2025-06-02)
%-------------------------------------------------------------------------------
% CHANGELOG  -- new entry on top
% - 2025-06-02  v1.3 : add section/subsection, hints, ASCII-only
%-------------------------------------------------------------------------------

%=== center_shift =============================================================
\section*{center\_shift}\nopagebreak[4]

%--- sigma ---------------------------------------------------------------------
\subsection*{sigma}\nopagebreak[4]

%--- Phase 2 : 自己適応 λ_shift 更新 -------------------------------------------
\subsubsection*{Phase 2:自己適応 $\lambda_{\text{shift}}$ 更新}\nopagebreak[4]
%────────────────────────────────────
\paragraph{ステップ/目的}
\begin{flushleft}
\begin{enumerate}
  \item \textbf{誤差系列}\;
        \(e_{t-k}=\Delta Cl_{t-k}^{2}-\sigma_{t-k}^{2}\)
  \item \textbf{局所 MSE}\;
        \(\mathrm{MSE}_t=\dfrac{1}{30}\sum_{k=1}^{30}e_{t-k}^{2}\)
  \item \textbf{勾配近似}\;
        \(g_t\approx-\dfrac{2}{30}\sum_{k=1}^{30}
          e_{t-k}\,\sigma_{t-k}^{2}\)
  \item \textbf{$\lambda_{\text{shift}}$ 更新}\;
        \(\lambda_{\text{shift},t}
          =\operatorname{clip}\bigl(
            \lambda_{\text{shift},t-1}-\eta g_t,\,
            0.90,\,0.98\bigr)\)
  \item \textbf{翌日へ反映}\;
        上式の \(\lambda_{\text{shift},t}\) で  
        \(\sigma_{t+1}^{2}\) を再計算
\end{enumerate}
\end{flushleft}

\subsubsection*{変数のポイント}
\begin{flushleft}
\begin{itemize}
  \item \(\lambda_{\text{shift}}\) は [0.90, 0.98] に制限
  \item \(|g_t|\le10\) でクリップし暴走を防止
\end{itemize}
\end{flushleft}

\subsubsection*{実装ヒント}
\begin{flushleft}
学習率 \(\eta=0.01\) が無難。  
ウォームアップ期間 (30~d) は固定 \(\lambda_{\text{shift}}=0.94\)。
\end{flushleft}

\subsubsection*{追加変数・係数}
\begin{flushleft}
\begin{minipage}{0.90\textwidth}
\begin{tabularx}{\textwidth}{@{}>{\hfil$\displaystyle}l<{$\hfil}@{\quad}X@{}}
\toprule
記号 & 定義・役割 \\
\midrule
\lambda_{\text{shift},t-1} & 前日 EWMA 定数 \\
\lambda_{\text{shift},t}   & 更新後 EWMA 定数 \\
g_t & 勾配近似 \\
\eta & 学習率 (0.01) \\
e_{t-k} & 誤差 \\
\mathrm{MSE}_t & 30~d MSE \\
\bottomrule
\end{tabularx}
\end{minipage}
\end{flushleft}
\bigskip
%===============================================================================
 % Phase-2:祝日係数テーブル
\clearpage

%-------------------------------------------------------------------------------
% event/weekday/phase3.tex   v1.1  (2025-06-02)
%-------------------------------------------------------------------------------
% CHANGELOG  -- newest -> oldest
% - 2025-06-02  v1.1 : beta^{(3)} 表記・集約ロジック明確化
% - 2025-05-31  v1.0 : weekday 系サブフェーズ集約
%-------------------------------------------------------------------------------

%=== Phase 3 : weekday 系集約 ===================================================
\section*{event / weekday / Phase 3}\nopagebreak[4]
%────────────────────────────────────
\subsection*{ステップ・目的}
\begin{flushleft}
\begin{enumerate}
  \item \textbf{holiday 側最終係数を取り込み}\;
        \verb|%-------------------------------------------------------------------------------
% event/weekday/phase3.tex   v1.1  (2025-06-02)
%-------------------------------------------------------------------------------
% CHANGELOG  -- newest -> oldest
% - 2025-06-02  v1.1 : beta^{(3)} 表記・集約ロジック明確化
% - 2025-05-31  v1.0 : weekday 系サブフェーズ集約
%-------------------------------------------------------------------------------

%=== Phase 3 : weekday 系集約 ===================================================
\section*{event / weekday / Phase 3}\nopagebreak[4]
%────────────────────────────────────
\subsection*{ステップ・目的}
\begin{flushleft}
\begin{enumerate}
  \item \textbf{holiday 側最終係数を取り込み}\;
        \verb|%-------------------------------------------------------------------------------
% event/weekday/phase3.tex   v1.1  (2025-06-02)
%-------------------------------------------------------------------------------
% CHANGELOG  -- newest -> oldest
% - 2025-06-02  v1.1 : beta^{(3)} 表記・集約ロジック明確化
% - 2025-05-31  v1.0 : weekday 系サブフェーズ集約
%-------------------------------------------------------------------------------

%=== Phase 3 : weekday 系集約 ===================================================
\section*{event / weekday / Phase 3}\nopagebreak[4]
%────────────────────────────────────
\subsection*{ステップ・目的}
\begin{flushleft}
\begin{enumerate}
  \item \textbf{holiday 側最終係数を取り込み}\;
        \verb|\input{event/weekday/holiday/phase3}| で  
        \(\tilde\beta_{\text{weekday},i,t}\) を取得。
  \item \textbf{最終 weekday 係数を宣言}\;
        \[
          \boxed{\beta_{\text{weekday},i,t}^{(3)}
          =\tilde\beta_{\text{weekday},i,t}}
        \]
  \item \textbf{イベント係数パイプラインへ出力}\;
        event/phase0.tex が  
        \(\beta_{\text{weekday},i,t}^{(3)}\) を利用。
\end{enumerate}
\end{flushleft}

\subsection*{追加変数・係数}
\begin{flushleft}
\begin{minipage}{0.88\textwidth}
\begin{tabularx}{\textwidth}{@{}>{\hfil$\displaystyle}l<{$\hfil}@{\quad}X@{}}
\toprule
記号 & 定義・役割 \\
\midrule
\tilde\beta_{\text{weekday},i,t} & holiday/phase3 出力係数 \\
\beta_{\text{weekday},i,t}^{(3)} & weekday 系最終係数 (本フェーズ) \\
\bottomrule
\end{tabularx}
\end{minipage}
\end{flushleft}
\bigskip
%===============================================================================
| で  
        \(\tilde\beta_{\text{weekday},i,t}\) を取得。
  \item \textbf{最終 weekday 係数を宣言}\;
        \[
          \boxed{\beta_{\text{weekday},i,t}^{(3)}
          =\tilde\beta_{\text{weekday},i,t}}
        \]
  \item \textbf{イベント係数パイプラインへ出力}\;
        event/phase0.tex が  
        \(\beta_{\text{weekday},i,t}^{(3)}\) を利用。
\end{enumerate}
\end{flushleft}

\subsection*{追加変数・係数}
\begin{flushleft}
\begin{minipage}{0.88\textwidth}
\begin{tabularx}{\textwidth}{@{}>{\hfil$\displaystyle}l<{$\hfil}@{\quad}X@{}}
\toprule
記号 & 定義・役割 \\
\midrule
\tilde\beta_{\text{weekday},i,t} & holiday/phase3 出力係数 \\
\beta_{\text{weekday},i,t}^{(3)} & weekday 系最終係数 (本フェーズ) \\
\bottomrule
\end{tabularx}
\end{minipage}
\end{flushleft}
\bigskip
%===============================================================================
| で  
        \(\tilde\beta_{\text{weekday},i,t}\) を取得。
  \item \textbf{最終 weekday 係数を宣言}\;
        \[
          \boxed{\beta_{\text{weekday},i,t}^{(3)}
          =\tilde\beta_{\text{weekday},i,t}}
        \]
  \item \textbf{イベント係数パイプラインへ出力}\;
        event/phase0.tex が  
        \(\beta_{\text{weekday},i,t}^{(3)}\) を利用。
\end{enumerate}
\end{flushleft}

\subsection*{追加変数・係数}
\begin{flushleft}
\begin{minipage}{0.88\textwidth}
\begin{tabularx}{\textwidth}{@{}>{\hfil$\displaystyle}l<{$\hfil}@{\quad}X@{}}
\toprule
記号 & 定義・役割 \\
\midrule
\tilde\beta_{\text{weekday},i,t} & holiday/phase3 出力係数 \\
\beta_{\text{weekday},i,t}^{(3)} & weekday 系最終係数 (本フェーズ) \\
\bottomrule
\end{tabularx}
\end{minipage}
\end{flushleft}
\bigskip
%===============================================================================
 % Phase-3:祝日係数テーブル
\clearpage

%-------------------------------------------------------------------------------
% event/weekday/phase3.tex   v1.1  (2025-06-02)
%-------------------------------------------------------------------------------
% CHANGELOG  -- newest -> oldest
% - 2025-06-02  v1.1 : beta^{(3)} 表記・集約ロジック明確化
% - 2025-05-31  v1.0 : weekday 系サブフェーズ集約
%-------------------------------------------------------------------------------

%=== Phase 3 : weekday 系集約 ===================================================
\section*{event / weekday / Phase 3}\nopagebreak[4]
%────────────────────────────────────
\subsection*{ステップ・目的}
\begin{flushleft}
\begin{enumerate}
  \item \textbf{holiday 側最終係数を取り込み}\;
        \verb|%-------------------------------------------------------------------------------
% event/weekday/phase3.tex   v1.1  (2025-06-02)
%-------------------------------------------------------------------------------
% CHANGELOG  -- newest -> oldest
% - 2025-06-02  v1.1 : beta^{(3)} 表記・集約ロジック明確化
% - 2025-05-31  v1.0 : weekday 系サブフェーズ集約
%-------------------------------------------------------------------------------

%=== Phase 3 : weekday 系集約 ===================================================
\section*{event / weekday / Phase 3}\nopagebreak[4]
%────────────────────────────────────
\subsection*{ステップ・目的}
\begin{flushleft}
\begin{enumerate}
  \item \textbf{holiday 側最終係数を取り込み}\;
        \verb|%-------------------------------------------------------------------------------
% event/weekday/phase3.tex   v1.1  (2025-06-02)
%-------------------------------------------------------------------------------
% CHANGELOG  -- newest -> oldest
% - 2025-06-02  v1.1 : beta^{(3)} 表記・集約ロジック明確化
% - 2025-05-31  v1.0 : weekday 系サブフェーズ集約
%-------------------------------------------------------------------------------

%=== Phase 3 : weekday 系集約 ===================================================
\section*{event / weekday / Phase 3}\nopagebreak[4]
%────────────────────────────────────
\subsection*{ステップ・目的}
\begin{flushleft}
\begin{enumerate}
  \item \textbf{holiday 側最終係数を取り込み}\;
        \verb|\input{event/weekday/holiday/phase3}| で  
        \(\tilde\beta_{\text{weekday},i,t}\) を取得。
  \item \textbf{最終 weekday 係数を宣言}\;
        \[
          \boxed{\beta_{\text{weekday},i,t}^{(3)}
          =\tilde\beta_{\text{weekday},i,t}}
        \]
  \item \textbf{イベント係数パイプラインへ出力}\;
        event/phase0.tex が  
        \(\beta_{\text{weekday},i,t}^{(3)}\) を利用。
\end{enumerate}
\end{flushleft}

\subsection*{追加変数・係数}
\begin{flushleft}
\begin{minipage}{0.88\textwidth}
\begin{tabularx}{\textwidth}{@{}>{\hfil$\displaystyle}l<{$\hfil}@{\quad}X@{}}
\toprule
記号 & 定義・役割 \\
\midrule
\tilde\beta_{\text{weekday},i,t} & holiday/phase3 出力係数 \\
\beta_{\text{weekday},i,t}^{(3)} & weekday 系最終係数 (本フェーズ) \\
\bottomrule
\end{tabularx}
\end{minipage}
\end{flushleft}
\bigskip
%===============================================================================
| で  
        \(\tilde\beta_{\text{weekday},i,t}\) を取得。
  \item \textbf{最終 weekday 係数を宣言}\;
        \[
          \boxed{\beta_{\text{weekday},i,t}^{(3)}
          =\tilde\beta_{\text{weekday},i,t}}
        \]
  \item \textbf{イベント係数パイプラインへ出力}\;
        event/phase0.tex が  
        \(\beta_{\text{weekday},i,t}^{(3)}\) を利用。
\end{enumerate}
\end{flushleft}

\subsection*{追加変数・係数}
\begin{flushleft}
\begin{minipage}{0.88\textwidth}
\begin{tabularx}{\textwidth}{@{}>{\hfil$\displaystyle}l<{$\hfil}@{\quad}X@{}}
\toprule
記号 & 定義・役割 \\
\midrule
\tilde\beta_{\text{weekday},i,t} & holiday/phase3 出力係数 \\
\beta_{\text{weekday},i,t}^{(3)} & weekday 系最終係数 (本フェーズ) \\
\bottomrule
\end{tabularx}
\end{minipage}
\end{flushleft}
\bigskip
%===============================================================================
| で  
        \(\tilde\beta_{\text{weekday},i,t}\) を取得。
  \item \textbf{最終 weekday 係数を宣言}\;
        \[
          \boxed{\beta_{\text{weekday},i,t}^{(3)}
          =\tilde\beta_{\text{weekday},i,t}}
        \]
  \item \textbf{イベント係数パイプラインへ出力}\;
        event/phase0.tex が  
        \(\beta_{\text{weekday},i,t}^{(3)}\) を利用。
\end{enumerate}
\end{flushleft}

\subsection*{追加変数・係数}
\begin{flushleft}
\begin{minipage}{0.88\textwidth}
\begin{tabularx}{\textwidth}{@{}>{\hfil$\displaystyle}l<{$\hfil}@{\quad}X@{}}
\toprule
記号 & 定義・役割 \\
\midrule
\tilde\beta_{\text{weekday},i,t} & holiday/phase3 出力係数 \\
\beta_{\text{weekday},i,t}^{(3)} & weekday 系最終係数 (本フェーズ) \\
\bottomrule
\end{tabularx}
\end{minipage}
\end{flushleft}
\bigskip
%===============================================================================
       % Phase-3:イベント係数テーブル
\clearpage

%-------------------------------------------------------------------------------
% event/earn/phase1.tex   v1.1  (2025-06-02)
%-------------------------------------------------------------------------------
% CHANGELOG  -- newest -> oldest
% - 2025-06-02  v1.1 : U+2212→ASCII "-", beta^{(1)} 表記, header tidy
% - 2025-05-31  v1.0 : 初版(決算 day±1 固定係数)
%-------------------------------------------------------------------------------

%=== Phase 1 : 決算係数(day±1) ==============================================
\section*{event / earn / Phase 1}\nopagebreak[4]
%────────────────────────────────────
\subsection*{ステップ・目的}
\begin{flushleft}
\begin{enumerate}
  \item \textbf{決算カレンダーでラグ判定}\;
        day -1(前日)/day 0(当日)/day +1(翌営業日)を抽出。
  \item \textbf{係数決定}\;
        \[
          \beta_{\text{earn},i,t}^{(1)}=
          \begin{cases}
            1.15 & (\text{day\,-1})\\
            1.20 & (\text{day\,0})\\
            1.10 & (\text{day\,+1})\\
            1.00 & (\text{otherwise})
          \end{cases}
        \]
  \item \textbf{イベント係数更新}\;
        \[
          \beta_{\text{event},i,t}^{(1)}
            =\beta_{\text{event},i,t}^{\text{prev}}
             \,\beta_{\text{earn},i,t}^{(1)},
          \quad 0.80 \le \beta_{\text{event},i,t}^{(1)} \le 1.50
        \]
\end{enumerate}
\end{flushleft}

\subsection*{追加変数・係数}
\begin{flushleft}
\begin{minipage}{0.90\textwidth}
\begin{tabularx}{\textwidth}{@{}>{\hfil$\displaystyle}l<{$\hfil}@{\quad}X@{}}
\toprule
記号 & 定義・役割 \\
\midrule
i & 銘柄コード \\
\beta_{\text{earn},i,t}^{(1)} & day±1 固定決算係数 \\
\beta_{\text{event},i,t}^{\text{prev}} & 直前フェーズ(weekday 等)出力 \\
\beta_{\text{event},i,t}^{(1)} & earn 系フェーズ 1 出力 \\
\bottomrule
\end{tabularx}
\end{minipage}
\end{flushleft}
\bigskip
%===============================================================================
       % Phase-1:イベント係数テーブル
\clearpage

%-------------------------------------------------------------------------------
% center_shift/sigma/phase2.tex   v1.3  (2025-06-02)
%-------------------------------------------------------------------------------
% CHANGELOG  -- new entry on top
% - 2025-06-02  v1.3 : add section/subsection, hints, ASCII-only
%-------------------------------------------------------------------------------

%=== center_shift =============================================================
\section*{center\_shift}\nopagebreak[4]

%--- sigma ---------------------------------------------------------------------
\subsection*{sigma}\nopagebreak[4]

%--- Phase 2 : 自己適応 λ_shift 更新 -------------------------------------------
\subsubsection*{Phase 2:自己適応 $\lambda_{\text{shift}}$ 更新}\nopagebreak[4]
%────────────────────────────────────
\paragraph{ステップ/目的}
\begin{flushleft}
\begin{enumerate}
  \item \textbf{誤差系列}\;
        \(e_{t-k}=\Delta Cl_{t-k}^{2}-\sigma_{t-k}^{2}\)
  \item \textbf{局所 MSE}\;
        \(\mathrm{MSE}_t=\dfrac{1}{30}\sum_{k=1}^{30}e_{t-k}^{2}\)
  \item \textbf{勾配近似}\;
        \(g_t\approx-\dfrac{2}{30}\sum_{k=1}^{30}
          e_{t-k}\,\sigma_{t-k}^{2}\)
  \item \textbf{$\lambda_{\text{shift}}$ 更新}\;
        \(\lambda_{\text{shift},t}
          =\operatorname{clip}\bigl(
            \lambda_{\text{shift},t-1}-\eta g_t,\,
            0.90,\,0.98\bigr)\)
  \item \textbf{翌日へ反映}\;
        上式の \(\lambda_{\text{shift},t}\) で  
        \(\sigma_{t+1}^{2}\) を再計算
\end{enumerate}
\end{flushleft}

\subsubsection*{変数のポイント}
\begin{flushleft}
\begin{itemize}
  \item \(\lambda_{\text{shift}}\) は [0.90, 0.98] に制限
  \item \(|g_t|\le10\) でクリップし暴走を防止
\end{itemize}
\end{flushleft}

\subsubsection*{実装ヒント}
\begin{flushleft}
学習率 \(\eta=0.01\) が無難。  
ウォームアップ期間 (30~d) は固定 \(\lambda_{\text{shift}}=0.94\)。
\end{flushleft}

\subsubsection*{追加変数・係数}
\begin{flushleft}
\begin{minipage}{0.90\textwidth}
\begin{tabularx}{\textwidth}{@{}>{\hfil$\displaystyle}l<{$\hfil}@{\quad}X@{}}
\toprule
記号 & 定義・役割 \\
\midrule
\lambda_{\text{shift},t-1} & 前日 EWMA 定数 \\
\lambda_{\text{shift},t}   & 更新後 EWMA 定数 \\
g_t & 勾配近似 \\
\eta & 学習率 (0.01) \\
e_{t-k} & 誤差 \\
\mathrm{MSE}_t & 30~d MSE \\
\bottomrule
\end{tabularx}
\end{minipage}
\end{flushleft}
\bigskip
%===============================================================================
       % Phase-2:イベント係数テーブル
\clearpage

%-------------------------------------------------------------------------------
% event/weekday/phase3.tex   v1.1  (2025-06-02)
%-------------------------------------------------------------------------------
% CHANGELOG  -- newest -> oldest
% - 2025-06-02  v1.1 : beta^{(3)} 表記・集約ロジック明確化
% - 2025-05-31  v1.0 : weekday 系サブフェーズ集約
%-------------------------------------------------------------------------------

%=== Phase 3 : weekday 系集約 ===================================================
\section*{event / weekday / Phase 3}\nopagebreak[4]
%────────────────────────────────────
\subsection*{ステップ・目的}
\begin{flushleft}
\begin{enumerate}
  \item \textbf{holiday 側最終係数を取り込み}\;
        \verb|%-------------------------------------------------------------------------------
% event/weekday/phase3.tex   v1.1  (2025-06-02)
%-------------------------------------------------------------------------------
% CHANGELOG  -- newest -> oldest
% - 2025-06-02  v1.1 : beta^{(3)} 表記・集約ロジック明確化
% - 2025-05-31  v1.0 : weekday 系サブフェーズ集約
%-------------------------------------------------------------------------------

%=== Phase 3 : weekday 系集約 ===================================================
\section*{event / weekday / Phase 3}\nopagebreak[4]
%────────────────────────────────────
\subsection*{ステップ・目的}
\begin{flushleft}
\begin{enumerate}
  \item \textbf{holiday 側最終係数を取り込み}\;
        \verb|%-------------------------------------------------------------------------------
% event/weekday/phase3.tex   v1.1  (2025-06-02)
%-------------------------------------------------------------------------------
% CHANGELOG  -- newest -> oldest
% - 2025-06-02  v1.1 : beta^{(3)} 表記・集約ロジック明確化
% - 2025-05-31  v1.0 : weekday 系サブフェーズ集約
%-------------------------------------------------------------------------------

%=== Phase 3 : weekday 系集約 ===================================================
\section*{event / weekday / Phase 3}\nopagebreak[4]
%────────────────────────────────────
\subsection*{ステップ・目的}
\begin{flushleft}
\begin{enumerate}
  \item \textbf{holiday 側最終係数を取り込み}\;
        \verb|\input{event/weekday/holiday/phase3}| で  
        \(\tilde\beta_{\text{weekday},i,t}\) を取得。
  \item \textbf{最終 weekday 係数を宣言}\;
        \[
          \boxed{\beta_{\text{weekday},i,t}^{(3)}
          =\tilde\beta_{\text{weekday},i,t}}
        \]
  \item \textbf{イベント係数パイプラインへ出力}\;
        event/phase0.tex が  
        \(\beta_{\text{weekday},i,t}^{(3)}\) を利用。
\end{enumerate}
\end{flushleft}

\subsection*{追加変数・係数}
\begin{flushleft}
\begin{minipage}{0.88\textwidth}
\begin{tabularx}{\textwidth}{@{}>{\hfil$\displaystyle}l<{$\hfil}@{\quad}X@{}}
\toprule
記号 & 定義・役割 \\
\midrule
\tilde\beta_{\text{weekday},i,t} & holiday/phase3 出力係数 \\
\beta_{\text{weekday},i,t}^{(3)} & weekday 系最終係数 (本フェーズ) \\
\bottomrule
\end{tabularx}
\end{minipage}
\end{flushleft}
\bigskip
%===============================================================================
| で  
        \(\tilde\beta_{\text{weekday},i,t}\) を取得。
  \item \textbf{最終 weekday 係数を宣言}\;
        \[
          \boxed{\beta_{\text{weekday},i,t}^{(3)}
          =\tilde\beta_{\text{weekday},i,t}}
        \]
  \item \textbf{イベント係数パイプラインへ出力}\;
        event/phase0.tex が  
        \(\beta_{\text{weekday},i,t}^{(3)}\) を利用。
\end{enumerate}
\end{flushleft}

\subsection*{追加変数・係数}
\begin{flushleft}
\begin{minipage}{0.88\textwidth}
\begin{tabularx}{\textwidth}{@{}>{\hfil$\displaystyle}l<{$\hfil}@{\quad}X@{}}
\toprule
記号 & 定義・役割 \\
\midrule
\tilde\beta_{\text{weekday},i,t} & holiday/phase3 出力係数 \\
\beta_{\text{weekday},i,t}^{(3)} & weekday 系最終係数 (本フェーズ) \\
\bottomrule
\end{tabularx}
\end{minipage}
\end{flushleft}
\bigskip
%===============================================================================
| で  
        \(\tilde\beta_{\text{weekday},i,t}\) を取得。
  \item \textbf{最終 weekday 係数を宣言}\;
        \[
          \boxed{\beta_{\text{weekday},i,t}^{(3)}
          =\tilde\beta_{\text{weekday},i,t}}
        \]
  \item \textbf{イベント係数パイプラインへ出力}\;
        event/phase0.tex が  
        \(\beta_{\text{weekday},i,t}^{(3)}\) を利用。
\end{enumerate}
\end{flushleft}

\subsection*{追加変数・係数}
\begin{flushleft}
\begin{minipage}{0.88\textwidth}
\begin{tabularx}{\textwidth}{@{}>{\hfil$\displaystyle}l<{$\hfil}@{\quad}X@{}}
\toprule
記号 & 定義・役割 \\
\midrule
\tilde\beta_{\text{weekday},i,t} & holiday/phase3 出力係数 \\
\beta_{\text{weekday},i,t}^{(3)} & weekday 系最終係数 (本フェーズ) \\
\bottomrule
\end{tabularx}
\end{minipage}
\end{flushleft}
\bigskip
%===============================================================================
       % Phase-3:イベント係数テーブル
\clearpage

%-------------------------------------------------------------------------------
% center_shift/phase4.tex   v1.0  (2025-06-06)
%-------------------------------------------------------------------------------
% CHANGELOG  -- new entry on top (latest -> oldest)
% - 2025-06-06  v1.0 : 初版
%-------------------------------------------------------------------------------

%=== center_shift =============================================================
\section*{center\_shift}\nopagebreak[4]

%=== Phase 4 : \eta / \lambda の深掘り ======================================
\subsection*{Phase 4:$\eta$ と $\lambda$ の深掘り}\nopagebreak[4]
%────────────────────────────────────
\paragraph{ステップ/目的}
\begin{flushleft}
\begin{enumerate}
  \item \textbf{学習率}
        \(\eta\) は $\lambda_{\text{shift}}$ 更新の歩幅を制御
  \item \textbf{勾配近似}
        \(g_t\approx-\dfrac{2}{30}\sum_{k=1}^{30}e_{t-k}\,\sigma_{t-k}^2\)
  \item \textbf{$\lambda_{\text{shift}}$ 更新}
        \(\lambda_{\text{shift},t}
          =\operatorname{clip}\bigl(\lambda_{\text{shift},t-1}
          -\eta\,g_t,\,0.90,\,0.98\bigr)\)
  \item \textbf{ウォームアップ}
        初期 30~d は固定 $\lambda_{\text{shift}}=0.94$ で安定化
\end{enumerate}
\end{flushleft}

\subsubsection*{変数のポイント}
\begin{flushleft}
\begin{itemize}
  \item 大きすぎる $\eta$ は \(\lambda_{\text{shift}}\) を振動させる
  \item 小さすぎる $\eta$ では収束が遅延
  \item 更新範囲 [0.90, 0.98] を超えないよう \(\operatorname{clip}\)
  \item $|g_t|>10$ なら勾配をクリップし安定化
\end{itemize}
\end{flushleft}

\subsubsection*{実装ヒント}
\begin{flushleft}
\begin{itemize}
  \item 経験的に $\eta=0.01$ が妥当な上限値
  \item 週次で $\eta$ の微調整を試し、予測 MAE を観察
  \item 勾配計算には 30~d の誤差系列を用意
\end{itemize}
\end{flushleft}

\subsubsection*{追加変数・係数}
\begin{flushleft}
\begin{minipage}{0.90\textwidth}
\begin{tabularx}{\textwidth}{@{}>{\hfil$\displaystyle}l<{$\hfil}@{\quad}X@{}}
\toprule
記号 & 定義・役割 \\
\midrule
\eta & 学習率 \\
\lambda_{\text{shift},t} & 更新後 EWMA 定数 \\
\lambda_{\text{shift},t-1} & 前日 EWMA 定数 \\
\sigma_t^2 & 分散推定値 \\
\operatorname{clip} & 範囲制限関数 \\
\end{tabularx}
\end{minipage}
\end{flushleft}
\bigskip
%==============================================================================
       % Phase-4:イベント係数テーブル
\clearpage

%-------------------------------------------------------------------------------
% event/earn/phase5.tex   v1.1  (2025-06-02)
%-------------------------------------------------------------------------------
% CHANGELOG  -- newest -> oldest
% - 2025-06-02  v1.1 : ASCII 統一, beta^{final} 表記, clip 修正
% - 2025-05-31  v1.0 : 初版(Bayes 縮小)
%-------------------------------------------------------------------------------

%=== Phase 5 : w_profit ベイズ縮小 =============================================
\section*{event / earn / Phase 5}\nopagebreak[4]
%────────────────────────────────────
\subsection*{ステップ・目的}
\begin{flushleft}
\begin{enumerate}
  \item \textbf{サンプル数取得}\;
        \( n_i=\text{count\_earnings}(i,\text{last 3Y}) \)

  \item \textbf{セクター平均重み}\;
        \( \bar w_{\text{profit},s}=\operatorname{mean}(w_{\text{profit},j}) \)

  \item \textbf{Bayes 縮小}\;
        \[
          \tilde w_{\text{profit},i}
            =\frac{n_i}{n_i+\tau}\,w_{\text{profit},i}
             +\frac{\tau}{n_i+\tau}\,\bar w_{\text{profit},s},
          \quad \tau = 10
        \]
        \( \tilde w_{\text{profit},i}=\operatorname{clip}(\tilde w_{\text{profit},i},0.50,0.90) \)

  \item \textbf{サプライズ率再計算} → $\beta_{\text{earn},i,t}^{(5)}$ を取得。

  \item \textbf{イベント係数最終更新}\;
        \[
          \beta_{\text{event},i,t}^{\text{final}}
            =\beta_{\text{event},i,t}^{(4)}\,
             \beta_{\text{earn},i,t}^{(5)}
        \]
\end{enumerate}
\end{flushleft}

\subsection*{追加変数・係数}
\begin{flushleft}
\begin{minipage}{0.92\textwidth}
\begin{tabularx}{\textwidth}{@{}>{\hfil$\displaystyle}l<{$\hfil}@{\quad}X@{}}
\toprule
記号 & 定義・役割 \\
\midrule
n_i & 過去 3 年の決算サンプル数 \\
\tau & 縮小ハイパーパラメータ (10) \\
\bar w_{\text{profit},s} & セクター平均利益重み \\
\beta_{\text{event},i,t}^{\text{final}} & earn 系最終係数 \\
\bottomrule
\end{tabularx}
\end{minipage}
\end{flushleft}
\bigskip
%===============================================================================
       % Phase-:イベント係数テーブル
\clearpage

%-------------------------------------------------------------------------------
% event/earn/phase1.tex   v1.1  (2025-06-02)
%-------------------------------------------------------------------------------
% CHANGELOG  -- newest -> oldest
% - 2025-06-02  v1.1 : U+2212→ASCII "-", beta^{(1)} 表記, header tidy
% - 2025-05-31  v1.0 : 初版(決算 day±1 固定係数)
%-------------------------------------------------------------------------------

%=== Phase 1 : 決算係数(day±1) ==============================================
\section*{event / earn / Phase 1}\nopagebreak[4]
%────────────────────────────────────
\subsection*{ステップ・目的}
\begin{flushleft}
\begin{enumerate}
  \item \textbf{決算カレンダーでラグ判定}\;
        day -1(前日)/day 0(当日)/day +1(翌営業日)を抽出。
  \item \textbf{係数決定}\;
        \[
          \beta_{\text{earn},i,t}^{(1)}=
          \begin{cases}
            1.15 & (\text{day\,-1})\\
            1.20 & (\text{day\,0})\\
            1.10 & (\text{day\,+1})\\
            1.00 & (\text{otherwise})
          \end{cases}
        \]
  \item \textbf{イベント係数更新}\;
        \[
          \beta_{\text{event},i,t}^{(1)}
            =\beta_{\text{event},i,t}^{\text{prev}}
             \,\beta_{\text{earn},i,t}^{(1)},
          \quad 0.80 \le \beta_{\text{event},i,t}^{(1)} \le 1.50
        \]
\end{enumerate}
\end{flushleft}

\subsection*{追加変数・係数}
\begin{flushleft}
\begin{minipage}{0.90\textwidth}
\begin{tabularx}{\textwidth}{@{}>{\hfil$\displaystyle}l<{$\hfil}@{\quad}X@{}}
\toprule
記号 & 定義・役割 \\
\midrule
i & 銘柄コード \\
\beta_{\text{earn},i,t}^{(1)} & day±1 固定決算係数 \\
\beta_{\text{event},i,t}^{\text{prev}} & 直前フェーズ(weekday 等)出力 \\
\beta_{\text{event},i,t}^{(1)} & earn 系フェーズ 1 出力 \\
\bottomrule
\end{tabularx}
\end{minipage}
\end{flushleft}
\bigskip
%===============================================================================
       % Phase-1:イベント係数テーブル
\clearpage

%-------------------------------------------------------------------------------
% center_shift/sigma/phase2.tex   v1.3  (2025-06-02)
%-------------------------------------------------------------------------------
% CHANGELOG  -- new entry on top
% - 2025-06-02  v1.3 : add section/subsection, hints, ASCII-only
%-------------------------------------------------------------------------------

%=== center_shift =============================================================
\section*{center\_shift}\nopagebreak[4]

%--- sigma ---------------------------------------------------------------------
\subsection*{sigma}\nopagebreak[4]

%--- Phase 2 : 自己適応 λ_shift 更新 -------------------------------------------
\subsubsection*{Phase 2:自己適応 $\lambda_{\text{shift}}$ 更新}\nopagebreak[4]
%────────────────────────────────────
\paragraph{ステップ/目的}
\begin{flushleft}
\begin{enumerate}
  \item \textbf{誤差系列}\;
        \(e_{t-k}=\Delta Cl_{t-k}^{2}-\sigma_{t-k}^{2}\)
  \item \textbf{局所 MSE}\;
        \(\mathrm{MSE}_t=\dfrac{1}{30}\sum_{k=1}^{30}e_{t-k}^{2}\)
  \item \textbf{勾配近似}\;
        \(g_t\approx-\dfrac{2}{30}\sum_{k=1}^{30}
          e_{t-k}\,\sigma_{t-k}^{2}\)
  \item \textbf{$\lambda_{\text{shift}}$ 更新}\;
        \(\lambda_{\text{shift},t}
          =\operatorname{clip}\bigl(
            \lambda_{\text{shift},t-1}-\eta g_t,\,
            0.90,\,0.98\bigr)\)
  \item \textbf{翌日へ反映}\;
        上式の \(\lambda_{\text{shift},t}\) で  
        \(\sigma_{t+1}^{2}\) を再計算
\end{enumerate}
\end{flushleft}

\subsubsection*{変数のポイント}
\begin{flushleft}
\begin{itemize}
  \item \(\lambda_{\text{shift}}\) は [0.90, 0.98] に制限
  \item \(|g_t|\le10\) でクリップし暴走を防止
\end{itemize}
\end{flushleft}

\subsubsection*{実装ヒント}
\begin{flushleft}
学習率 \(\eta=0.01\) が無難。  
ウォームアップ期間 (30~d) は固定 \(\lambda_{\text{shift}}=0.94\)。
\end{flushleft}

\subsubsection*{追加変数・係数}
\begin{flushleft}
\begin{minipage}{0.90\textwidth}
\begin{tabularx}{\textwidth}{@{}>{\hfil$\displaystyle}l<{$\hfil}@{\quad}X@{}}
\toprule
記号 & 定義・役割 \\
\midrule
\lambda_{\text{shift},t-1} & 前日 EWMA 定数 \\
\lambda_{\text{shift},t}   & 更新後 EWMA 定数 \\
g_t & 勾配近似 \\
\eta & 学習率 (0.01) \\
e_{t-k} & 誤差 \\
\mathrm{MSE}_t & 30~d MSE \\
\bottomrule
\end{tabularx}
\end{minipage}
\end{flushleft}
\bigskip
%===============================================================================
       % Phase-2:イベント係数テーブル
\clearpage

%-------------------------------------------------------------------------------
% event/weekday/phase3.tex   v1.1  (2025-06-02)
%-------------------------------------------------------------------------------
% CHANGELOG  -- newest -> oldest
% - 2025-06-02  v1.1 : beta^{(3)} 表記・集約ロジック明確化
% - 2025-05-31  v1.0 : weekday 系サブフェーズ集約
%-------------------------------------------------------------------------------

%=== Phase 3 : weekday 系集約 ===================================================
\section*{event / weekday / Phase 3}\nopagebreak[4]
%────────────────────────────────────
\subsection*{ステップ・目的}
\begin{flushleft}
\begin{enumerate}
  \item \textbf{holiday 側最終係数を取り込み}\;
        \verb|%-------------------------------------------------------------------------------
% event/weekday/phase3.tex   v1.1  (2025-06-02)
%-------------------------------------------------------------------------------
% CHANGELOG  -- newest -> oldest
% - 2025-06-02  v1.1 : beta^{(3)} 表記・集約ロジック明確化
% - 2025-05-31  v1.0 : weekday 系サブフェーズ集約
%-------------------------------------------------------------------------------

%=== Phase 3 : weekday 系集約 ===================================================
\section*{event / weekday / Phase 3}\nopagebreak[4]
%────────────────────────────────────
\subsection*{ステップ・目的}
\begin{flushleft}
\begin{enumerate}
  \item \textbf{holiday 側最終係数を取り込み}\;
        \verb|%-------------------------------------------------------------------------------
% event/weekday/phase3.tex   v1.1  (2025-06-02)
%-------------------------------------------------------------------------------
% CHANGELOG  -- newest -> oldest
% - 2025-06-02  v1.1 : beta^{(3)} 表記・集約ロジック明確化
% - 2025-05-31  v1.0 : weekday 系サブフェーズ集約
%-------------------------------------------------------------------------------

%=== Phase 3 : weekday 系集約 ===================================================
\section*{event / weekday / Phase 3}\nopagebreak[4]
%────────────────────────────────────
\subsection*{ステップ・目的}
\begin{flushleft}
\begin{enumerate}
  \item \textbf{holiday 側最終係数を取り込み}\;
        \verb|\input{event/weekday/holiday/phase3}| で  
        \(\tilde\beta_{\text{weekday},i,t}\) を取得。
  \item \textbf{最終 weekday 係数を宣言}\;
        \[
          \boxed{\beta_{\text{weekday},i,t}^{(3)}
          =\tilde\beta_{\text{weekday},i,t}}
        \]
  \item \textbf{イベント係数パイプラインへ出力}\;
        event/phase0.tex が  
        \(\beta_{\text{weekday},i,t}^{(3)}\) を利用。
\end{enumerate}
\end{flushleft}

\subsection*{追加変数・係数}
\begin{flushleft}
\begin{minipage}{0.88\textwidth}
\begin{tabularx}{\textwidth}{@{}>{\hfil$\displaystyle}l<{$\hfil}@{\quad}X@{}}
\toprule
記号 & 定義・役割 \\
\midrule
\tilde\beta_{\text{weekday},i,t} & holiday/phase3 出力係数 \\
\beta_{\text{weekday},i,t}^{(3)} & weekday 系最終係数 (本フェーズ) \\
\bottomrule
\end{tabularx}
\end{minipage}
\end{flushleft}
\bigskip
%===============================================================================
| で  
        \(\tilde\beta_{\text{weekday},i,t}\) を取得。
  \item \textbf{最終 weekday 係数を宣言}\;
        \[
          \boxed{\beta_{\text{weekday},i,t}^{(3)}
          =\tilde\beta_{\text{weekday},i,t}}
        \]
  \item \textbf{イベント係数パイプラインへ出力}\;
        event/phase0.tex が  
        \(\beta_{\text{weekday},i,t}^{(3)}\) を利用。
\end{enumerate}
\end{flushleft}

\subsection*{追加変数・係数}
\begin{flushleft}
\begin{minipage}{0.88\textwidth}
\begin{tabularx}{\textwidth}{@{}>{\hfil$\displaystyle}l<{$\hfil}@{\quad}X@{}}
\toprule
記号 & 定義・役割 \\
\midrule
\tilde\beta_{\text{weekday},i,t} & holiday/phase3 出力係数 \\
\beta_{\text{weekday},i,t}^{(3)} & weekday 系最終係数 (本フェーズ) \\
\bottomrule
\end{tabularx}
\end{minipage}
\end{flushleft}
\bigskip
%===============================================================================
| で  
        \(\tilde\beta_{\text{weekday},i,t}\) を取得。
  \item \textbf{最終 weekday 係数を宣言}\;
        \[
          \boxed{\beta_{\text{weekday},i,t}^{(3)}
          =\tilde\beta_{\text{weekday},i,t}}
        \]
  \item \textbf{イベント係数パイプラインへ出力}\;
        event/phase0.tex が  
        \(\beta_{\text{weekday},i,t}^{(3)}\) を利用。
\end{enumerate}
\end{flushleft}

\subsection*{追加変数・係数}
\begin{flushleft}
\begin{minipage}{0.88\textwidth}
\begin{tabularx}{\textwidth}{@{}>{\hfil$\displaystyle}l<{$\hfil}@{\quad}X@{}}
\toprule
記号 & 定義・役割 \\
\midrule
\tilde\beta_{\text{weekday},i,t} & holiday/phase3 出力係数 \\
\beta_{\text{weekday},i,t}^{(3)} & weekday 系最終係数 (本フェーズ) \\
\bottomrule
\end{tabularx}
\end{minipage}
\end{flushleft}
\bigskip
%===============================================================================
       % Phase-3:イベント係数テーブル
\clearpage

%-------------------------------------------------------------------------------
% center_shift/phase4.tex   v1.0  (2025-06-06)
%-------------------------------------------------------------------------------
% CHANGELOG  -- new entry on top (latest -> oldest)
% - 2025-06-06  v1.0 : 初版
%-------------------------------------------------------------------------------

%=== center_shift =============================================================
\section*{center\_shift}\nopagebreak[4]

%=== Phase 4 : \eta / \lambda の深掘り ======================================
\subsection*{Phase 4:$\eta$ と $\lambda$ の深掘り}\nopagebreak[4]
%────────────────────────────────────
\paragraph{ステップ/目的}
\begin{flushleft}
\begin{enumerate}
  \item \textbf{学習率}
        \(\eta\) は $\lambda_{\text{shift}}$ 更新の歩幅を制御
  \item \textbf{勾配近似}
        \(g_t\approx-\dfrac{2}{30}\sum_{k=1}^{30}e_{t-k}\,\sigma_{t-k}^2\)
  \item \textbf{$\lambda_{\text{shift}}$ 更新}
        \(\lambda_{\text{shift},t}
          =\operatorname{clip}\bigl(\lambda_{\text{shift},t-1}
          -\eta\,g_t,\,0.90,\,0.98\bigr)\)
  \item \textbf{ウォームアップ}
        初期 30~d は固定 $\lambda_{\text{shift}}=0.94$ で安定化
\end{enumerate}
\end{flushleft}

\subsubsection*{変数のポイント}
\begin{flushleft}
\begin{itemize}
  \item 大きすぎる $\eta$ は \(\lambda_{\text{shift}}\) を振動させる
  \item 小さすぎる $\eta$ では収束が遅延
  \item 更新範囲 [0.90, 0.98] を超えないよう \(\operatorname{clip}\)
  \item $|g_t|>10$ なら勾配をクリップし安定化
\end{itemize}
\end{flushleft}

\subsubsection*{実装ヒント}
\begin{flushleft}
\begin{itemize}
  \item 経験的に $\eta=0.01$ が妥当な上限値
  \item 週次で $\eta$ の微調整を試し、予測 MAE を観察
  \item 勾配計算には 30~d の誤差系列を用意
\end{itemize}
\end{flushleft}

\subsubsection*{追加変数・係数}
\begin{flushleft}
\begin{minipage}{0.90\textwidth}
\begin{tabularx}{\textwidth}{@{}>{\hfil$\displaystyle}l<{$\hfil}@{\quad}X@{}}
\toprule
記号 & 定義・役割 \\
\midrule
\eta & 学習率 \\
\lambda_{\text{shift},t} & 更新後 EWMA 定数 \\
\lambda_{\text{shift},t-1} & 前日 EWMA 定数 \\
\sigma_t^2 & 分散推定値 \\
\operatorname{clip} & 範囲制限関数 \\
\end{tabularx}
\end{minipage}
\end{flushleft}
\bigskip
%==============================================================================
       % Phase-4:イベント係数テーブル
\clearpage

%-------------------------------------------------------------------------------
% event/earn/phase5.tex   v1.1  (2025-06-02)
%-------------------------------------------------------------------------------
% CHANGELOG  -- newest -> oldest
% - 2025-06-02  v1.1 : ASCII 統一, beta^{final} 表記, clip 修正
% - 2025-05-31  v1.0 : 初版(Bayes 縮小)
%-------------------------------------------------------------------------------

%=== Phase 5 : w_profit ベイズ縮小 =============================================
\section*{event / earn / Phase 5}\nopagebreak[4]
%────────────────────────────────────
\subsection*{ステップ・目的}
\begin{flushleft}
\begin{enumerate}
  \item \textbf{サンプル数取得}\;
        \( n_i=\text{count\_earnings}(i,\text{last 3Y}) \)

  \item \textbf{セクター平均重み}\;
        \( \bar w_{\text{profit},s}=\operatorname{mean}(w_{\text{profit},j}) \)

  \item \textbf{Bayes 縮小}\;
        \[
          \tilde w_{\text{profit},i}
            =\frac{n_i}{n_i+\tau}\,w_{\text{profit},i}
             +\frac{\tau}{n_i+\tau}\,\bar w_{\text{profit},s},
          \quad \tau = 10
        \]
        \( \tilde w_{\text{profit},i}=\operatorname{clip}(\tilde w_{\text{profit},i},0.50,0.90) \)

  \item \textbf{サプライズ率再計算} → $\beta_{\text{earn},i,t}^{(5)}$ を取得。

  \item \textbf{イベント係数最終更新}\;
        \[
          \beta_{\text{event},i,t}^{\text{final}}
            =\beta_{\text{event},i,t}^{(4)}\,
             \beta_{\text{earn},i,t}^{(5)}
        \]
\end{enumerate}
\end{flushleft}

\subsection*{追加変数・係数}
\begin{flushleft}
\begin{minipage}{0.92\textwidth}
\begin{tabularx}{\textwidth}{@{}>{\hfil$\displaystyle}l<{$\hfil}@{\quad}X@{}}
\toprule
記号 & 定義・役割 \\
\midrule
n_i & 過去 3 年の決算サンプル数 \\
\tau & 縮小ハイパーパラメータ (10) \\
\bar w_{\text{profit},s} & セクター平均利益重み \\
\beta_{\text{event},i,t}^{\text{final}} & earn 系最終係数 \\
\bottomrule
\end{tabularx}
\end{minipage}
\end{flushleft}
\bigskip
%===============================================================================
       % Phase-5:イベント係数テーブル
\clearpage

%-------------------------------------------------------------------------------
% event/phase0.tex   v1.4  (2025-06-02)
%-------------------------------------------------------------------------------
% CHANGELOG  -- newest -> oldest
% - 2025-06-02  v1.4 : section 階層見直し・ASCII 化・beta^{(3)} 表記へ統一
% - 2025-06-02  v1.3 : CHANGELOG 復元・整形を明記(rules.md 準拠)
% - 2025-05-31  v1.2 : fixed tabularx preamble to 3 columns (l X l)
% - 2025-05-31  v1.1 : \beta_event,i,t = \beta_weekday \times \beta_earn \times \beta_market
% - 2025-05-31  v1.0 : \beta_event,t = 1.0 fallback
%-------------------------------------------------------------------------------

%=== Phase 0 : イベント係数 基本定義 ============================================
\section*{event / Phase 0 : 基本定義}\nopagebreak[4]
%────────────────────────────────────
\begin{flushleft}
\begin{flalign*}
&\text{イベント係数(銘柄 }i\text{)}\quad
  \boxed{%
    \beta_{\text{event},i,t}
      =\beta_{\text{weekday},i,t}^{(3)}\,
       \beta_{\text{earn},i,t}\,
       \beta_{\text{market},i,t}
  } &&\\[6pt]
\end{flalign*}
\end{flushleft}

\subsection*{因子の役割}
\begin{flushleft}
\begin{minipage}{0.92\textwidth}
\begin{tabularx}{\textwidth}{@{}>{\hfil$\displaystyle}l<{$\hfil}@{\quad}X@{\quad}l@{}}
\toprule
因子 & 定義・データソース & 既定レンジ \\
\midrule
\beta_{\text{weekday},i,t}^{(3)} & 曜日+祝日+平滑済み最終係数 & 0.8--1.2 \\
\beta_{\text{earn},i,t}          & 決算ラグ・内容反映係数        & 0.8--1.5 \\
\beta_{\text{market},i,t}        & 指標相関係数                 & 0.8--1.2 \\
\bottomrule
\end{tabularx}
\end{minipage}
\end{flushleft}

\subsection*{備考}
\begin{flushleft}
\begin{itemize}
  \item 欠損時は 1.0 にフォールバック。  
  \item 係数更新は weekday / earn / market サブディレクトリで実施。  
\end{itemize}
\end{flushleft}
\bigskip
%===============================================================================

\clearpage

%-------------------------------------------------------------------------------
% event/earn/phase1.tex   v1.1  (2025-06-02)
%-------------------------------------------------------------------------------
% CHANGELOG  -- newest -> oldest
% - 2025-06-02  v1.1 : U+2212→ASCII "-", beta^{(1)} 表記, header tidy
% - 2025-05-31  v1.0 : 初版(決算 day±1 固定係数)
%-------------------------------------------------------------------------------

%=== Phase 1 : 決算係数(day±1) ==============================================
\section*{event / earn / Phase 1}\nopagebreak[4]
%────────────────────────────────────
\subsection*{ステップ・目的}
\begin{flushleft}
\begin{enumerate}
  \item \textbf{決算カレンダーでラグ判定}\;
        day -1(前日)/day 0(当日)/day +1(翌営業日)を抽出。
  \item \textbf{係数決定}\;
        \[
          \beta_{\text{earn},i,t}^{(1)}=
          \begin{cases}
            1.15 & (\text{day\,-1})\\
            1.20 & (\text{day\,0})\\
            1.10 & (\text{day\,+1})\\
            1.00 & (\text{otherwise})
          \end{cases}
        \]
  \item \textbf{イベント係数更新}\;
        \[
          \beta_{\text{event},i,t}^{(1)}
            =\beta_{\text{event},i,t}^{\text{prev}}
             \,\beta_{\text{earn},i,t}^{(1)},
          \quad 0.80 \le \beta_{\text{event},i,t}^{(1)} \le 1.50
        \]
\end{enumerate}
\end{flushleft}

\subsection*{追加変数・係数}
\begin{flushleft}
\begin{minipage}{0.90\textwidth}
\begin{tabularx}{\textwidth}{@{}>{\hfil$\displaystyle}l<{$\hfil}@{\quad}X@{}}
\toprule
記号 & 定義・役割 \\
\midrule
i & 銘柄コード \\
\beta_{\text{earn},i,t}^{(1)} & day±1 固定決算係数 \\
\beta_{\text{event},i,t}^{\text{prev}} & 直前フェーズ(weekday 等)出力 \\
\beta_{\text{event},i,t}^{(1)} & earn 系フェーズ 1 出力 \\
\bottomrule
\end{tabularx}
\end{minipage}
\end{flushleft}
\bigskip
%===============================================================================
       % Phase-1:EMA5 符号
\clearpage

%-------------------------------------------------------------------------------
% center_shift/sigma/phase2.tex   v1.3  (2025-06-02)
%-------------------------------------------------------------------------------
% CHANGELOG  -- new entry on top
% - 2025-06-02  v1.3 : add section/subsection, hints, ASCII-only
%-------------------------------------------------------------------------------

%=== center_shift =============================================================
\section*{center\_shift}\nopagebreak[4]

%--- sigma ---------------------------------------------------------------------
\subsection*{sigma}\nopagebreak[4]

%--- Phase 2 : 自己適応 λ_shift 更新 -------------------------------------------
\subsubsection*{Phase 2:自己適応 $\lambda_{\text{shift}}$ 更新}\nopagebreak[4]
%────────────────────────────────────
\paragraph{ステップ/目的}
\begin{flushleft}
\begin{enumerate}
  \item \textbf{誤差系列}\;
        \(e_{t-k}=\Delta Cl_{t-k}^{2}-\sigma_{t-k}^{2}\)
  \item \textbf{局所 MSE}\;
        \(\mathrm{MSE}_t=\dfrac{1}{30}\sum_{k=1}^{30}e_{t-k}^{2}\)
  \item \textbf{勾配近似}\;
        \(g_t\approx-\dfrac{2}{30}\sum_{k=1}^{30}
          e_{t-k}\,\sigma_{t-k}^{2}\)
  \item \textbf{$\lambda_{\text{shift}}$ 更新}\;
        \(\lambda_{\text{shift},t}
          =\operatorname{clip}\bigl(
            \lambda_{\text{shift},t-1}-\eta g_t,\,
            0.90,\,0.98\bigr)\)
  \item \textbf{翌日へ反映}\;
        上式の \(\lambda_{\text{shift},t}\) で  
        \(\sigma_{t+1}^{2}\) を再計算
\end{enumerate}
\end{flushleft}

\subsubsection*{変数のポイント}
\begin{flushleft}
\begin{itemize}
  \item \(\lambda_{\text{shift}}\) は [0.90, 0.98] に制限
  \item \(|g_t|\le10\) でクリップし暴走を防止
\end{itemize}
\end{flushleft}

\subsubsection*{実装ヒント}
\begin{flushleft}
学習率 \(\eta=0.01\) が無難。  
ウォームアップ期間 (30~d) は固定 \(\lambda_{\text{shift}}=0.94\)。
\end{flushleft}

\subsubsection*{追加変数・係数}
\begin{flushleft}
\begin{minipage}{0.90\textwidth}
\begin{tabularx}{\textwidth}{@{}>{\hfil$\displaystyle}l<{$\hfil}@{\quad}X@{}}
\toprule
記号 & 定義・役割 \\
\midrule
\lambda_{\text{shift},t-1} & 前日 EWMA 定数 \\
\lambda_{\text{shift},t}   & 更新後 EWMA 定数 \\
g_t & 勾配近似 \\
\eta & 学習率 (0.01) \\
e_{t-k} & 誤差 \\
\mathrm{MSE}_t & 30~d MSE \\
\bottomrule
\end{tabularx}
\end{minipage}
\end{flushleft}
\bigskip
%===============================================================================
       % Phase-2:σ 比補正
\clearpage

%-------------------------------------------------------------------------------
% event/weekday/phase3.tex   v1.1  (2025-06-02)
%-------------------------------------------------------------------------------
% CHANGELOG  -- newest -> oldest
% - 2025-06-02  v1.1 : beta^{(3)} 表記・集約ロジック明確化
% - 2025-05-31  v1.0 : weekday 系サブフェーズ集約
%-------------------------------------------------------------------------------

%=== Phase 3 : weekday 系集約 ===================================================
\section*{event / weekday / Phase 3}\nopagebreak[4]
%────────────────────────────────────
\subsection*{ステップ・目的}
\begin{flushleft}
\begin{enumerate}
  \item \textbf{holiday 側最終係数を取り込み}\;
        \verb|%-------------------------------------------------------------------------------
% event/weekday/phase3.tex   v1.1  (2025-06-02)
%-------------------------------------------------------------------------------
% CHANGELOG  -- newest -> oldest
% - 2025-06-02  v1.1 : beta^{(3)} 表記・集約ロジック明確化
% - 2025-05-31  v1.0 : weekday 系サブフェーズ集約
%-------------------------------------------------------------------------------

%=== Phase 3 : weekday 系集約 ===================================================
\section*{event / weekday / Phase 3}\nopagebreak[4]
%────────────────────────────────────
\subsection*{ステップ・目的}
\begin{flushleft}
\begin{enumerate}
  \item \textbf{holiday 側最終係数を取り込み}\;
        \verb|%-------------------------------------------------------------------------------
% event/weekday/phase3.tex   v1.1  (2025-06-02)
%-------------------------------------------------------------------------------
% CHANGELOG  -- newest -> oldest
% - 2025-06-02  v1.1 : beta^{(3)} 表記・集約ロジック明確化
% - 2025-05-31  v1.0 : weekday 系サブフェーズ集約
%-------------------------------------------------------------------------------

%=== Phase 3 : weekday 系集約 ===================================================
\section*{event / weekday / Phase 3}\nopagebreak[4]
%────────────────────────────────────
\subsection*{ステップ・目的}
\begin{flushleft}
\begin{enumerate}
  \item \textbf{holiday 側最終係数を取り込み}\;
        \verb|\input{event/weekday/holiday/phase3}| で  
        \(\tilde\beta_{\text{weekday},i,t}\) を取得。
  \item \textbf{最終 weekday 係数を宣言}\;
        \[
          \boxed{\beta_{\text{weekday},i,t}^{(3)}
          =\tilde\beta_{\text{weekday},i,t}}
        \]
  \item \textbf{イベント係数パイプラインへ出力}\;
        event/phase0.tex が  
        \(\beta_{\text{weekday},i,t}^{(3)}\) を利用。
\end{enumerate}
\end{flushleft}

\subsection*{追加変数・係数}
\begin{flushleft}
\begin{minipage}{0.88\textwidth}
\begin{tabularx}{\textwidth}{@{}>{\hfil$\displaystyle}l<{$\hfil}@{\quad}X@{}}
\toprule
記号 & 定義・役割 \\
\midrule
\tilde\beta_{\text{weekday},i,t} & holiday/phase3 出力係数 \\
\beta_{\text{weekday},i,t}^{(3)} & weekday 系最終係数 (本フェーズ) \\
\bottomrule
\end{tabularx}
\end{minipage}
\end{flushleft}
\bigskip
%===============================================================================
| で  
        \(\tilde\beta_{\text{weekday},i,t}\) を取得。
  \item \textbf{最終 weekday 係数を宣言}\;
        \[
          \boxed{\beta_{\text{weekday},i,t}^{(3)}
          =\tilde\beta_{\text{weekday},i,t}}
        \]
  \item \textbf{イベント係数パイプラインへ出力}\;
        event/phase0.tex が  
        \(\beta_{\text{weekday},i,t}^{(3)}\) を利用。
\end{enumerate}
\end{flushleft}

\subsection*{追加変数・係数}
\begin{flushleft}
\begin{minipage}{0.88\textwidth}
\begin{tabularx}{\textwidth}{@{}>{\hfil$\displaystyle}l<{$\hfil}@{\quad}X@{}}
\toprule
記号 & 定義・役割 \\
\midrule
\tilde\beta_{\text{weekday},i,t} & holiday/phase3 出力係数 \\
\beta_{\text{weekday},i,t}^{(3)} & weekday 系最終係数 (本フェーズ) \\
\bottomrule
\end{tabularx}
\end{minipage}
\end{flushleft}
\bigskip
%===============================================================================
| で  
        \(\tilde\beta_{\text{weekday},i,t}\) を取得。
  \item \textbf{最終 weekday 係数を宣言}\;
        \[
          \boxed{\beta_{\text{weekday},i,t}^{(3)}
          =\tilde\beta_{\text{weekday},i,t}}
        \]
  \item \textbf{イベント係数パイプラインへ出力}\;
        event/phase0.tex が  
        \(\beta_{\text{weekday},i,t}^{(3)}\) を利用。
\end{enumerate}
\end{flushleft}

\subsection*{追加変数・係数}
\begin{flushleft}
\begin{minipage}{0.88\textwidth}
\begin{tabularx}{\textwidth}{@{}>{\hfil$\displaystyle}l<{$\hfil}@{\quad}X@{}}
\toprule
記号 & 定義・役割 \\
\midrule
\tilde\beta_{\text{weekday},i,t} & holiday/phase3 出力係数 \\
\beta_{\text{weekday},i,t}^{(3)} & weekday 系最終係数 (本フェーズ) \\
\bottomrule
\end{tabularx}
\end{minipage}
\end{flushleft}
\bigskip
%===============================================================================
       % Phase-3:ボラレジーム補正
\clearpage

%-------------------------------------------------------------------------------
% center_shift/phase4.tex   v1.0  (2025-06-06)
%-------------------------------------------------------------------------------
% CHANGELOG  -- new entry on top (latest -> oldest)
% - 2025-06-06  v1.0 : 初版
%-------------------------------------------------------------------------------

%=== center_shift =============================================================
\section*{center\_shift}\nopagebreak[4]

%=== Phase 4 : \eta / \lambda の深掘り ======================================
\subsection*{Phase 4:$\eta$ と $\lambda$ の深掘り}\nopagebreak[4]
%────────────────────────────────────
\paragraph{ステップ/目的}
\begin{flushleft}
\begin{enumerate}
  \item \textbf{学習率}
        \(\eta\) は $\lambda_{\text{shift}}$ 更新の歩幅を制御
  \item \textbf{勾配近似}
        \(g_t\approx-\dfrac{2}{30}\sum_{k=1}^{30}e_{t-k}\,\sigma_{t-k}^2\)
  \item \textbf{$\lambda_{\text{shift}}$ 更新}
        \(\lambda_{\text{shift},t}
          =\operatorname{clip}\bigl(\lambda_{\text{shift},t-1}
          -\eta\,g_t,\,0.90,\,0.98\bigr)\)
  \item \textbf{ウォームアップ}
        初期 30~d は固定 $\lambda_{\text{shift}}=0.94$ で安定化
\end{enumerate}
\end{flushleft}

\subsubsection*{変数のポイント}
\begin{flushleft}
\begin{itemize}
  \item 大きすぎる $\eta$ は \(\lambda_{\text{shift}}\) を振動させる
  \item 小さすぎる $\eta$ では収束が遅延
  \item 更新範囲 [0.90, 0.98] を超えないよう \(\operatorname{clip}\)
  \item $|g_t|>10$ なら勾配をクリップし安定化
\end{itemize}
\end{flushleft}

\subsubsection*{実装ヒント}
\begin{flushleft}
\begin{itemize}
  \item 経験的に $\eta=0.01$ が妥当な上限値
  \item 週次で $\eta$ の微調整を試し、予測 MAE を観察
  \item 勾配計算には 30~d の誤差系列を用意
\end{itemize}
\end{flushleft}

\subsubsection*{追加変数・係数}
\begin{flushleft}
\begin{minipage}{0.90\textwidth}
\begin{tabularx}{\textwidth}{@{}>{\hfil$\displaystyle}l<{$\hfil}@{\quad}X@{}}
\toprule
記号 & 定義・役割 \\
\midrule
\eta & 学習率 \\
\lambda_{\text{shift},t} & 更新後 EWMA 定数 \\
\lambda_{\text{shift},t-1} & 前日 EWMA 定数 \\
\sigma_t^2 & 分散推定値 \\
\operatorname{clip} & 範囲制限関数 \\
\end{tabularx}
\end{minipage}
\end{flushleft}
\bigskip
%==============================================================================
       % Phase-4:λ_gamma 自己適応
\clearpage

%-------------------------------------------------------------------------------
% event/earn/phase5.tex   v1.1  (2025-06-02)
%-------------------------------------------------------------------------------
% CHANGELOG  -- newest -> oldest
% - 2025-06-02  v1.1 : ASCII 統一, beta^{final} 表記, clip 修正
% - 2025-05-31  v1.0 : 初版(Bayes 縮小)
%-------------------------------------------------------------------------------

%=== Phase 5 : w_profit ベイズ縮小 =============================================
\section*{event / earn / Phase 5}\nopagebreak[4]
%────────────────────────────────────
\subsection*{ステップ・目的}
\begin{flushleft}
\begin{enumerate}
  \item \textbf{サンプル数取得}\;
        \( n_i=\text{count\_earnings}(i,\text{last 3Y}) \)

  \item \textbf{セクター平均重み}\;
        \( \bar w_{\text{profit},s}=\operatorname{mean}(w_{\text{profit},j}) \)

  \item \textbf{Bayes 縮小}\;
        \[
          \tilde w_{\text{profit},i}
            =\frac{n_i}{n_i+\tau}\,w_{\text{profit},i}
             +\frac{\tau}{n_i+\tau}\,\bar w_{\text{profit},s},
          \quad \tau = 10
        \]
        \( \tilde w_{\text{profit},i}=\operatorname{clip}(\tilde w_{\text{profit},i},0.50,0.90) \)

  \item \textbf{サプライズ率再計算} → $\beta_{\text{earn},i,t}^{(5)}$ を取得。

  \item \textbf{イベント係数最終更新}\;
        \[
          \beta_{\text{event},i,t}^{\text{final}}
            =\beta_{\text{event},i,t}^{(4)}\,
             \beta_{\text{earn},i,t}^{(5)}
        \]
\end{enumerate}
\end{flushleft}

\subsection*{追加変数・係数}
\begin{flushleft}
\begin{minipage}{0.92\textwidth}
\begin{tabularx}{\textwidth}{@{}>{\hfil$\displaystyle}l<{$\hfil}@{\quad}X@{}}
\toprule
記号 & 定義・役割 \\
\midrule
n_i & 過去 3 年の決算サンプル数 \\
\tau & 縮小ハイパーパラメータ (10) \\
\bar w_{\text{profit},s} & セクター平均利益重み \\
\beta_{\text{event},i,t}^{\text{final}} & earn 系最終係数 \\
\bottomrule
\end{tabularx}
\end{minipage}
\end{flushleft}
\bigskip
%===============================================================================
       % Phase-5:モメンタム係数テーブル
\clearpage


\end{document}
