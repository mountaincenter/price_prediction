% parent.tex   v1.0  (2025-05-29)
% ───────────────────────────────────────────
% CHANGELOG:
% - 2025-06-11  range phase6 追加
% - 2025-06-07  range phase5 追加
% - 2025-06-10  open_price phase5 追加
% - 2025-06-06  center_shift phase3 追加
% - 2025-05-29  open_price ディレクトリ分割に合わせて新規作成
%
% Header-rules:
% • 未来日・過去日を入れない(Created は初回作成日だけ)
% • 修正があれば最上段に “- YYYY-MM-DD  概要” を追記
% • LaTeX では行頭 % を docstring の代替に使う
% • 内容を本文中や他コメントへ重複させない
% ───────────────────────────────────────────

\documentclass[dvipdfmx,openany,oneside]{jsbook}
\usepackage{amsmath,amssymb,tabularx,booktabs,graphicx}
\renewcommand{\arraystretch}{1.2}

\begin{document}

%-------------------------------------------------------------------------------
% basic_form.tex   v1.1  (2025-05-28)
%───────────────────────────────────────────────────────────────────────────────
% CHANGELOG:  最新→過去(降順)で追加してください
% - 2025-05-29  数式・表すべてを flushleft/flalign* ベースに統一
%-------------------------------------------------------------------------------

%=== 基本形:日中レンジを伴うモデル ==========================================
\section*{1. 基本形:日中レンジを伴うモデル}\nopagebreak[4]
%────────────────────────────────────
\begin{flushleft}
\begin{flalign*}
&\text{ベース値}\quad
  B_{t-1}=
    \begin{cases}
      Cl_{t-1} & (\text{デイトレ中心})\\[4pt]
      \dfrac{H_{t-1}+L_{t-1}}{2} & (\text{振れの大きい銘柄})
    \end{cases} &&\\[10pt]
%
&\text{中心シフト量}\quad
  \alpha_t=\kappa(\sigma_t)\,S_t,\qquad 0\le|S_t|\le1 &&\\[6pt]
%
&\text{日中中心値}\quad
  C_t=B_{t-1}(1+\alpha_t)\,\beta_{\text{event},t} &&\\[10pt]
%
&\text{半レンジ}\quad
  m_t=\sigma_t\,\beta_{\text{vol},t} &&\\[6pt]
%
&\text{高値}\quad
  H_t=C_t+m_t &&\\[4pt]
&\text{安値}\quad
  L_t=C_t-m_t &&\\[10pt]
%
&\text{始値}\quad
  O_t=C_t+\gamma_t\sigma_t &&\\[4pt]
&\text{終値}\quad
  Cl_t=C_t-\gamma_t\sigma_t &&
\end{flalign*}
\end{flushleft}

\subsection*{主要変数・係数(基本形)}
\begin{flushleft}
\begin{minipage}{0.85\textwidth}
\begin{tabularx}{\textwidth}{@{}>{\hfil$\displaystyle}l<{$\hfil}@{\quad}X@{}}
\toprule
記号 & 定義・役割 \\
\midrule
Cl_{t-1} & 前日終値 \\
H_{t-1},L_{t-1} & 前日高値・安値 \\
B_{t-1} & 前日リファレンス値 \\
\sigma_{t} & 当日ボラティリティ推定 \\
\kappa(\sigma) & ボラ依存シフトスケール \\
S_{t} & direction\_score \\
\beta_{\text{event},t} & 曜日・決算などバイアス係数 \\
\beta_{\text{vol},t} & 幅倍率(\(\sigma\) 拡大率) \\
\gamma_{t} & モメンタム偏位係数 \\
\bottomrule
\end{tabularx}
\end{minipage}
\end{flushleft}
\bigskip
%===============================================================================
   % 1 章:基本形
\clearpage

%-------------------------------------------------------------------------------
% momentum/phase0.tex   v1.1  (2025-06-02)
%-------------------------------------------------------------------------------
% CHANGELOG  -- new entry on top (latest -> oldest)
% - 2025-06-02  v1.1 : 「変数のポイント」節を追加
% - 2025-05-31  v1.0 : モメンタム係数 γ_t ベースライン(γ_t=0)
%-------------------------------------------------------------------------------

%=== Phase-0 : モメンタム係数 γ_t ベースライン ================================
\section*{Phase 0:モメンタム係数 $\gamma_t$ ベースライン}\nopagebreak[4]
%────────────────────────────────────
\begin{flushleft}
\[
  \boxed{\gamma_t = 0}
\]
\end{flushleft}

\subsection*{変数のポイント}
\begin{flushleft}
\begin{itemize}
  \item データ欠損・学習初期のフォールバックとして **常に \(\gamma_t=0\)** を使用。  
        始値と終値のシフトを一切行わない。
\end{itemize}
\end{flushleft}

\subsection*{変数メモ}
\begin{flushleft}
本フェーズは学習初期・データ欠損時のフォールバックとして使用し、  
当日寄り付きと引けの偏位を考慮しない設定($\gamma_t=0$)を採用する。
\end{flushleft}
%===============================================================================
       % Phase-0:EWMA ギャップ
\clearpage

%-------------------------------------------------------------------------------
% event/market/phase1.tex   v1.2  (2025-06-02)
%-------------------------------------------------------------------------------
% CHANGELOG: newest -> oldest
% - 2025-06-02  v1.2 : beta_{i,t}^{(m1)} 表記・section 階層化
% - 2025-05-31  v1.1 : 指標セットを {TOPIX,SPX,USDJPY} に縮小
% - 2025-05-31  v1.0 : 初版(旧 225 系 + DJI)
%-------------------------------------------------------------------------------

%=== Phase 1 : マーケット指標係数 ==============================================
\section*{event / market / Phase 1}\nopagebreak[4]
%────────────────────────────────────
\subsection*{ステップ・目的}
\begin{flushleft}
\begin{enumerate}
  \item \textbf{63 d 相関係数}
        \[
          \rho_t^{(i)}
            =\operatorname{corr}\!\bigl(
              \Delta Cl_{t-62 \ldots t},
              \Delta M_{t-62 \ldots t}^{(i)}
            \bigr)
        \]
  \item \textbf{当日 Z-score}
        \(
          z_t^{(i)}
            =\dfrac{\Delta M_t^{(i)}}{\sigma_{63}^{(i)}}
        \)
  \item \textbf{指標係数(クリップ 0.8--1.2)}
        \[
          \beta_{i,t}^{(m1)}
            =\operatorname{clip}\!\bigl(
               1+\rho_t^{(i)}z_t^{(i)},\,0.8,\,1.2
             \bigr)
        \]
  \item \textbf{イベント係数を更新}
        \[
          \beta_{\text{event},i,t}^{(m1)}
            =\beta_{\text{event},i,t}^{\text{prev}}
             \prod_{i \in S} \beta_{i,t}^{(m1)},
          \quad
          S=\{\text{TOPIX},\text{SPX},\text{USDJPY}\}
        \]
\end{enumerate}
\end{flushleft}

\subsection*{追加変数・係数}
\begin{flushleft}
\begin{minipage}{0.88\textwidth}
\begin{tabularx}{\textwidth}{@{}>{\hfil$\displaystyle}l<{$\hfil}@{\quad}X@{}}
\toprule
記号 & 定義・役割 \\
\midrule
\Delta M_t^{(i)} & 指標 \(i\) の当日リターン \\
\sigma_{63}^{(i)} & 指標 \(i\) の 63 日標準偏差 \\
\rho_t^{(i)} & 63 日相関係数 \\
\beta_{i,t}^{(m1)} & Phase 1 指標係数 (0.8--1.2) \\
\beta_{\text{event},i,t}^{\text{prev}} & 直前フェーズ出力 \\
\beta_{\text{event},i,t}^{(m1)} & Phase 1 出力 (市場要因反映) \\
\bottomrule
\end{tabularx}
\end{minipage}
\end{flushleft}
\bigskip
%===============================================================================
       % Phase-1:IQR スケーリング
\clearpage

%-------------------------------------------------------------------------------
% event/phase2.tex   v1.0  (2025-06-09)
%-------------------------------------------------------------------------------
% CHANGELOG
% - 2025-06-09  v1.0 : weekday/earn/market 改良版を追加
%-------------------------------------------------------------------------------

%=== Phase 2 : イベント係数改良版 ===============================================
\section*{event / Phase 2 : 改良版}\nopagebreak[4]
%────────────────────────────────────
\subsection*{ステップ・目的}
\begin{flushleft}
\begin{enumerate}
  \item weekday 係数を自己適応的 EWMA\,(\(\lambda_{\text{wd}}\)) で平滑。
  \item 決算日の近傍 \(\pm2\) 日まで減衰する係数 \(w_d\) を適用。
  \item 市場係数は日経平均VIを加え、42 日相関で EWMA 更新。
\end{enumerate}
\end{flushleft}

\subsection*{追加変数・係数}
\begin{flushleft}
\begin{minipage}{0.90\textwidth}
\begin{tabularx}{\textwidth}{@{}>{\hfil$\displaystyle}l<{$\hfil}@{\quad}X@{}}
\toprule
記号 & 定義・役割 \\
\midrule
\lambda_{\text{wd}} & len(dates)>150 なら 0.92, それ以外 0.88 \\
 w_d & 決算日からの距離による係数 (1.20--1.10) \\
\beta_{\text{market},i,t}^{(m2)} & 42日相関 + VI 平滑後係数 \\
\bottomrule
\end{tabularx}
\end{minipage}
\end{flushleft}
\bigskip
%===============================================================================
       % Phase-2:σ 比補正
\clearpage

%-------------------------------------------------------------------------------
% event/market/phase3.tex   v1.2  (2025-06-02)
%-------------------------------------------------------------------------------
% CHANGELOG: newest -> oldest
% - 2025-06-02  v1.2 : beta_{63,i} 算出式を明示し ASCII 化
% - 2025-05-31  v1.1 : 書式統一
% - 2025-05-31  v1.0 : 初版
%-------------------------------------------------------------------------------

%=== Phase 3 : 銘柄 beta 補正 ==================================================
\section*{event / market / Phase 3}\nopagebreak[4]
%────────────────────────────────────
\subsection*{ステップ・目的}
\begin{flushleft}
\begin{enumerate}
  \item \textbf{63 d beta を計算}
        \[
          \beta_{63,i}=
          \frac{\operatorname{Cov}\!\bigl(r_i,r_{\text{TOPIX}}\bigr)}
               {\operatorname{Var}\!\bigl(r_{\text{TOPIX}}\bigr)}
        \]
  \item \textbf{補正係数}
        \[
          c_i=
          \begin{cases}
            1.05 & \beta_{63,i}>1.0\\
            0.95 & \beta_{63,i}<0.5\\
            1.00 & \text{otherwise}
          \end{cases}
        \]
  \item \textbf{イベント係数を更新}
        \(
          \beta_{\text{event},i,t}^{(m3)}
            =\beta_{\text{event},i,t}^{(m2)}\,c_i
        \)
\end{enumerate}
\end{flushleft}

\subsection*{追加変数・係数}
\begin{flushleft}
\begin{minipage}{0.88\textwidth}
\begin{tabularx}{\textwidth}{@{}>{\hfil$\displaystyle}l<{$\hfil}@{\quad}X@{}}
\toprule
記号 & 定義・役割 \\
\midrule
\beta_{63,i} & 63 日 TOPIX beta \\
c_i & 補正係数 (0.95 / 1.05) \\
\beta_{\text{event},i,t}^{(m2)} & Phase 2 出力 \\
\beta_{\text{event},i,t}^{(m3)} & Phase 3 出力 \\
\bottomrule
\end{tabularx}
\end{minipage}
\end{flushleft}
\bigskip
%===============================================================================
       % Phase-3:Proxy Board Gap
\clearpage

%-------------------------------------------------------------------------------
% center_shift/phase4.tex   v1.0  (2025-06-06)
%-------------------------------------------------------------------------------
% CHANGELOG  -- new entry on top (latest -> oldest)
% - 2025-06-06  v1.0 : 初版
%-------------------------------------------------------------------------------

%=== center_shift =============================================================
\section*{center\_shift}\nopagebreak[4]

%=== Phase 4 : \eta / \lambda の深掘り ======================================
\subsection*{Phase 4:$\eta$ と $\lambda$ の深掘り}\nopagebreak[4]
%────────────────────────────────────
\paragraph{ステップ/目的}
\begin{flushleft}
\begin{enumerate}
  \item \textbf{学習率}
        \(\eta\) は $\lambda_{\text{shift}}$ 更新の歩幅を制御
  \item \textbf{勾配近似}
        \(g_t\approx-\dfrac{2}{30}\sum_{k=1}^{30}e_{t-k}\,\sigma_{t-k}^2\)
  \item \textbf{$\lambda_{\text{shift}}$ 更新}
        \(\lambda_{\text{shift},t}
          =\operatorname{clip}\bigl(\lambda_{\text{shift},t-1}
          -\eta\,g_t,\,0.90,\,0.98\bigr)\)
  \item \textbf{ウォームアップ}
        初期 30~d は固定 $\lambda_{\text{shift}}=0.94$ で安定化
\end{enumerate}
\end{flushleft}

\subsubsection*{変数のポイント}
\begin{flushleft}
\begin{itemize}
  \item 大きすぎる $\eta$ は \(\lambda_{\text{shift}}\) を振動させる
  \item 小さすぎる $\eta$ では収束が遅延
  \item 更新範囲 [0.90, 0.98] を超えないよう \(\operatorname{clip}\)
  \item $|g_t|>10$ なら勾配をクリップし安定化
\end{itemize}
\end{flushleft}

\subsubsection*{実装ヒント}
\begin{flushleft}
\begin{itemize}
  \item 経験的に $\eta=0.01$ が妥当な上限値
  \item 週次で $\eta$ の微調整を試し、予測 MAE を観察
  \item 勾配計算には 30~d の誤差系列を用意
\end{itemize}
\end{flushleft}

\subsubsection*{追加変数・係数}
\begin{flushleft}
\begin{minipage}{0.90\textwidth}
\begin{tabularx}{\textwidth}{@{}>{\hfil$\displaystyle}l<{$\hfil}@{\quad}X@{}}
\toprule
記号 & 定義・役割 \\
\midrule
\eta & 学習率 \\
\lambda_{\text{shift},t} & 更新後 EWMA 定数 \\
\lambda_{\text{shift},t-1} & 前日 EWMA 定数 \\
\sigma_t^2 & 分散推定値 \\
\operatorname{clip} & 範囲制限関数 \\
\end{tabularx}
\end{minipage}
\end{flushleft}
\bigskip
%==============================================================================

\clearpage

%-------------------------------------------------------------------------------
% event/earn/phase5.tex   v1.1  (2025-06-02)
%-------------------------------------------------------------------------------
% CHANGELOG  -- newest -> oldest
% - 2025-06-02  v1.1 : ASCII 統一, beta^{final} 表記, clip 修正
% - 2025-05-31  v1.0 : 初版(Bayes 縮小)
%-------------------------------------------------------------------------------

%=== Phase 5 : w_profit ベイズ縮小 =============================================
\section*{event / earn / Phase 5}\nopagebreak[4]
%────────────────────────────────────
\subsection*{ステップ・目的}
\begin{flushleft}
\begin{enumerate}
  \item \textbf{サンプル数取得}\;
        \( n_i=\text{count\_earnings}(i,\text{last 3Y}) \)

  \item \textbf{セクター平均重み}\;
        \( \bar w_{\text{profit},s}=\operatorname{mean}(w_{\text{profit},j}) \)

  \item \textbf{Bayes 縮小}\;
        \[
          \tilde w_{\text{profit},i}
            =\frac{n_i}{n_i+\tau}\,w_{\text{profit},i}
             +\frac{\tau}{n_i+\tau}\,\bar w_{\text{profit},s},
          \quad \tau = 10
        \]
        \( \tilde w_{\text{profit},i}=\operatorname{clip}(\tilde w_{\text{profit},i},0.50,0.90) \)

  \item \textbf{サプライズ率再計算} → $\beta_{\text{earn},i,t}^{(5)}$ を取得。

  \item \textbf{イベント係数最終更新}\;
        \[
          \beta_{\text{event},i,t}^{\text{final}}
            =\beta_{\text{event},i,t}^{(4)}\,
             \beta_{\text{earn},i,t}^{(5)}
        \]
\end{enumerate}
\end{flushleft}

\subsection*{追加変数・係数}
\begin{flushleft}
\begin{minipage}{0.92\textwidth}
\begin{tabularx}{\textwidth}{@{}>{\hfil$\displaystyle}l<{$\hfil}@{\quad}X@{}}
\toprule
記号 & 定義・役割 \\
\midrule
n_i & 過去 3 年の決算サンプル数 \\
\tau & 縮小ハイパーパラメータ (10) \\
\bar w_{\text{profit},s} & セクター平均利益重み \\
\beta_{\text{event},i,t}^{\text{final}} & earn 系最終係数 \\
\bottomrule
\end{tabularx}
\end{minipage}
\end{flushleft}
\bigskip
%===============================================================================

\clearpage

%-------------------------------------------------------------------------------
% momentum/phase0.tex   v1.1  (2025-06-02)
%-------------------------------------------------------------------------------
% CHANGELOG  -- new entry on top (latest -> oldest)
% - 2025-06-02  v1.1 : 「変数のポイント」節を追加
% - 2025-05-31  v1.0 : モメンタム係数 γ_t ベースライン(γ_t=0)
%-------------------------------------------------------------------------------

%=== Phase-0 : モメンタム係数 γ_t ベースライン ================================
\section*{Phase 0:モメンタム係数 $\gamma_t$ ベースライン}\nopagebreak[4]
%────────────────────────────────────
\begin{flushleft}
\[
  \boxed{\gamma_t = 0}
\]
\end{flushleft}

\subsection*{変数のポイント}
\begin{flushleft}
\begin{itemize}
  \item データ欠損・学習初期のフォールバックとして **常に \(\gamma_t=0\)** を使用。  
        始値と終値のシフトを一切行わない。
\end{itemize}
\end{flushleft}

\subsection*{変数メモ}
\begin{flushleft}
本フェーズは学習初期・データ欠損時のフォールバックとして使用し、  
当日寄り付きと引けの偏位を考慮しない設定($\gamma_t=0$)を採用する。
\end{flushleft}
%===============================================================================

\clearpage

%-------------------------------------------------------------------------------
% event/market/phase1.tex   v1.2  (2025-06-02)
%-------------------------------------------------------------------------------
% CHANGELOG: newest -> oldest
% - 2025-06-02  v1.2 : beta_{i,t}^{(m1)} 表記・section 階層化
% - 2025-05-31  v1.1 : 指標セットを {TOPIX,SPX,USDJPY} に縮小
% - 2025-05-31  v1.0 : 初版(旧 225 系 + DJI)
%-------------------------------------------------------------------------------

%=== Phase 1 : マーケット指標係数 ==============================================
\section*{event / market / Phase 1}\nopagebreak[4]
%────────────────────────────────────
\subsection*{ステップ・目的}
\begin{flushleft}
\begin{enumerate}
  \item \textbf{63 d 相関係数}
        \[
          \rho_t^{(i)}
            =\operatorname{corr}\!\bigl(
              \Delta Cl_{t-62 \ldots t},
              \Delta M_{t-62 \ldots t}^{(i)}
            \bigr)
        \]
  \item \textbf{当日 Z-score}
        \(
          z_t^{(i)}
            =\dfrac{\Delta M_t^{(i)}}{\sigma_{63}^{(i)}}
        \)
  \item \textbf{指標係数(クリップ 0.8--1.2)}
        \[
          \beta_{i,t}^{(m1)}
            =\operatorname{clip}\!\bigl(
               1+\rho_t^{(i)}z_t^{(i)},\,0.8,\,1.2
             \bigr)
        \]
  \item \textbf{イベント係数を更新}
        \[
          \beta_{\text{event},i,t}^{(m1)}
            =\beta_{\text{event},i,t}^{\text{prev}}
             \prod_{i \in S} \beta_{i,t}^{(m1)},
          \quad
          S=\{\text{TOPIX},\text{SPX},\text{USDJPY}\}
        \]
\end{enumerate}
\end{flushleft}

\subsection*{追加変数・係数}
\begin{flushleft}
\begin{minipage}{0.88\textwidth}
\begin{tabularx}{\textwidth}{@{}>{\hfil$\displaystyle}l<{$\hfil}@{\quad}X@{}}
\toprule
記号 & 定義・役割 \\
\midrule
\Delta M_t^{(i)} & 指標 \(i\) の当日リターン \\
\sigma_{63}^{(i)} & 指標 \(i\) の 63 日標準偏差 \\
\rho_t^{(i)} & 63 日相関係数 \\
\beta_{i,t}^{(m1)} & Phase 1 指標係数 (0.8--1.2) \\
\beta_{\text{event},i,t}^{\text{prev}} & 直前フェーズ出力 \\
\beta_{\text{event},i,t}^{(m1)} & Phase 1 出力 (市場要因反映) \\
\bottomrule
\end{tabularx}
\end{minipage}
\end{flushleft}
\bigskip
%===============================================================================
 % κ(σ) Phase 1:段階定数モデル
\clearpage

%-------------------------------------------------------------------------------
% event/market/phase1.tex   v1.2  (2025-06-02)
%-------------------------------------------------------------------------------
% CHANGELOG: newest -> oldest
% - 2025-06-02  v1.2 : beta_{i,t}^{(m1)} 表記・section 階層化
% - 2025-05-31  v1.1 : 指標セットを {TOPIX,SPX,USDJPY} に縮小
% - 2025-05-31  v1.0 : 初版(旧 225 系 + DJI)
%-------------------------------------------------------------------------------

%=== Phase 1 : マーケット指標係数 ==============================================
\section*{event / market / Phase 1}\nopagebreak[4]
%────────────────────────────────────
\subsection*{ステップ・目的}
\begin{flushleft}
\begin{enumerate}
  \item \textbf{63 d 相関係数}
        \[
          \rho_t^{(i)}
            =\operatorname{corr}\!\bigl(
              \Delta Cl_{t-62 \ldots t},
              \Delta M_{t-62 \ldots t}^{(i)}
            \bigr)
        \]
  \item \textbf{当日 Z-score}
        \(
          z_t^{(i)}
            =\dfrac{\Delta M_t^{(i)}}{\sigma_{63}^{(i)}}
        \)
  \item \textbf{指標係数(クリップ 0.8--1.2)}
        \[
          \beta_{i,t}^{(m1)}
            =\operatorname{clip}\!\bigl(
               1+\rho_t^{(i)}z_t^{(i)},\,0.8,\,1.2
             \bigr)
        \]
  \item \textbf{イベント係数を更新}
        \[
          \beta_{\text{event},i,t}^{(m1)}
            =\beta_{\text{event},i,t}^{\text{prev}}
             \prod_{i \in S} \beta_{i,t}^{(m1)},
          \quad
          S=\{\text{TOPIX},\text{SPX},\text{USDJPY}\}
        \]
\end{enumerate}
\end{flushleft}

\subsection*{追加変数・係数}
\begin{flushleft}
\begin{minipage}{0.88\textwidth}
\begin{tabularx}{\textwidth}{@{}>{\hfil$\displaystyle}l<{$\hfil}@{\quad}X@{}}
\toprule
記号 & 定義・役割 \\
\midrule
\Delta M_t^{(i)} & 指標 \(i\) の当日リターン \\
\sigma_{63}^{(i)} & 指標 \(i\) の 63 日標準偏差 \\
\rho_t^{(i)} & 63 日相関係数 \\
\beta_{i,t}^{(m1)} & Phase 1 指標係数 (0.8--1.2) \\
\beta_{\text{event},i,t}^{\text{prev}} & 直前フェーズ出力 \\
\beta_{\text{event},i,t}^{(m1)} & Phase 1 出力 (市場要因反映) \\
\bottomrule
\end{tabularx}
\end{minipage}
\end{flushleft}
\bigskip
%===============================================================================
 % σ_t Phase 1:EWMA14 推定
\clearpage

%-------------------------------------------------------------------------------
% event/phase2.tex   v1.0  (2025-06-09)
%-------------------------------------------------------------------------------
% CHANGELOG
% - 2025-06-09  v1.0 : weekday/earn/market 改良版を追加
%-------------------------------------------------------------------------------

%=== Phase 2 : イベント係数改良版 ===============================================
\section*{event / Phase 2 : 改良版}\nopagebreak[4]
%────────────────────────────────────
\subsection*{ステップ・目的}
\begin{flushleft}
\begin{enumerate}
  \item weekday 係数を自己適応的 EWMA\,(\(\lambda_{\text{wd}}\)) で平滑。
  \item 決算日の近傍 \(\pm2\) 日まで減衰する係数 \(w_d\) を適用。
  \item 市場係数は日経平均VIを加え、42 日相関で EWMA 更新。
\end{enumerate}
\end{flushleft}

\subsection*{追加変数・係数}
\begin{flushleft}
\begin{minipage}{0.90\textwidth}
\begin{tabularx}{\textwidth}{@{}>{\hfil$\displaystyle}l<{$\hfil}@{\quad}X@{}}
\toprule
記号 & 定義・役割 \\
\midrule
\lambda_{\text{wd}} & len(dates)>150 なら 0.92, それ以外 0.88 \\
 w_d & 決算日からの距離による係数 (1.20--1.10) \\
\beta_{\text{market},i,t}^{(m2)} & 42日相関 + VI 平滑後係数 \\
\bottomrule
\end{tabularx}
\end{minipage}
\end{flushleft}
\bigskip
%===============================================================================
 % σ_t Phase 2:IQR スケーリング
\clearpage

%-------------------------------------------------------------------------------
% event/market/phase1.tex   v1.2  (2025-06-02)
%-------------------------------------------------------------------------------
% CHANGELOG: newest -> oldest
% - 2025-06-02  v1.2 : beta_{i,t}^{(m1)} 表記・section 階層化
% - 2025-05-31  v1.1 : 指標セットを {TOPIX,SPX,USDJPY} に縮小
% - 2025-05-31  v1.0 : 初版(旧 225 系 + DJI)
%-------------------------------------------------------------------------------

%=== Phase 1 : マーケット指標係数 ==============================================
\section*{event / market / Phase 1}\nopagebreak[4]
%────────────────────────────────────
\subsection*{ステップ・目的}
\begin{flushleft}
\begin{enumerate}
  \item \textbf{63 d 相関係数}
        \[
          \rho_t^{(i)}
            =\operatorname{corr}\!\bigl(
              \Delta Cl_{t-62 \ldots t},
              \Delta M_{t-62 \ldots t}^{(i)}
            \bigr)
        \]
  \item \textbf{当日 Z-score}
        \(
          z_t^{(i)}
            =\dfrac{\Delta M_t^{(i)}}{\sigma_{63}^{(i)}}
        \)
  \item \textbf{指標係数(クリップ 0.8--1.2)}
        \[
          \beta_{i,t}^{(m1)}
            =\operatorname{clip}\!\bigl(
               1+\rho_t^{(i)}z_t^{(i)},\,0.8,\,1.2
             \bigr)
        \]
  \item \textbf{イベント係数を更新}
        \[
          \beta_{\text{event},i,t}^{(m1)}
            =\beta_{\text{event},i,t}^{\text{prev}}
             \prod_{i \in S} \beta_{i,t}^{(m1)},
          \quad
          S=\{\text{TOPIX},\text{SPX},\text{USDJPY}\}
        \]
\end{enumerate}
\end{flushleft}

\subsection*{追加変数・係数}
\begin{flushleft}
\begin{minipage}{0.88\textwidth}
\begin{tabularx}{\textwidth}{@{}>{\hfil$\displaystyle}l<{$\hfil}@{\quad}X@{}}
\toprule
記号 & 定義・役割 \\
\midrule
\Delta M_t^{(i)} & 指標 \(i\) の当日リターン \\
\sigma_{63}^{(i)} & 指標 \(i\) の 63 日標準偏差 \\
\rho_t^{(i)} & 63 日相関係数 \\
\beta_{i,t}^{(m1)} & Phase 1 指標係数 (0.8--1.2) \\
\beta_{\text{event},i,t}^{\text{prev}} & 直前フェーズ出力 \\
\beta_{\text{event},i,t}^{(m1)} & Phase 1 出力 (市場要因反映) \\
\bottomrule
\end{tabularx}
\end{minipage}
\end{flushleft}
\bigskip
%===============================================================================

\clearpage

%-------------------------------------------------------------------------------
% event/phase2.tex   v1.0  (2025-06-09)
%-------------------------------------------------------------------------------
% CHANGELOG
% - 2025-06-09  v1.0 : weekday/earn/market 改良版を追加
%-------------------------------------------------------------------------------

%=== Phase 2 : イベント係数改良版 ===============================================
\section*{event / Phase 2 : 改良版}\nopagebreak[4]
%────────────────────────────────────
\subsection*{ステップ・目的}
\begin{flushleft}
\begin{enumerate}
  \item weekday 係数を自己適応的 EWMA\,(\(\lambda_{\text{wd}}\)) で平滑。
  \item 決算日の近傍 \(\pm2\) 日まで減衰する係数 \(w_d\) を適用。
  \item 市場係数は日経平均VIを加え、42 日相関で EWMA 更新。
\end{enumerate}
\end{flushleft}

\subsection*{追加変数・係数}
\begin{flushleft}
\begin{minipage}{0.90\textwidth}
\begin{tabularx}{\textwidth}{@{}>{\hfil$\displaystyle}l<{$\hfil}@{\quad}X@{}}
\toprule
記号 & 定義・役割 \\
\midrule
\lambda_{\text{wd}} & len(dates)>150 なら 0.92, それ以外 0.88 \\
 w_d & 決算日からの距離による係数 (1.20--1.10) \\
\beta_{\text{market},i,t}^{(m2)} & 42日相関 + VI 平滑後係数 \\
\bottomrule
\end{tabularx}
\end{minipage}
\end{flushleft}
\bigskip
%===============================================================================
       % Phase-2:σ 比補正
\clearpage

%-------------------------------------------------------------------------------
% event/market/phase3.tex   v1.2  (2025-06-02)
%-------------------------------------------------------------------------------
% CHANGELOG: newest -> oldest
% - 2025-06-02  v1.2 : beta_{63,i} 算出式を明示し ASCII 化
% - 2025-05-31  v1.1 : 書式統一
% - 2025-05-31  v1.0 : 初版
%-------------------------------------------------------------------------------

%=== Phase 3 : 銘柄 beta 補正 ==================================================
\section*{event / market / Phase 3}\nopagebreak[4]
%────────────────────────────────────
\subsection*{ステップ・目的}
\begin{flushleft}
\begin{enumerate}
  \item \textbf{63 d beta を計算}
        \[
          \beta_{63,i}=
          \frac{\operatorname{Cov}\!\bigl(r_i,r_{\text{TOPIX}}\bigr)}
               {\operatorname{Var}\!\bigl(r_{\text{TOPIX}}\bigr)}
        \]
  \item \textbf{補正係数}
        \[
          c_i=
          \begin{cases}
            1.05 & \beta_{63,i}>1.0\\
            0.95 & \beta_{63,i}<0.5\\
            1.00 & \text{otherwise}
          \end{cases}
        \]
  \item \textbf{イベント係数を更新}
        \(
          \beta_{\text{event},i,t}^{(m3)}
            =\beta_{\text{event},i,t}^{(m2)}\,c_i
        \)
\end{enumerate}
\end{flushleft}

\subsection*{追加変数・係数}
\begin{flushleft}
\begin{minipage}{0.88\textwidth}
\begin{tabularx}{\textwidth}{@{}>{\hfil$\displaystyle}l<{$\hfil}@{\quad}X@{}}
\toprule
記号 & 定義・役割 \\
\midrule
\beta_{63,i} & 63 日 TOPIX beta \\
c_i & 補正係数 (0.95 / 1.05) \\
\beta_{\text{event},i,t}^{(m2)} & Phase 2 出力 \\
\beta_{\text{event},i,t}^{(m3)} & Phase 3 出力 \\
\bottomrule
\end{tabularx}
\end{minipage}
\end{flushleft}
\bigskip
%===============================================================================
       % Phase-3:\alpha_t 深掘り
\clearpage

%-------------------------------------------------------------------------------
% center_shift/phase4.tex   v1.0  (2025-06-06)
%-------------------------------------------------------------------------------
% CHANGELOG  -- new entry on top (latest -> oldest)
% - 2025-06-06  v1.0 : 初版
%-------------------------------------------------------------------------------

%=== center_shift =============================================================
\section*{center\_shift}\nopagebreak[4]

%=== Phase 4 : \eta / \lambda の深掘り ======================================
\subsection*{Phase 4:$\eta$ と $\lambda$ の深掘り}\nopagebreak[4]
%────────────────────────────────────
\paragraph{ステップ/目的}
\begin{flushleft}
\begin{enumerate}
  \item \textbf{学習率}
        \(\eta\) は $\lambda_{\text{shift}}$ 更新の歩幅を制御
  \item \textbf{勾配近似}
        \(g_t\approx-\dfrac{2}{30}\sum_{k=1}^{30}e_{t-k}\,\sigma_{t-k}^2\)
  \item \textbf{$\lambda_{\text{shift}}$ 更新}
        \(\lambda_{\text{shift},t}
          =\operatorname{clip}\bigl(\lambda_{\text{shift},t-1}
          -\eta\,g_t,\,0.90,\,0.98\bigr)\)
  \item \textbf{ウォームアップ}
        初期 30~d は固定 $\lambda_{\text{shift}}=0.94$ で安定化
\end{enumerate}
\end{flushleft}

\subsubsection*{変数のポイント}
\begin{flushleft}
\begin{itemize}
  \item 大きすぎる $\eta$ は \(\lambda_{\text{shift}}\) を振動させる
  \item 小さすぎる $\eta$ では収束が遅延
  \item 更新範囲 [0.90, 0.98] を超えないよう \(\operatorname{clip}\)
  \item $|g_t|>10$ なら勾配をクリップし安定化
\end{itemize}
\end{flushleft}

\subsubsection*{実装ヒント}
\begin{flushleft}
\begin{itemize}
  \item 経験的に $\eta=0.01$ が妥当な上限値
  \item 週次で $\eta$ の微調整を試し、予測 MAE を観察
  \item 勾配計算には 30~d の誤差系列を用意
\end{itemize}
\end{flushleft}

\subsubsection*{追加変数・係数}
\begin{flushleft}
\begin{minipage}{0.90\textwidth}
\begin{tabularx}{\textwidth}{@{}>{\hfil$\displaystyle}l<{$\hfil}@{\quad}X@{}}
\toprule
記号 & 定義・役割 \\
\midrule
\eta & 学習率 \\
\lambda_{\text{shift},t} & 更新後 EWMA 定数 \\
\lambda_{\text{shift},t-1} & 前日 EWMA 定数 \\
\sigma_t^2 & 分散推定値 \\
\operatorname{clip} & 範囲制限関数 \\
\end{tabularx}
\end{minipage}
\end{flushleft}
\bigskip
%==============================================================================
       % Phase-4:$\eta$/$\lambda$ 深掘り
\clearpage

%-------------------------------------------------------------------------------
% momentum/phase0.tex   v1.1  (2025-06-02)
%-------------------------------------------------------------------------------
% CHANGELOG  -- new entry on top (latest -> oldest)
% - 2025-06-02  v1.1 : 「変数のポイント」節を追加
% - 2025-05-31  v1.0 : モメンタム係数 γ_t ベースライン(γ_t=0)
%-------------------------------------------------------------------------------

%=== Phase-0 : モメンタム係数 γ_t ベースライン ================================
\section*{Phase 0:モメンタム係数 $\gamma_t$ ベースライン}\nopagebreak[4]
%────────────────────────────────────
\begin{flushleft}
\[
  \boxed{\gamma_t = 0}
\]
\end{flushleft}

\subsection*{変数のポイント}
\begin{flushleft}
\begin{itemize}
  \item データ欠損・学習初期のフォールバックとして **常に \(\gamma_t=0\)** を使用。  
        始値と終値のシフトを一切行わない。
\end{itemize}
\end{flushleft}

\subsection*{変数メモ}
\begin{flushleft}
本フェーズは学習初期・データ欠損時のフォールバックとして使用し、  
当日寄り付きと引けの偏位を考慮しない設定($\gamma_t=0$)を採用する。
\end{flushleft}
%===============================================================================
             % Phase-0:半レンジ m_t
\clearpage

%-------------------------------------------------------------------------------
% event/market/phase1.tex   v1.2  (2025-06-02)
%-------------------------------------------------------------------------------
% CHANGELOG: newest -> oldest
% - 2025-06-02  v1.2 : beta_{i,t}^{(m1)} 表記・section 階層化
% - 2025-05-31  v1.1 : 指標セットを {TOPIX,SPX,USDJPY} に縮小
% - 2025-05-31  v1.0 : 初版(旧 225 系 + DJI)
%-------------------------------------------------------------------------------

%=== Phase 1 : マーケット指標係数 ==============================================
\section*{event / market / Phase 1}\nopagebreak[4]
%────────────────────────────────────
\subsection*{ステップ・目的}
\begin{flushleft}
\begin{enumerate}
  \item \textbf{63 d 相関係数}
        \[
          \rho_t^{(i)}
            =\operatorname{corr}\!\bigl(
              \Delta Cl_{t-62 \ldots t},
              \Delta M_{t-62 \ldots t}^{(i)}
            \bigr)
        \]
  \item \textbf{当日 Z-score}
        \(
          z_t^{(i)}
            =\dfrac{\Delta M_t^{(i)}}{\sigma_{63}^{(i)}}
        \)
  \item \textbf{指標係数(クリップ 0.8--1.2)}
        \[
          \beta_{i,t}^{(m1)}
            =\operatorname{clip}\!\bigl(
               1+\rho_t^{(i)}z_t^{(i)},\,0.8,\,1.2
             \bigr)
        \]
  \item \textbf{イベント係数を更新}
        \[
          \beta_{\text{event},i,t}^{(m1)}
            =\beta_{\text{event},i,t}^{\text{prev}}
             \prod_{i \in S} \beta_{i,t}^{(m1)},
          \quad
          S=\{\text{TOPIX},\text{SPX},\text{USDJPY}\}
        \]
\end{enumerate}
\end{flushleft}

\subsection*{追加変数・係数}
\begin{flushleft}
\begin{minipage}{0.88\textwidth}
\begin{tabularx}{\textwidth}{@{}>{\hfil$\displaystyle}l<{$\hfil}@{\quad}X@{}}
\toprule
記号 & 定義・役割 \\
\midrule
\Delta M_t^{(i)} & 指標 \(i\) の当日リターン \\
\sigma_{63}^{(i)} & 指標 \(i\) の 63 日標準偏差 \\
\rho_t^{(i)} & 63 日相関係数 \\
\beta_{i,t}^{(m1)} & Phase 1 指標係数 (0.8--1.2) \\
\beta_{\text{event},i,t}^{\text{prev}} & 直前フェーズ出力 \\
\beta_{\text{event},i,t}^{(m1)} & Phase 1 出力 (市場要因反映) \\
\bottomrule
\end{tabularx}
\end{minipage}
\end{flushleft}
\bigskip
%===============================================================================
             % Phase-1:σ 比補正
\clearpage

%-------------------------------------------------------------------------------
% event/phase2.tex   v1.0  (2025-06-09)
%-------------------------------------------------------------------------------
% CHANGELOG
% - 2025-06-09  v1.0 : weekday/earn/market 改良版を追加
%-------------------------------------------------------------------------------

%=== Phase 2 : イベント係数改良版 ===============================================
\section*{event / Phase 2 : 改良版}\nopagebreak[4]
%────────────────────────────────────
\subsection*{ステップ・目的}
\begin{flushleft}
\begin{enumerate}
  \item weekday 係数を自己適応的 EWMA\,(\(\lambda_{\text{wd}}\)) で平滑。
  \item 決算日の近傍 \(\pm2\) 日まで減衰する係数 \(w_d\) を適用。
  \item 市場係数は日経平均VIを加え、42 日相関で EWMA 更新。
\end{enumerate}
\end{flushleft}

\subsection*{追加変数・係数}
\begin{flushleft}
\begin{minipage}{0.90\textwidth}
\begin{tabularx}{\textwidth}{@{}>{\hfil$\displaystyle}l<{$\hfil}@{\quad}X@{}}
\toprule
記号 & 定義・役割 \\
\midrule
\lambda_{\text{wd}} & len(dates)>150 なら 0.92, それ以外 0.88 \\
 w_d & 決算日からの距離による係数 (1.20--1.10) \\
\beta_{\text{market},i,t}^{(m2)} & 42日相関 + VI 平滑後係数 \\
\bottomrule
\end{tabularx}
\end{minipage}
\end{flushleft}
\bigskip
%===============================================================================
             % Phase-2:ボラ依存補正
\clearpage

%-------------------------------------------------------------------------------
% event/market/phase3.tex   v1.2  (2025-06-02)
%-------------------------------------------------------------------------------
% CHANGELOG: newest -> oldest
% - 2025-06-02  v1.2 : beta_{63,i} 算出式を明示し ASCII 化
% - 2025-05-31  v1.1 : 書式統一
% - 2025-05-31  v1.0 : 初版
%-------------------------------------------------------------------------------

%=== Phase 3 : 銘柄 beta 補正 ==================================================
\section*{event / market / Phase 3}\nopagebreak[4]
%────────────────────────────────────
\subsection*{ステップ・目的}
\begin{flushleft}
\begin{enumerate}
  \item \textbf{63 d beta を計算}
        \[
          \beta_{63,i}=
          \frac{\operatorname{Cov}\!\bigl(r_i,r_{\text{TOPIX}}\bigr)}
               {\operatorname{Var}\!\bigl(r_{\text{TOPIX}}\bigr)}
        \]
  \item \textbf{補正係数}
        \[
          c_i=
          \begin{cases}
            1.05 & \beta_{63,i}>1.0\\
            0.95 & \beta_{63,i}<0.5\\
            1.00 & \text{otherwise}
          \end{cases}
        \]
  \item \textbf{イベント係数を更新}
        \(
          \beta_{\text{event},i,t}^{(m3)}
            =\beta_{\text{event},i,t}^{(m2)}\,c_i
        \)
\end{enumerate}
\end{flushleft}

\subsection*{追加変数・係数}
\begin{flushleft}
\begin{minipage}{0.88\textwidth}
\begin{tabularx}{\textwidth}{@{}>{\hfil$\displaystyle}l<{$\hfil}@{\quad}X@{}}
\toprule
記号 & 定義・役割 \\
\midrule
\beta_{63,i} & 63 日 TOPIX beta \\
c_i & 補正係数 (0.95 / 1.05) \\
\beta_{\text{event},i,t}^{(m2)} & Phase 2 出力 \\
\beta_{\text{event},i,t}^{(m3)} & Phase 3 出力 \\
\bottomrule
\end{tabularx}
\end{minipage}
\end{flushleft}
\bigskip
%===============================================================================
             % Phase-3:イベント/曜日バイアス
\clearpage

%-------------------------------------------------------------------------------
% center_shift/phase4.tex   v1.0  (2025-06-06)
%-------------------------------------------------------------------------------
% CHANGELOG  -- new entry on top (latest -> oldest)
% - 2025-06-06  v1.0 : 初版
%-------------------------------------------------------------------------------

%=== center_shift =============================================================
\section*{center\_shift}\nopagebreak[4]

%=== Phase 4 : \eta / \lambda の深掘り ======================================
\subsection*{Phase 4:$\eta$ と $\lambda$ の深掘り}\nopagebreak[4]
%────────────────────────────────────
\paragraph{ステップ/目的}
\begin{flushleft}
\begin{enumerate}
  \item \textbf{学習率}
        \(\eta\) は $\lambda_{\text{shift}}$ 更新の歩幅を制御
  \item \textbf{勾配近似}
        \(g_t\approx-\dfrac{2}{30}\sum_{k=1}^{30}e_{t-k}\,\sigma_{t-k}^2\)
  \item \textbf{$\lambda_{\text{shift}}$ 更新}
        \(\lambda_{\text{shift},t}
          =\operatorname{clip}\bigl(\lambda_{\text{shift},t-1}
          -\eta\,g_t,\,0.90,\,0.98\bigr)\)
  \item \textbf{ウォームアップ}
        初期 30~d は固定 $\lambda_{\text{shift}}=0.94$ で安定化
\end{enumerate}
\end{flushleft}

\subsubsection*{変数のポイント}
\begin{flushleft}
\begin{itemize}
  \item 大きすぎる $\eta$ は \(\lambda_{\text{shift}}\) を振動させる
  \item 小さすぎる $\eta$ では収束が遅延
  \item 更新範囲 [0.90, 0.98] を超えないよう \(\operatorname{clip}\)
  \item $|g_t|>10$ なら勾配をクリップし安定化
\end{itemize}
\end{flushleft}

\subsubsection*{実装ヒント}
\begin{flushleft}
\begin{itemize}
  \item 経験的に $\eta=0.01$ が妥当な上限値
  \item 週次で $\eta$ の微調整を試し、予測 MAE を観察
  \item 勾配計算には 30~d の誤差系列を用意
\end{itemize}
\end{flushleft}

\subsubsection*{追加変数・係数}
\begin{flushleft}
\begin{minipage}{0.90\textwidth}
\begin{tabularx}{\textwidth}{@{}>{\hfil$\displaystyle}l<{$\hfil}@{\quad}X@{}}
\toprule
記号 & 定義・役割 \\
\midrule
\eta & 学習率 \\
\lambda_{\text{shift},t} & 更新後 EWMA 定数 \\
\lambda_{\text{shift},t-1} & 前日 EWMA 定数 \\
\sigma_t^2 & 分散推定値 \\
\operatorname{clip} & 範囲制限関数 \\
\end{tabularx}
\end{minipage}
\end{flushleft}
\bigskip
%==============================================================================
             % Phase-4:自己適応 λ_vol 更新
\clearpage

%-------------------------------------------------------------------------------
% event/earn/phase5.tex   v1.1  (2025-06-02)
%-------------------------------------------------------------------------------
% CHANGELOG  -- newest -> oldest
% - 2025-06-02  v1.1 : ASCII 統一, beta^{final} 表記, clip 修正
% - 2025-05-31  v1.0 : 初版(Bayes 縮小)
%-------------------------------------------------------------------------------

%=== Phase 5 : w_profit ベイズ縮小 =============================================
\section*{event / earn / Phase 5}\nopagebreak[4]
%────────────────────────────────────
\subsection*{ステップ・目的}
\begin{flushleft}
\begin{enumerate}
  \item \textbf{サンプル数取得}\;
        \( n_i=\text{count\_earnings}(i,\text{last 3Y}) \)

  \item \textbf{セクター平均重み}\;
        \( \bar w_{\text{profit},s}=\operatorname{mean}(w_{\text{profit},j}) \)

  \item \textbf{Bayes 縮小}\;
        \[
          \tilde w_{\text{profit},i}
            =\frac{n_i}{n_i+\tau}\,w_{\text{profit},i}
             +\frac{\tau}{n_i+\tau}\,\bar w_{\text{profit},s},
          \quad \tau = 10
        \]
        \( \tilde w_{\text{profit},i}=\operatorname{clip}(\tilde w_{\text{profit},i},0.50,0.90) \)

  \item \textbf{サプライズ率再計算} → $\beta_{\text{earn},i,t}^{(5)}$ を取得。

  \item \textbf{イベント係数最終更新}\;
        \[
          \beta_{\text{event},i,t}^{\text{final}}
            =\beta_{\text{event},i,t}^{(4)}\,
             \beta_{\text{earn},i,t}^{(5)}
        \]
\end{enumerate}
\end{flushleft}

\subsection*{追加変数・係数}
\begin{flushleft}
\begin{minipage}{0.92\textwidth}
\begin{tabularx}{\textwidth}{@{}>{\hfil$\displaystyle}l<{$\hfil}@{\quad}X@{}}
\toprule
記号 & 定義・役割 \\
\midrule
n_i & 過去 3 年の決算サンプル数 \\
\tau & 縮小ハイパーパラメータ (10) \\
\bar w_{\text{profit},s} & セクター平均利益重み \\
\beta_{\text{event},i,t}^{\text{final}} & earn 系最終係数 \\
\bottomrule
\end{tabularx}
\end{minipage}
\end{flushleft}
\bigskip
%===============================================================================
             % Phase-5:残差補正
\clearpage

%-------------------------------------------------------------------------------
% center_shift/phase6.tex   v1.0  (2025-06-13)
%-------------------------------------------------------------------------------
% CHANGELOG  -- new entry on top (latest -> oldest)
% - 2025-06-13  v1.0 : 10日移動平均を用いた外れ値平滑化
%-------------------------------------------------------------------------------

%=== center_shift =============================================================
\section*{center\_shift}\nopagebreak[4]

%=== Phase 6 : 10 日移動平均による平滑化 =====================================
\subsection*{Phase 6:10 日移動平均による平滑化}\nopagebreak[4]
%────────────────────────────────────
\paragraph{ステップ/目的}
\begin{flushleft}
\begin{enumerate}
  \item \textbf{10DMA}\; \( \overline{C_{\Delta}/C_r} = \frac{1}{10}\sum_{k=0}^{9} \frac{C_{\Delta,t-k}}{C_{r,t-k}} \)
  \item \textbf{判定基準}\; \( \left|\overline{C_{\Delta}/C_r}\right| \ge 1\% \) なら外れ値
\end{enumerate}
\end{flushleft}

\subsubsection*{変数のポイント}
\begin{flushleft}
\begin{itemize}
  \item 短期ノイズを抑えるため直近 10 日の平均を採用
  \item 単日で 1\% を超えても平均が小さければ外れ値扱いしない
\end{itemize}
\end{flushleft}

             % Phase-6:騰落キャップ
\clearpage

%-------------------------------------------------------------------------------
% momentum/phase0.tex   v1.1  (2025-06-02)
%-------------------------------------------------------------------------------
% CHANGELOG  -- new entry on top (latest -> oldest)
% - 2025-06-02  v1.1 : 「変数のポイント」節を追加
% - 2025-05-31  v1.0 : モメンタム係数 γ_t ベースライン(γ_t=0)
%-------------------------------------------------------------------------------

%=== Phase-0 : モメンタム係数 γ_t ベースライン ================================
\section*{Phase 0:モメンタム係数 $\gamma_t$ ベースライン}\nopagebreak[4]
%────────────────────────────────────
\begin{flushleft}
\[
  \boxed{\gamma_t = 0}
\]
\end{flushleft}

\subsection*{変数のポイント}
\begin{flushleft}
\begin{itemize}
  \item データ欠損・学習初期のフォールバックとして **常に \(\gamma_t=0\)** を使用。  
        始値と終値のシフトを一切行わない。
\end{itemize}
\end{flushleft}

\subsection*{変数メモ}
\begin{flushleft}
本フェーズは学習初期・データ欠損時のフォールバックとして使用し、  
当日寄り付きと引けの偏位を考慮しない設定($\gamma_t=0$)を採用する。
\end{flushleft}
%===============================================================================

\clearpage

%-------------------------------------------------------------------------------
% event/phase2.tex   v1.0  (2025-06-09)
%-------------------------------------------------------------------------------
% CHANGELOG
% - 2025-06-09  v1.0 : weekday/earn/market 改良版を追加
%-------------------------------------------------------------------------------

%=== Phase 2 : イベント係数改良版 ===============================================
\section*{event / Phase 2 : 改良版}\nopagebreak[4]
%────────────────────────────────────
\subsection*{ステップ・目的}
\begin{flushleft}
\begin{enumerate}
  \item weekday 係数を自己適応的 EWMA\,(\(\lambda_{\text{wd}}\)) で平滑。
  \item 決算日の近傍 \(\pm2\) 日まで減衰する係数 \(w_d\) を適用。
  \item 市場係数は日経平均VIを加え、42 日相関で EWMA 更新。
\end{enumerate}
\end{flushleft}

\subsection*{追加変数・係数}
\begin{flushleft}
\begin{minipage}{0.90\textwidth}
\begin{tabularx}{\textwidth}{@{}>{\hfil$\displaystyle}l<{$\hfil}@{\quad}X@{}}
\toprule
記号 & 定義・役割 \\
\midrule
\lambda_{\text{wd}} & len(dates)>150 なら 0.92, それ以外 0.88 \\
 w_d & 決算日からの距離による係数 (1.20--1.10) \\
\beta_{\text{market},i,t}^{(m2)} & 42日相関 + VI 平滑後係数 \\
\bottomrule
\end{tabularx}
\end{minipage}
\end{flushleft}
\bigskip
%===============================================================================

\clearpage

%-------------------------------------------------------------------------------
% event/market/phase1.tex   v1.2  (2025-06-02)
%-------------------------------------------------------------------------------
% CHANGELOG: newest -> oldest
% - 2025-06-02  v1.2 : beta_{i,t}^{(m1)} 表記・section 階層化
% - 2025-05-31  v1.1 : 指標セットを {TOPIX,SPX,USDJPY} に縮小
% - 2025-05-31  v1.0 : 初版(旧 225 系 + DJI)
%-------------------------------------------------------------------------------

%=== Phase 1 : マーケット指標係数 ==============================================
\section*{event / market / Phase 1}\nopagebreak[4]
%────────────────────────────────────
\subsection*{ステップ・目的}
\begin{flushleft}
\begin{enumerate}
  \item \textbf{63 d 相関係数}
        \[
          \rho_t^{(i)}
            =\operatorname{corr}\!\bigl(
              \Delta Cl_{t-62 \ldots t},
              \Delta M_{t-62 \ldots t}^{(i)}
            \bigr)
        \]
  \item \textbf{当日 Z-score}
        \(
          z_t^{(i)}
            =\dfrac{\Delta M_t^{(i)}}{\sigma_{63}^{(i)}}
        \)
  \item \textbf{指標係数(クリップ 0.8--1.2)}
        \[
          \beta_{i,t}^{(m1)}
            =\operatorname{clip}\!\bigl(
               1+\rho_t^{(i)}z_t^{(i)},\,0.8,\,1.2
             \bigr)
        \]
  \item \textbf{イベント係数を更新}
        \[
          \beta_{\text{event},i,t}^{(m1)}
            =\beta_{\text{event},i,t}^{\text{prev}}
             \prod_{i \in S} \beta_{i,t}^{(m1)},
          \quad
          S=\{\text{TOPIX},\text{SPX},\text{USDJPY}\}
        \]
\end{enumerate}
\end{flushleft}

\subsection*{追加変数・係数}
\begin{flushleft}
\begin{minipage}{0.88\textwidth}
\begin{tabularx}{\textwidth}{@{}>{\hfil$\displaystyle}l<{$\hfil}@{\quad}X@{}}
\toprule
記号 & 定義・役割 \\
\midrule
\Delta M_t^{(i)} & 指標 \(i\) の当日リターン \\
\sigma_{63}^{(i)} & 指標 \(i\) の 63 日標準偏差 \\
\rho_t^{(i)} & 63 日相関係数 \\
\beta_{i,t}^{(m1)} & Phase 1 指標係数 (0.8--1.2) \\
\beta_{\text{event},i,t}^{\text{prev}} & 直前フェーズ出力 \\
\beta_{\text{event},i,t}^{(m1)} & Phase 1 出力 (市場要因反映) \\
\bottomrule
\end{tabularx}
\end{minipage}
\end{flushleft}
\bigskip
%===============================================================================
       % Phase-1:イベント係数テーブル
\clearpage

%-------------------------------------------------------------------------------
% event/phase2.tex   v1.0  (2025-06-09)
%-------------------------------------------------------------------------------
% CHANGELOG
% - 2025-06-09  v1.0 : weekday/earn/market 改良版を追加
%-------------------------------------------------------------------------------

%=== Phase 2 : イベント係数改良版 ===============================================
\section*{event / Phase 2 : 改良版}\nopagebreak[4]
%────────────────────────────────────
\subsection*{ステップ・目的}
\begin{flushleft}
\begin{enumerate}
  \item weekday 係数を自己適応的 EWMA\,(\(\lambda_{\text{wd}}\)) で平滑。
  \item 決算日の近傍 \(\pm2\) 日まで減衰する係数 \(w_d\) を適用。
  \item 市場係数は日経平均VIを加え、42 日相関で EWMA 更新。
\end{enumerate}
\end{flushleft}

\subsection*{追加変数・係数}
\begin{flushleft}
\begin{minipage}{0.90\textwidth}
\begin{tabularx}{\textwidth}{@{}>{\hfil$\displaystyle}l<{$\hfil}@{\quad}X@{}}
\toprule
記号 & 定義・役割 \\
\midrule
\lambda_{\text{wd}} & len(dates)>150 なら 0.92, それ以外 0.88 \\
 w_d & 決算日からの距離による係数 (1.20--1.10) \\
\beta_{\text{market},i,t}^{(m2)} & 42日相関 + VI 平滑後係数 \\
\bottomrule
\end{tabularx}
\end{minipage}
\end{flushleft}
\bigskip
%===============================================================================
       % Phase-2:イベント係数テーブル
\clearpage

%-------------------------------------------------------------------------------
% event/market/phase1.tex   v1.2  (2025-06-02)
%-------------------------------------------------------------------------------
% CHANGELOG: newest -> oldest
% - 2025-06-02  v1.2 : beta_{i,t}^{(m1)} 表記・section 階層化
% - 2025-05-31  v1.1 : 指標セットを {TOPIX,SPX,USDJPY} に縮小
% - 2025-05-31  v1.0 : 初版(旧 225 系 + DJI)
%-------------------------------------------------------------------------------

%=== Phase 1 : マーケット指標係数 ==============================================
\section*{event / market / Phase 1}\nopagebreak[4]
%────────────────────────────────────
\subsection*{ステップ・目的}
\begin{flushleft}
\begin{enumerate}
  \item \textbf{63 d 相関係数}
        \[
          \rho_t^{(i)}
            =\operatorname{corr}\!\bigl(
              \Delta Cl_{t-62 \ldots t},
              \Delta M_{t-62 \ldots t}^{(i)}
            \bigr)
        \]
  \item \textbf{当日 Z-score}
        \(
          z_t^{(i)}
            =\dfrac{\Delta M_t^{(i)}}{\sigma_{63}^{(i)}}
        \)
  \item \textbf{指標係数(クリップ 0.8--1.2)}
        \[
          \beta_{i,t}^{(m1)}
            =\operatorname{clip}\!\bigl(
               1+\rho_t^{(i)}z_t^{(i)},\,0.8,\,1.2
             \bigr)
        \]
  \item \textbf{イベント係数を更新}
        \[
          \beta_{\text{event},i,t}^{(m1)}
            =\beta_{\text{event},i,t}^{\text{prev}}
             \prod_{i \in S} \beta_{i,t}^{(m1)},
          \quad
          S=\{\text{TOPIX},\text{SPX},\text{USDJPY}\}
        \]
\end{enumerate}
\end{flushleft}

\subsection*{追加変数・係数}
\begin{flushleft}
\begin{minipage}{0.88\textwidth}
\begin{tabularx}{\textwidth}{@{}>{\hfil$\displaystyle}l<{$\hfil}@{\quad}X@{}}
\toprule
記号 & 定義・役割 \\
\midrule
\Delta M_t^{(i)} & 指標 \(i\) の当日リターン \\
\sigma_{63}^{(i)} & 指標 \(i\) の 63 日標準偏差 \\
\rho_t^{(i)} & 63 日相関係数 \\
\beta_{i,t}^{(m1)} & Phase 1 指標係数 (0.8--1.2) \\
\beta_{\text{event},i,t}^{\text{prev}} & 直前フェーズ出力 \\
\beta_{\text{event},i,t}^{(m1)} & Phase 1 出力 (市場要因反映) \\
\bottomrule
\end{tabularx}
\end{minipage}
\end{flushleft}
\bigskip
%===============================================================================
 % Phase-1:祝日係数テーブル
\clearpage

%-------------------------------------------------------------------------------
% event/phase2.tex   v1.0  (2025-06-09)
%-------------------------------------------------------------------------------
% CHANGELOG
% - 2025-06-09  v1.0 : weekday/earn/market 改良版を追加
%-------------------------------------------------------------------------------

%=== Phase 2 : イベント係数改良版 ===============================================
\section*{event / Phase 2 : 改良版}\nopagebreak[4]
%────────────────────────────────────
\subsection*{ステップ・目的}
\begin{flushleft}
\begin{enumerate}
  \item weekday 係数を自己適応的 EWMA\,(\(\lambda_{\text{wd}}\)) で平滑。
  \item 決算日の近傍 \(\pm2\) 日まで減衰する係数 \(w_d\) を適用。
  \item 市場係数は日経平均VIを加え、42 日相関で EWMA 更新。
\end{enumerate}
\end{flushleft}

\subsection*{追加変数・係数}
\begin{flushleft}
\begin{minipage}{0.90\textwidth}
\begin{tabularx}{\textwidth}{@{}>{\hfil$\displaystyle}l<{$\hfil}@{\quad}X@{}}
\toprule
記号 & 定義・役割 \\
\midrule
\lambda_{\text{wd}} & len(dates)>150 なら 0.92, それ以外 0.88 \\
 w_d & 決算日からの距離による係数 (1.20--1.10) \\
\beta_{\text{market},i,t}^{(m2)} & 42日相関 + VI 平滑後係数 \\
\bottomrule
\end{tabularx}
\end{minipage}
\end{flushleft}
\bigskip
%===============================================================================
 % Phase-2:祝日係数テーブル
\clearpage

%-------------------------------------------------------------------------------
% event/market/phase3.tex   v1.2  (2025-06-02)
%-------------------------------------------------------------------------------
% CHANGELOG: newest -> oldest
% - 2025-06-02  v1.2 : beta_{63,i} 算出式を明示し ASCII 化
% - 2025-05-31  v1.1 : 書式統一
% - 2025-05-31  v1.0 : 初版
%-------------------------------------------------------------------------------

%=== Phase 3 : 銘柄 beta 補正 ==================================================
\section*{event / market / Phase 3}\nopagebreak[4]
%────────────────────────────────────
\subsection*{ステップ・目的}
\begin{flushleft}
\begin{enumerate}
  \item \textbf{63 d beta を計算}
        \[
          \beta_{63,i}=
          \frac{\operatorname{Cov}\!\bigl(r_i,r_{\text{TOPIX}}\bigr)}
               {\operatorname{Var}\!\bigl(r_{\text{TOPIX}}\bigr)}
        \]
  \item \textbf{補正係数}
        \[
          c_i=
          \begin{cases}
            1.05 & \beta_{63,i}>1.0\\
            0.95 & \beta_{63,i}<0.5\\
            1.00 & \text{otherwise}
          \end{cases}
        \]
  \item \textbf{イベント係数を更新}
        \(
          \beta_{\text{event},i,t}^{(m3)}
            =\beta_{\text{event},i,t}^{(m2)}\,c_i
        \)
\end{enumerate}
\end{flushleft}

\subsection*{追加変数・係数}
\begin{flushleft}
\begin{minipage}{0.88\textwidth}
\begin{tabularx}{\textwidth}{@{}>{\hfil$\displaystyle}l<{$\hfil}@{\quad}X@{}}
\toprule
記号 & 定義・役割 \\
\midrule
\beta_{63,i} & 63 日 TOPIX beta \\
c_i & 補正係数 (0.95 / 1.05) \\
\beta_{\text{event},i,t}^{(m2)} & Phase 2 出力 \\
\beta_{\text{event},i,t}^{(m3)} & Phase 3 出力 \\
\bottomrule
\end{tabularx}
\end{minipage}
\end{flushleft}
\bigskip
%===============================================================================
 % Phase-3:祝日係数テーブル
\clearpage

%-------------------------------------------------------------------------------
% event/market/phase3.tex   v1.2  (2025-06-02)
%-------------------------------------------------------------------------------
% CHANGELOG: newest -> oldest
% - 2025-06-02  v1.2 : beta_{63,i} 算出式を明示し ASCII 化
% - 2025-05-31  v1.1 : 書式統一
% - 2025-05-31  v1.0 : 初版
%-------------------------------------------------------------------------------

%=== Phase 3 : 銘柄 beta 補正 ==================================================
\section*{event / market / Phase 3}\nopagebreak[4]
%────────────────────────────────────
\subsection*{ステップ・目的}
\begin{flushleft}
\begin{enumerate}
  \item \textbf{63 d beta を計算}
        \[
          \beta_{63,i}=
          \frac{\operatorname{Cov}\!\bigl(r_i,r_{\text{TOPIX}}\bigr)}
               {\operatorname{Var}\!\bigl(r_{\text{TOPIX}}\bigr)}
        \]
  \item \textbf{補正係数}
        \[
          c_i=
          \begin{cases}
            1.05 & \beta_{63,i}>1.0\\
            0.95 & \beta_{63,i}<0.5\\
            1.00 & \text{otherwise}
          \end{cases}
        \]
  \item \textbf{イベント係数を更新}
        \(
          \beta_{\text{event},i,t}^{(m3)}
            =\beta_{\text{event},i,t}^{(m2)}\,c_i
        \)
\end{enumerate}
\end{flushleft}

\subsection*{追加変数・係数}
\begin{flushleft}
\begin{minipage}{0.88\textwidth}
\begin{tabularx}{\textwidth}{@{}>{\hfil$\displaystyle}l<{$\hfil}@{\quad}X@{}}
\toprule
記号 & 定義・役割 \\
\midrule
\beta_{63,i} & 63 日 TOPIX beta \\
c_i & 補正係数 (0.95 / 1.05) \\
\beta_{\text{event},i,t}^{(m2)} & Phase 2 出力 \\
\beta_{\text{event},i,t}^{(m3)} & Phase 3 出力 \\
\bottomrule
\end{tabularx}
\end{minipage}
\end{flushleft}
\bigskip
%===============================================================================
       % Phase-3:イベント係数テーブル
\clearpage

%-------------------------------------------------------------------------------
% event/market/phase1.tex   v1.2  (2025-06-02)
%-------------------------------------------------------------------------------
% CHANGELOG: newest -> oldest
% - 2025-06-02  v1.2 : beta_{i,t}^{(m1)} 表記・section 階層化
% - 2025-05-31  v1.1 : 指標セットを {TOPIX,SPX,USDJPY} に縮小
% - 2025-05-31  v1.0 : 初版(旧 225 系 + DJI)
%-------------------------------------------------------------------------------

%=== Phase 1 : マーケット指標係数 ==============================================
\section*{event / market / Phase 1}\nopagebreak[4]
%────────────────────────────────────
\subsection*{ステップ・目的}
\begin{flushleft}
\begin{enumerate}
  \item \textbf{63 d 相関係数}
        \[
          \rho_t^{(i)}
            =\operatorname{corr}\!\bigl(
              \Delta Cl_{t-62 \ldots t},
              \Delta M_{t-62 \ldots t}^{(i)}
            \bigr)
        \]
  \item \textbf{当日 Z-score}
        \(
          z_t^{(i)}
            =\dfrac{\Delta M_t^{(i)}}{\sigma_{63}^{(i)}}
        \)
  \item \textbf{指標係数(クリップ 0.8--1.2)}
        \[
          \beta_{i,t}^{(m1)}
            =\operatorname{clip}\!\bigl(
               1+\rho_t^{(i)}z_t^{(i)},\,0.8,\,1.2
             \bigr)
        \]
  \item \textbf{イベント係数を更新}
        \[
          \beta_{\text{event},i,t}^{(m1)}
            =\beta_{\text{event},i,t}^{\text{prev}}
             \prod_{i \in S} \beta_{i,t}^{(m1)},
          \quad
          S=\{\text{TOPIX},\text{SPX},\text{USDJPY}\}
        \]
\end{enumerate}
\end{flushleft}

\subsection*{追加変数・係数}
\begin{flushleft}
\begin{minipage}{0.88\textwidth}
\begin{tabularx}{\textwidth}{@{}>{\hfil$\displaystyle}l<{$\hfil}@{\quad}X@{}}
\toprule
記号 & 定義・役割 \\
\midrule
\Delta M_t^{(i)} & 指標 \(i\) の当日リターン \\
\sigma_{63}^{(i)} & 指標 \(i\) の 63 日標準偏差 \\
\rho_t^{(i)} & 63 日相関係数 \\
\beta_{i,t}^{(m1)} & Phase 1 指標係数 (0.8--1.2) \\
\beta_{\text{event},i,t}^{\text{prev}} & 直前フェーズ出力 \\
\beta_{\text{event},i,t}^{(m1)} & Phase 1 出力 (市場要因反映) \\
\bottomrule
\end{tabularx}
\end{minipage}
\end{flushleft}
\bigskip
%===============================================================================
       % Phase-1:イベント係数テーブル
\clearpage

%-------------------------------------------------------------------------------
% event/phase2.tex   v1.0  (2025-06-09)
%-------------------------------------------------------------------------------
% CHANGELOG
% - 2025-06-09  v1.0 : weekday/earn/market 改良版を追加
%-------------------------------------------------------------------------------

%=== Phase 2 : イベント係数改良版 ===============================================
\section*{event / Phase 2 : 改良版}\nopagebreak[4]
%────────────────────────────────────
\subsection*{ステップ・目的}
\begin{flushleft}
\begin{enumerate}
  \item weekday 係数を自己適応的 EWMA\,(\(\lambda_{\text{wd}}\)) で平滑。
  \item 決算日の近傍 \(\pm2\) 日まで減衰する係数 \(w_d\) を適用。
  \item 市場係数は日経平均VIを加え、42 日相関で EWMA 更新。
\end{enumerate}
\end{flushleft}

\subsection*{追加変数・係数}
\begin{flushleft}
\begin{minipage}{0.90\textwidth}
\begin{tabularx}{\textwidth}{@{}>{\hfil$\displaystyle}l<{$\hfil}@{\quad}X@{}}
\toprule
記号 & 定義・役割 \\
\midrule
\lambda_{\text{wd}} & len(dates)>150 なら 0.92, それ以外 0.88 \\
 w_d & 決算日からの距離による係数 (1.20--1.10) \\
\beta_{\text{market},i,t}^{(m2)} & 42日相関 + VI 平滑後係数 \\
\bottomrule
\end{tabularx}
\end{minipage}
\end{flushleft}
\bigskip
%===============================================================================
       % Phase-2:イベント係数テーブル
\clearpage

%-------------------------------------------------------------------------------
% event/market/phase3.tex   v1.2  (2025-06-02)
%-------------------------------------------------------------------------------
% CHANGELOG: newest -> oldest
% - 2025-06-02  v1.2 : beta_{63,i} 算出式を明示し ASCII 化
% - 2025-05-31  v1.1 : 書式統一
% - 2025-05-31  v1.0 : 初版
%-------------------------------------------------------------------------------

%=== Phase 3 : 銘柄 beta 補正 ==================================================
\section*{event / market / Phase 3}\nopagebreak[4]
%────────────────────────────────────
\subsection*{ステップ・目的}
\begin{flushleft}
\begin{enumerate}
  \item \textbf{63 d beta を計算}
        \[
          \beta_{63,i}=
          \frac{\operatorname{Cov}\!\bigl(r_i,r_{\text{TOPIX}}\bigr)}
               {\operatorname{Var}\!\bigl(r_{\text{TOPIX}}\bigr)}
        \]
  \item \textbf{補正係数}
        \[
          c_i=
          \begin{cases}
            1.05 & \beta_{63,i}>1.0\\
            0.95 & \beta_{63,i}<0.5\\
            1.00 & \text{otherwise}
          \end{cases}
        \]
  \item \textbf{イベント係数を更新}
        \(
          \beta_{\text{event},i,t}^{(m3)}
            =\beta_{\text{event},i,t}^{(m2)}\,c_i
        \)
\end{enumerate}
\end{flushleft}

\subsection*{追加変数・係数}
\begin{flushleft}
\begin{minipage}{0.88\textwidth}
\begin{tabularx}{\textwidth}{@{}>{\hfil$\displaystyle}l<{$\hfil}@{\quad}X@{}}
\toprule
記号 & 定義・役割 \\
\midrule
\beta_{63,i} & 63 日 TOPIX beta \\
c_i & 補正係数 (0.95 / 1.05) \\
\beta_{\text{event},i,t}^{(m2)} & Phase 2 出力 \\
\beta_{\text{event},i,t}^{(m3)} & Phase 3 出力 \\
\bottomrule
\end{tabularx}
\end{minipage}
\end{flushleft}
\bigskip
%===============================================================================
       % Phase-3:イベント係数テーブル
\clearpage

%-------------------------------------------------------------------------------
% center_shift/phase4.tex   v1.0  (2025-06-06)
%-------------------------------------------------------------------------------
% CHANGELOG  -- new entry on top (latest -> oldest)
% - 2025-06-06  v1.0 : 初版
%-------------------------------------------------------------------------------

%=== center_shift =============================================================
\section*{center\_shift}\nopagebreak[4]

%=== Phase 4 : \eta / \lambda の深掘り ======================================
\subsection*{Phase 4:$\eta$ と $\lambda$ の深掘り}\nopagebreak[4]
%────────────────────────────────────
\paragraph{ステップ/目的}
\begin{flushleft}
\begin{enumerate}
  \item \textbf{学習率}
        \(\eta\) は $\lambda_{\text{shift}}$ 更新の歩幅を制御
  \item \textbf{勾配近似}
        \(g_t\approx-\dfrac{2}{30}\sum_{k=1}^{30}e_{t-k}\,\sigma_{t-k}^2\)
  \item \textbf{$\lambda_{\text{shift}}$ 更新}
        \(\lambda_{\text{shift},t}
          =\operatorname{clip}\bigl(\lambda_{\text{shift},t-1}
          -\eta\,g_t,\,0.90,\,0.98\bigr)\)
  \item \textbf{ウォームアップ}
        初期 30~d は固定 $\lambda_{\text{shift}}=0.94$ で安定化
\end{enumerate}
\end{flushleft}

\subsubsection*{変数のポイント}
\begin{flushleft}
\begin{itemize}
  \item 大きすぎる $\eta$ は \(\lambda_{\text{shift}}\) を振動させる
  \item 小さすぎる $\eta$ では収束が遅延
  \item 更新範囲 [0.90, 0.98] を超えないよう \(\operatorname{clip}\)
  \item $|g_t|>10$ なら勾配をクリップし安定化
\end{itemize}
\end{flushleft}

\subsubsection*{実装ヒント}
\begin{flushleft}
\begin{itemize}
  \item 経験的に $\eta=0.01$ が妥当な上限値
  \item 週次で $\eta$ の微調整を試し、予測 MAE を観察
  \item 勾配計算には 30~d の誤差系列を用意
\end{itemize}
\end{flushleft}

\subsubsection*{追加変数・係数}
\begin{flushleft}
\begin{minipage}{0.90\textwidth}
\begin{tabularx}{\textwidth}{@{}>{\hfil$\displaystyle}l<{$\hfil}@{\quad}X@{}}
\toprule
記号 & 定義・役割 \\
\midrule
\eta & 学習率 \\
\lambda_{\text{shift},t} & 更新後 EWMA 定数 \\
\lambda_{\text{shift},t-1} & 前日 EWMA 定数 \\
\sigma_t^2 & 分散推定値 \\
\operatorname{clip} & 範囲制限関数 \\
\end{tabularx}
\end{minipage}
\end{flushleft}
\bigskip
%==============================================================================
       % Phase-4:イベント係数テーブル
\clearpage

%-------------------------------------------------------------------------------
% event/earn/phase5.tex   v1.1  (2025-06-02)
%-------------------------------------------------------------------------------
% CHANGELOG  -- newest -> oldest
% - 2025-06-02  v1.1 : ASCII 統一, beta^{final} 表記, clip 修正
% - 2025-05-31  v1.0 : 初版(Bayes 縮小)
%-------------------------------------------------------------------------------

%=== Phase 5 : w_profit ベイズ縮小 =============================================
\section*{event / earn / Phase 5}\nopagebreak[4]
%────────────────────────────────────
\subsection*{ステップ・目的}
\begin{flushleft}
\begin{enumerate}
  \item \textbf{サンプル数取得}\;
        \( n_i=\text{count\_earnings}(i,\text{last 3Y}) \)

  \item \textbf{セクター平均重み}\;
        \( \bar w_{\text{profit},s}=\operatorname{mean}(w_{\text{profit},j}) \)

  \item \textbf{Bayes 縮小}\;
        \[
          \tilde w_{\text{profit},i}
            =\frac{n_i}{n_i+\tau}\,w_{\text{profit},i}
             +\frac{\tau}{n_i+\tau}\,\bar w_{\text{profit},s},
          \quad \tau = 10
        \]
        \( \tilde w_{\text{profit},i}=\operatorname{clip}(\tilde w_{\text{profit},i},0.50,0.90) \)

  \item \textbf{サプライズ率再計算} → $\beta_{\text{earn},i,t}^{(5)}$ を取得。

  \item \textbf{イベント係数最終更新}\;
        \[
          \beta_{\text{event},i,t}^{\text{final}}
            =\beta_{\text{event},i,t}^{(4)}\,
             \beta_{\text{earn},i,t}^{(5)}
        \]
\end{enumerate}
\end{flushleft}

\subsection*{追加変数・係数}
\begin{flushleft}
\begin{minipage}{0.92\textwidth}
\begin{tabularx}{\textwidth}{@{}>{\hfil$\displaystyle}l<{$\hfil}@{\quad}X@{}}
\toprule
記号 & 定義・役割 \\
\midrule
n_i & 過去 3 年の決算サンプル数 \\
\tau & 縮小ハイパーパラメータ (10) \\
\bar w_{\text{profit},s} & セクター平均利益重み \\
\beta_{\text{event},i,t}^{\text{final}} & earn 系最終係数 \\
\bottomrule
\end{tabularx}
\end{minipage}
\end{flushleft}
\bigskip
%===============================================================================
       % Phase-:イベント係数テーブル
\clearpage

%-------------------------------------------------------------------------------
% event/market/phase1.tex   v1.2  (2025-06-02)
%-------------------------------------------------------------------------------
% CHANGELOG: newest -> oldest
% - 2025-06-02  v1.2 : beta_{i,t}^{(m1)} 表記・section 階層化
% - 2025-05-31  v1.1 : 指標セットを {TOPIX,SPX,USDJPY} に縮小
% - 2025-05-31  v1.0 : 初版(旧 225 系 + DJI)
%-------------------------------------------------------------------------------

%=== Phase 1 : マーケット指標係数 ==============================================
\section*{event / market / Phase 1}\nopagebreak[4]
%────────────────────────────────────
\subsection*{ステップ・目的}
\begin{flushleft}
\begin{enumerate}
  \item \textbf{63 d 相関係数}
        \[
          \rho_t^{(i)}
            =\operatorname{corr}\!\bigl(
              \Delta Cl_{t-62 \ldots t},
              \Delta M_{t-62 \ldots t}^{(i)}
            \bigr)
        \]
  \item \textbf{当日 Z-score}
        \(
          z_t^{(i)}
            =\dfrac{\Delta M_t^{(i)}}{\sigma_{63}^{(i)}}
        \)
  \item \textbf{指標係数(クリップ 0.8--1.2)}
        \[
          \beta_{i,t}^{(m1)}
            =\operatorname{clip}\!\bigl(
               1+\rho_t^{(i)}z_t^{(i)},\,0.8,\,1.2
             \bigr)
        \]
  \item \textbf{イベント係数を更新}
        \[
          \beta_{\text{event},i,t}^{(m1)}
            =\beta_{\text{event},i,t}^{\text{prev}}
             \prod_{i \in S} \beta_{i,t}^{(m1)},
          \quad
          S=\{\text{TOPIX},\text{SPX},\text{USDJPY}\}
        \]
\end{enumerate}
\end{flushleft}

\subsection*{追加変数・係数}
\begin{flushleft}
\begin{minipage}{0.88\textwidth}
\begin{tabularx}{\textwidth}{@{}>{\hfil$\displaystyle}l<{$\hfil}@{\quad}X@{}}
\toprule
記号 & 定義・役割 \\
\midrule
\Delta M_t^{(i)} & 指標 \(i\) の当日リターン \\
\sigma_{63}^{(i)} & 指標 \(i\) の 63 日標準偏差 \\
\rho_t^{(i)} & 63 日相関係数 \\
\beta_{i,t}^{(m1)} & Phase 1 指標係数 (0.8--1.2) \\
\beta_{\text{event},i,t}^{\text{prev}} & 直前フェーズ出力 \\
\beta_{\text{event},i,t}^{(m1)} & Phase 1 出力 (市場要因反映) \\
\bottomrule
\end{tabularx}
\end{minipage}
\end{flushleft}
\bigskip
%===============================================================================
       % Phase-1:イベント係数テーブル
\clearpage

%-------------------------------------------------------------------------------
% event/phase2.tex   v1.0  (2025-06-09)
%-------------------------------------------------------------------------------
% CHANGELOG
% - 2025-06-09  v1.0 : weekday/earn/market 改良版を追加
%-------------------------------------------------------------------------------

%=== Phase 2 : イベント係数改良版 ===============================================
\section*{event / Phase 2 : 改良版}\nopagebreak[4]
%────────────────────────────────────
\subsection*{ステップ・目的}
\begin{flushleft}
\begin{enumerate}
  \item weekday 係数を自己適応的 EWMA\,(\(\lambda_{\text{wd}}\)) で平滑。
  \item 決算日の近傍 \(\pm2\) 日まで減衰する係数 \(w_d\) を適用。
  \item 市場係数は日経平均VIを加え、42 日相関で EWMA 更新。
\end{enumerate}
\end{flushleft}

\subsection*{追加変数・係数}
\begin{flushleft}
\begin{minipage}{0.90\textwidth}
\begin{tabularx}{\textwidth}{@{}>{\hfil$\displaystyle}l<{$\hfil}@{\quad}X@{}}
\toprule
記号 & 定義・役割 \\
\midrule
\lambda_{\text{wd}} & len(dates)>150 なら 0.92, それ以外 0.88 \\
 w_d & 決算日からの距離による係数 (1.20--1.10) \\
\beta_{\text{market},i,t}^{(m2)} & 42日相関 + VI 平滑後係数 \\
\bottomrule
\end{tabularx}
\end{minipage}
\end{flushleft}
\bigskip
%===============================================================================
       % Phase-2:イベント係数テーブル
\clearpage

%-------------------------------------------------------------------------------
% event/market/phase3.tex   v1.2  (2025-06-02)
%-------------------------------------------------------------------------------
% CHANGELOG: newest -> oldest
% - 2025-06-02  v1.2 : beta_{63,i} 算出式を明示し ASCII 化
% - 2025-05-31  v1.1 : 書式統一
% - 2025-05-31  v1.0 : 初版
%-------------------------------------------------------------------------------

%=== Phase 3 : 銘柄 beta 補正 ==================================================
\section*{event / market / Phase 3}\nopagebreak[4]
%────────────────────────────────────
\subsection*{ステップ・目的}
\begin{flushleft}
\begin{enumerate}
  \item \textbf{63 d beta を計算}
        \[
          \beta_{63,i}=
          \frac{\operatorname{Cov}\!\bigl(r_i,r_{\text{TOPIX}}\bigr)}
               {\operatorname{Var}\!\bigl(r_{\text{TOPIX}}\bigr)}
        \]
  \item \textbf{補正係数}
        \[
          c_i=
          \begin{cases}
            1.05 & \beta_{63,i}>1.0\\
            0.95 & \beta_{63,i}<0.5\\
            1.00 & \text{otherwise}
          \end{cases}
        \]
  \item \textbf{イベント係数を更新}
        \(
          \beta_{\text{event},i,t}^{(m3)}
            =\beta_{\text{event},i,t}^{(m2)}\,c_i
        \)
\end{enumerate}
\end{flushleft}

\subsection*{追加変数・係数}
\begin{flushleft}
\begin{minipage}{0.88\textwidth}
\begin{tabularx}{\textwidth}{@{}>{\hfil$\displaystyle}l<{$\hfil}@{\quad}X@{}}
\toprule
記号 & 定義・役割 \\
\midrule
\beta_{63,i} & 63 日 TOPIX beta \\
c_i & 補正係数 (0.95 / 1.05) \\
\beta_{\text{event},i,t}^{(m2)} & Phase 2 出力 \\
\beta_{\text{event},i,t}^{(m3)} & Phase 3 出力 \\
\bottomrule
\end{tabularx}
\end{minipage}
\end{flushleft}
\bigskip
%===============================================================================
       % Phase-3:イベント係数テーブル
\clearpage

%-------------------------------------------------------------------------------
% center_shift/phase4.tex   v1.0  (2025-06-06)
%-------------------------------------------------------------------------------
% CHANGELOG  -- new entry on top (latest -> oldest)
% - 2025-06-06  v1.0 : 初版
%-------------------------------------------------------------------------------

%=== center_shift =============================================================
\section*{center\_shift}\nopagebreak[4]

%=== Phase 4 : \eta / \lambda の深掘り ======================================
\subsection*{Phase 4:$\eta$ と $\lambda$ の深掘り}\nopagebreak[4]
%────────────────────────────────────
\paragraph{ステップ/目的}
\begin{flushleft}
\begin{enumerate}
  \item \textbf{学習率}
        \(\eta\) は $\lambda_{\text{shift}}$ 更新の歩幅を制御
  \item \textbf{勾配近似}
        \(g_t\approx-\dfrac{2}{30}\sum_{k=1}^{30}e_{t-k}\,\sigma_{t-k}^2\)
  \item \textbf{$\lambda_{\text{shift}}$ 更新}
        \(\lambda_{\text{shift},t}
          =\operatorname{clip}\bigl(\lambda_{\text{shift},t-1}
          -\eta\,g_t,\,0.90,\,0.98\bigr)\)
  \item \textbf{ウォームアップ}
        初期 30~d は固定 $\lambda_{\text{shift}}=0.94$ で安定化
\end{enumerate}
\end{flushleft}

\subsubsection*{変数のポイント}
\begin{flushleft}
\begin{itemize}
  \item 大きすぎる $\eta$ は \(\lambda_{\text{shift}}\) を振動させる
  \item 小さすぎる $\eta$ では収束が遅延
  \item 更新範囲 [0.90, 0.98] を超えないよう \(\operatorname{clip}\)
  \item $|g_t|>10$ なら勾配をクリップし安定化
\end{itemize}
\end{flushleft}

\subsubsection*{実装ヒント}
\begin{flushleft}
\begin{itemize}
  \item 経験的に $\eta=0.01$ が妥当な上限値
  \item 週次で $\eta$ の微調整を試し、予測 MAE を観察
  \item 勾配計算には 30~d の誤差系列を用意
\end{itemize}
\end{flushleft}

\subsubsection*{追加変数・係数}
\begin{flushleft}
\begin{minipage}{0.90\textwidth}
\begin{tabularx}{\textwidth}{@{}>{\hfil$\displaystyle}l<{$\hfil}@{\quad}X@{}}
\toprule
記号 & 定義・役割 \\
\midrule
\eta & 学習率 \\
\lambda_{\text{shift},t} & 更新後 EWMA 定数 \\
\lambda_{\text{shift},t-1} & 前日 EWMA 定数 \\
\sigma_t^2 & 分散推定値 \\
\operatorname{clip} & 範囲制限関数 \\
\end{tabularx}
\end{minipage}
\end{flushleft}
\bigskip
%==============================================================================
       % Phase-4:イベント係数テーブル
\clearpage

%-------------------------------------------------------------------------------
% event/earn/phase5.tex   v1.1  (2025-06-02)
%-------------------------------------------------------------------------------
% CHANGELOG  -- newest -> oldest
% - 2025-06-02  v1.1 : ASCII 統一, beta^{final} 表記, clip 修正
% - 2025-05-31  v1.0 : 初版(Bayes 縮小)
%-------------------------------------------------------------------------------

%=== Phase 5 : w_profit ベイズ縮小 =============================================
\section*{event / earn / Phase 5}\nopagebreak[4]
%────────────────────────────────────
\subsection*{ステップ・目的}
\begin{flushleft}
\begin{enumerate}
  \item \textbf{サンプル数取得}\;
        \( n_i=\text{count\_earnings}(i,\text{last 3Y}) \)

  \item \textbf{セクター平均重み}\;
        \( \bar w_{\text{profit},s}=\operatorname{mean}(w_{\text{profit},j}) \)

  \item \textbf{Bayes 縮小}\;
        \[
          \tilde w_{\text{profit},i}
            =\frac{n_i}{n_i+\tau}\,w_{\text{profit},i}
             +\frac{\tau}{n_i+\tau}\,\bar w_{\text{profit},s},
          \quad \tau = 10
        \]
        \( \tilde w_{\text{profit},i}=\operatorname{clip}(\tilde w_{\text{profit},i},0.50,0.90) \)

  \item \textbf{サプライズ率再計算} → $\beta_{\text{earn},i,t}^{(5)}$ を取得。

  \item \textbf{イベント係数最終更新}\;
        \[
          \beta_{\text{event},i,t}^{\text{final}}
            =\beta_{\text{event},i,t}^{(4)}\,
             \beta_{\text{earn},i,t}^{(5)}
        \]
\end{enumerate}
\end{flushleft}

\subsection*{追加変数・係数}
\begin{flushleft}
\begin{minipage}{0.92\textwidth}
\begin{tabularx}{\textwidth}{@{}>{\hfil$\displaystyle}l<{$\hfil}@{\quad}X@{}}
\toprule
記号 & 定義・役割 \\
\midrule
n_i & 過去 3 年の決算サンプル数 \\
\tau & 縮小ハイパーパラメータ (10) \\
\bar w_{\text{profit},s} & セクター平均利益重み \\
\beta_{\text{event},i,t}^{\text{final}} & earn 系最終係数 \\
\bottomrule
\end{tabularx}
\end{minipage}
\end{flushleft}
\bigskip
%===============================================================================
       % Phase-5:イベント係数テーブル
\clearpage

%-------------------------------------------------------------------------------
% momentum/phase0.tex   v1.1  (2025-06-02)
%-------------------------------------------------------------------------------
% CHANGELOG  -- new entry on top (latest -> oldest)
% - 2025-06-02  v1.1 : 「変数のポイント」節を追加
% - 2025-05-31  v1.0 : モメンタム係数 γ_t ベースライン(γ_t=0)
%-------------------------------------------------------------------------------

%=== Phase-0 : モメンタム係数 γ_t ベースライン ================================
\section*{Phase 0:モメンタム係数 $\gamma_t$ ベースライン}\nopagebreak[4]
%────────────────────────────────────
\begin{flushleft}
\[
  \boxed{\gamma_t = 0}
\]
\end{flushleft}

\subsection*{変数のポイント}
\begin{flushleft}
\begin{itemize}
  \item データ欠損・学習初期のフォールバックとして **常に \(\gamma_t=0\)** を使用。  
        始値と終値のシフトを一切行わない。
\end{itemize}
\end{flushleft}

\subsection*{変数メモ}
\begin{flushleft}
本フェーズは学習初期・データ欠損時のフォールバックとして使用し、  
当日寄り付きと引けの偏位を考慮しない設定($\gamma_t=0$)を採用する。
\end{flushleft}
%===============================================================================

\clearpage

%-------------------------------------------------------------------------------
% event/market/phase1.tex   v1.2  (2025-06-02)
%-------------------------------------------------------------------------------
% CHANGELOG: newest -> oldest
% - 2025-06-02  v1.2 : beta_{i,t}^{(m1)} 表記・section 階層化
% - 2025-05-31  v1.1 : 指標セットを {TOPIX,SPX,USDJPY} に縮小
% - 2025-05-31  v1.0 : 初版(旧 225 系 + DJI)
%-------------------------------------------------------------------------------

%=== Phase 1 : マーケット指標係数 ==============================================
\section*{event / market / Phase 1}\nopagebreak[4]
%────────────────────────────────────
\subsection*{ステップ・目的}
\begin{flushleft}
\begin{enumerate}
  \item \textbf{63 d 相関係数}
        \[
          \rho_t^{(i)}
            =\operatorname{corr}\!\bigl(
              \Delta Cl_{t-62 \ldots t},
              \Delta M_{t-62 \ldots t}^{(i)}
            \bigr)
        \]
  \item \textbf{当日 Z-score}
        \(
          z_t^{(i)}
            =\dfrac{\Delta M_t^{(i)}}{\sigma_{63}^{(i)}}
        \)
  \item \textbf{指標係数(クリップ 0.8--1.2)}
        \[
          \beta_{i,t}^{(m1)}
            =\operatorname{clip}\!\bigl(
               1+\rho_t^{(i)}z_t^{(i)},\,0.8,\,1.2
             \bigr)
        \]
  \item \textbf{イベント係数を更新}
        \[
          \beta_{\text{event},i,t}^{(m1)}
            =\beta_{\text{event},i,t}^{\text{prev}}
             \prod_{i \in S} \beta_{i,t}^{(m1)},
          \quad
          S=\{\text{TOPIX},\text{SPX},\text{USDJPY}\}
        \]
\end{enumerate}
\end{flushleft}

\subsection*{追加変数・係数}
\begin{flushleft}
\begin{minipage}{0.88\textwidth}
\begin{tabularx}{\textwidth}{@{}>{\hfil$\displaystyle}l<{$\hfil}@{\quad}X@{}}
\toprule
記号 & 定義・役割 \\
\midrule
\Delta M_t^{(i)} & 指標 \(i\) の当日リターン \\
\sigma_{63}^{(i)} & 指標 \(i\) の 63 日標準偏差 \\
\rho_t^{(i)} & 63 日相関係数 \\
\beta_{i,t}^{(m1)} & Phase 1 指標係数 (0.8--1.2) \\
\beta_{\text{event},i,t}^{\text{prev}} & 直前フェーズ出力 \\
\beta_{\text{event},i,t}^{(m1)} & Phase 1 出力 (市場要因反映) \\
\bottomrule
\end{tabularx}
\end{minipage}
\end{flushleft}
\bigskip
%===============================================================================
       % Phase-1:EMA5 符号
\clearpage

%-------------------------------------------------------------------------------
% event/phase2.tex   v1.0  (2025-06-09)
%-------------------------------------------------------------------------------
% CHANGELOG
% - 2025-06-09  v1.0 : weekday/earn/market 改良版を追加
%-------------------------------------------------------------------------------

%=== Phase 2 : イベント係数改良版 ===============================================
\section*{event / Phase 2 : 改良版}\nopagebreak[4]
%────────────────────────────────────
\subsection*{ステップ・目的}
\begin{flushleft}
\begin{enumerate}
  \item weekday 係数を自己適応的 EWMA\,(\(\lambda_{\text{wd}}\)) で平滑。
  \item 決算日の近傍 \(\pm2\) 日まで減衰する係数 \(w_d\) を適用。
  \item 市場係数は日経平均VIを加え、42 日相関で EWMA 更新。
\end{enumerate}
\end{flushleft}

\subsection*{追加変数・係数}
\begin{flushleft}
\begin{minipage}{0.90\textwidth}
\begin{tabularx}{\textwidth}{@{}>{\hfil$\displaystyle}l<{$\hfil}@{\quad}X@{}}
\toprule
記号 & 定義・役割 \\
\midrule
\lambda_{\text{wd}} & len(dates)>150 なら 0.92, それ以外 0.88 \\
 w_d & 決算日からの距離による係数 (1.20--1.10) \\
\beta_{\text{market},i,t}^{(m2)} & 42日相関 + VI 平滑後係数 \\
\bottomrule
\end{tabularx}
\end{minipage}
\end{flushleft}
\bigskip
%===============================================================================
       % Phase-2:σ 比補正
\clearpage

%-------------------------------------------------------------------------------
% event/market/phase3.tex   v1.2  (2025-06-02)
%-------------------------------------------------------------------------------
% CHANGELOG: newest -> oldest
% - 2025-06-02  v1.2 : beta_{63,i} 算出式を明示し ASCII 化
% - 2025-05-31  v1.1 : 書式統一
% - 2025-05-31  v1.0 : 初版
%-------------------------------------------------------------------------------

%=== Phase 3 : 銘柄 beta 補正 ==================================================
\section*{event / market / Phase 3}\nopagebreak[4]
%────────────────────────────────────
\subsection*{ステップ・目的}
\begin{flushleft}
\begin{enumerate}
  \item \textbf{63 d beta を計算}
        \[
          \beta_{63,i}=
          \frac{\operatorname{Cov}\!\bigl(r_i,r_{\text{TOPIX}}\bigr)}
               {\operatorname{Var}\!\bigl(r_{\text{TOPIX}}\bigr)}
        \]
  \item \textbf{補正係数}
        \[
          c_i=
          \begin{cases}
            1.05 & \beta_{63,i}>1.0\\
            0.95 & \beta_{63,i}<0.5\\
            1.00 & \text{otherwise}
          \end{cases}
        \]
  \item \textbf{イベント係数を更新}
        \(
          \beta_{\text{event},i,t}^{(m3)}
            =\beta_{\text{event},i,t}^{(m2)}\,c_i
        \)
\end{enumerate}
\end{flushleft}

\subsection*{追加変数・係数}
\begin{flushleft}
\begin{minipage}{0.88\textwidth}
\begin{tabularx}{\textwidth}{@{}>{\hfil$\displaystyle}l<{$\hfil}@{\quad}X@{}}
\toprule
記号 & 定義・役割 \\
\midrule
\beta_{63,i} & 63 日 TOPIX beta \\
c_i & 補正係数 (0.95 / 1.05) \\
\beta_{\text{event},i,t}^{(m2)} & Phase 2 出力 \\
\beta_{\text{event},i,t}^{(m3)} & Phase 3 出力 \\
\bottomrule
\end{tabularx}
\end{minipage}
\end{flushleft}
\bigskip
%===============================================================================
       % Phase-3:ボラレジーム補正
\clearpage

%-------------------------------------------------------------------------------
% center_shift/phase4.tex   v1.0  (2025-06-06)
%-------------------------------------------------------------------------------
% CHANGELOG  -- new entry on top (latest -> oldest)
% - 2025-06-06  v1.0 : 初版
%-------------------------------------------------------------------------------

%=== center_shift =============================================================
\section*{center\_shift}\nopagebreak[4]

%=== Phase 4 : \eta / \lambda の深掘り ======================================
\subsection*{Phase 4:$\eta$ と $\lambda$ の深掘り}\nopagebreak[4]
%────────────────────────────────────
\paragraph{ステップ/目的}
\begin{flushleft}
\begin{enumerate}
  \item \textbf{学習率}
        \(\eta\) は $\lambda_{\text{shift}}$ 更新の歩幅を制御
  \item \textbf{勾配近似}
        \(g_t\approx-\dfrac{2}{30}\sum_{k=1}^{30}e_{t-k}\,\sigma_{t-k}^2\)
  \item \textbf{$\lambda_{\text{shift}}$ 更新}
        \(\lambda_{\text{shift},t}
          =\operatorname{clip}\bigl(\lambda_{\text{shift},t-1}
          -\eta\,g_t,\,0.90,\,0.98\bigr)\)
  \item \textbf{ウォームアップ}
        初期 30~d は固定 $\lambda_{\text{shift}}=0.94$ で安定化
\end{enumerate}
\end{flushleft}

\subsubsection*{変数のポイント}
\begin{flushleft}
\begin{itemize}
  \item 大きすぎる $\eta$ は \(\lambda_{\text{shift}}\) を振動させる
  \item 小さすぎる $\eta$ では収束が遅延
  \item 更新範囲 [0.90, 0.98] を超えないよう \(\operatorname{clip}\)
  \item $|g_t|>10$ なら勾配をクリップし安定化
\end{itemize}
\end{flushleft}

\subsubsection*{実装ヒント}
\begin{flushleft}
\begin{itemize}
  \item 経験的に $\eta=0.01$ が妥当な上限値
  \item 週次で $\eta$ の微調整を試し、予測 MAE を観察
  \item 勾配計算には 30~d の誤差系列を用意
\end{itemize}
\end{flushleft}

\subsubsection*{追加変数・係数}
\begin{flushleft}
\begin{minipage}{0.90\textwidth}
\begin{tabularx}{\textwidth}{@{}>{\hfil$\displaystyle}l<{$\hfil}@{\quad}X@{}}
\toprule
記号 & 定義・役割 \\
\midrule
\eta & 学習率 \\
\lambda_{\text{shift},t} & 更新後 EWMA 定数 \\
\lambda_{\text{shift},t-1} & 前日 EWMA 定数 \\
\sigma_t^2 & 分散推定値 \\
\operatorname{clip} & 範囲制限関数 \\
\end{tabularx}
\end{minipage}
\end{flushleft}
\bigskip
%==============================================================================
       % Phase-4:λ_gamma 自己適応
\clearpage

%-------------------------------------------------------------------------------
% event/earn/phase5.tex   v1.1  (2025-06-02)
%-------------------------------------------------------------------------------
% CHANGELOG  -- newest -> oldest
% - 2025-06-02  v1.1 : ASCII 統一, beta^{final} 表記, clip 修正
% - 2025-05-31  v1.0 : 初版(Bayes 縮小)
%-------------------------------------------------------------------------------

%=== Phase 5 : w_profit ベイズ縮小 =============================================
\section*{event / earn / Phase 5}\nopagebreak[4]
%────────────────────────────────────
\subsection*{ステップ・目的}
\begin{flushleft}
\begin{enumerate}
  \item \textbf{サンプル数取得}\;
        \( n_i=\text{count\_earnings}(i,\text{last 3Y}) \)

  \item \textbf{セクター平均重み}\;
        \( \bar w_{\text{profit},s}=\operatorname{mean}(w_{\text{profit},j}) \)

  \item \textbf{Bayes 縮小}\;
        \[
          \tilde w_{\text{profit},i}
            =\frac{n_i}{n_i+\tau}\,w_{\text{profit},i}
             +\frac{\tau}{n_i+\tau}\,\bar w_{\text{profit},s},
          \quad \tau = 10
        \]
        \( \tilde w_{\text{profit},i}=\operatorname{clip}(\tilde w_{\text{profit},i},0.50,0.90) \)

  \item \textbf{サプライズ率再計算} → $\beta_{\text{earn},i,t}^{(5)}$ を取得。

  \item \textbf{イベント係数最終更新}\;
        \[
          \beta_{\text{event},i,t}^{\text{final}}
            =\beta_{\text{event},i,t}^{(4)}\,
             \beta_{\text{earn},i,t}^{(5)}
        \]
\end{enumerate}
\end{flushleft}

\subsection*{追加変数・係数}
\begin{flushleft}
\begin{minipage}{0.92\textwidth}
\begin{tabularx}{\textwidth}{@{}>{\hfil$\displaystyle}l<{$\hfil}@{\quad}X@{}}
\toprule
記号 & 定義・役割 \\
\midrule
n_i & 過去 3 年の決算サンプル数 \\
\tau & 縮小ハイパーパラメータ (10) \\
\bar w_{\text{profit},s} & セクター平均利益重み \\
\beta_{\text{event},i,t}^{\text{final}} & earn 系最終係数 \\
\bottomrule
\end{tabularx}
\end{minipage}
\end{flushleft}
\bigskip
%===============================================================================
       % Phase-5:モメンタム係数テーブル
\clearpage


\end{document}
