%-------------------------------------------------------------------------------
% event/weekday/holiday/phase1.tex   v1.2  (2025-06-02)
%-------------------------------------------------------------------------------
% CHANGELOG  -- newest -> oldest
% - 2025-06-02  v1.2 : beta_holiday,t 表記確認・ASCII 統一
% - 2025-05-31  v1.1 : 書式を weekday/phase2.tex と統一
% - 2025-05-31  v1.0 : 初版(固定祝日係数)
%-------------------------------------------------------------------------------

%=== Phase 1 : 祝日前後固定係数 ===============================================
\section*{event / weekday / holiday / Phase 1}\nopagebreak[4]
%────────────────────────────────────
\subsection*{ステップ・目的}
\begin{flushleft}
\begin{enumerate}
  \item \textbf{休場日判定}\;
        JPX カレンダー JSON で \texttt{HolidayDivision}\(\neq1\) を休場日とする。
  \item \textbf{前後日フラグ抽出}\;
        休場直前営業日フラグ \(h_{t,-1}\),休場明け初日フラグ \(h_{t,+1}\)。
  \item \textbf{祝日係数決定}\;
        \[
          \beta_{\text{holiday},t}=
          \begin{cases}
            0.90 & (h_{t,-1}=1)\\
            0.95 & (h_{t,+1}=1)\\
            1.00 & \text{otherwise}
          \end{cases}
        \]
\end{enumerate}
\end{flushleft}

\subsection*{追加変数・係数}
\begin{flushleft}
\begin{minipage}{0.88\textwidth}
\begin{tabularx}{\textwidth}{@{}>{\hfil$\displaystyle}l<{$\hfil}@{\quad}X@{}}
\toprule
記号 & 定義・役割 \\
\midrule
h_{t,-1},h_{t,+1} & 休日前営業日/休み明け初日フラグ \\
\beta_{\text{holiday},t} & 祝日固定係数 \\
\bottomrule
\end{tabularx}
\end{minipage}
\end{flushleft}
\bigskip
%===============================================================================
