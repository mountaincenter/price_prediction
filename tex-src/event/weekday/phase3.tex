%-------------------------------------------------------------------------------
% event/weekday/phase3.tex   v1.1  (2025-06-02)
%-------------------------------------------------------------------------------
% CHANGELOG  -- newest -> oldest
% - 2025-06-02  v1.1 : beta^{(3)} 表記・集約ロジック明確化
% - 2025-05-31  v1.0 : weekday 系サブフェーズ集約
%-------------------------------------------------------------------------------

%=== Phase 3 : weekday 系集約 ===================================================
\section*{event / weekday / Phase 3}\nopagebreak[4]
%────────────────────────────────────
\subsection*{ステップ・目的}
\begin{flushleft}
\begin{enumerate}
  \item \textbf{holiday 側最終係数を取り込み}\;
        \verb|%-------------------------------------------------------------------------------
% event/weekday/phase3.tex   v1.1  (2025-06-02)
%-------------------------------------------------------------------------------
% CHANGELOG  -- newest -> oldest
% - 2025-06-02  v1.1 : beta^{(3)} 表記・集約ロジック明確化
% - 2025-05-31  v1.0 : weekday 系サブフェーズ集約
%-------------------------------------------------------------------------------

%=== Phase 3 : weekday 系集約 ===================================================
\section*{event / weekday / Phase 3}\nopagebreak[4]
%────────────────────────────────────
\subsection*{ステップ・目的}
\begin{flushleft}
\begin{enumerate}
  \item \textbf{holiday 側最終係数を取り込み}\;
        \verb|%-------------------------------------------------------------------------------
% event/weekday/phase3.tex   v1.1  (2025-06-02)
%-------------------------------------------------------------------------------
% CHANGELOG  -- newest -> oldest
% - 2025-06-02  v1.1 : beta^{(3)} 表記・集約ロジック明確化
% - 2025-05-31  v1.0 : weekday 系サブフェーズ集約
%-------------------------------------------------------------------------------

%=== Phase 3 : weekday 系集約 ===================================================
\section*{event / weekday / Phase 3}\nopagebreak[4]
%────────────────────────────────────
\subsection*{ステップ・目的}
\begin{flushleft}
\begin{enumerate}
  \item \textbf{holiday 側最終係数を取り込み}\;
        \verb|%-------------------------------------------------------------------------------
% event/weekday/phase3.tex   v1.1  (2025-06-02)
%-------------------------------------------------------------------------------
% CHANGELOG  -- newest -> oldest
% - 2025-06-02  v1.1 : beta^{(3)} 表記・集約ロジック明確化
% - 2025-05-31  v1.0 : weekday 系サブフェーズ集約
%-------------------------------------------------------------------------------

%=== Phase 3 : weekday 系集約 ===================================================
\section*{event / weekday / Phase 3}\nopagebreak[4]
%────────────────────────────────────
\subsection*{ステップ・目的}
\begin{flushleft}
\begin{enumerate}
  \item \textbf{holiday 側最終係数を取り込み}\;
        \verb|\input{event/weekday/holiday/phase3}| で  
        \(\tilde\beta_{\text{weekday},i,t}\) を取得。
  \item \textbf{最終 weekday 係数を宣言}\;
        \[
          \boxed{\beta_{\text{weekday},i,t}^{(3)}
          =\tilde\beta_{\text{weekday},i,t}}
        \]
  \item \textbf{イベント係数パイプラインへ出力}\;
        event/phase0.tex が  
        \(\beta_{\text{weekday},i,t}^{(3)}\) を利用。
\end{enumerate}
\end{flushleft}

\subsection*{追加変数・係数}
\begin{flushleft}
\begin{minipage}{0.88\textwidth}
\begin{tabularx}{\textwidth}{@{}>{\hfil$\displaystyle}l<{$\hfil}@{\quad}X@{}}
\toprule
記号 & 定義・役割 \\
\midrule
\tilde\beta_{\text{weekday},i,t} & holiday/phase3 出力係数 \\
\beta_{\text{weekday},i,t}^{(3)} & weekday 系最終係数 (本フェーズ) \\
\bottomrule
\end{tabularx}
\end{minipage}
\end{flushleft}
\bigskip
%===============================================================================
| で  
        \(\tilde\beta_{\text{weekday},i,t}\) を取得。
  \item \textbf{最終 weekday 係数を宣言}\;
        \[
          \boxed{\beta_{\text{weekday},i,t}^{(3)}
          =\tilde\beta_{\text{weekday},i,t}}
        \]
  \item \textbf{イベント係数パイプラインへ出力}\;
        event/phase0.tex が  
        \(\beta_{\text{weekday},i,t}^{(3)}\) を利用。
\end{enumerate}
\end{flushleft}

\subsection*{追加変数・係数}
\begin{flushleft}
\begin{minipage}{0.88\textwidth}
\begin{tabularx}{\textwidth}{@{}>{\hfil$\displaystyle}l<{$\hfil}@{\quad}X@{}}
\toprule
記号 & 定義・役割 \\
\midrule
\tilde\beta_{\text{weekday},i,t} & holiday/phase3 出力係数 \\
\beta_{\text{weekday},i,t}^{(3)} & weekday 系最終係数 (本フェーズ) \\
\bottomrule
\end{tabularx}
\end{minipage}
\end{flushleft}
\bigskip
%===============================================================================
| で  
        \(\tilde\beta_{\text{weekday},i,t}\) を取得。
  \item \textbf{最終 weekday 係数を宣言}\;
        \[
          \boxed{\beta_{\text{weekday},i,t}^{(3)}
          =\tilde\beta_{\text{weekday},i,t}}
        \]
  \item \textbf{イベント係数パイプラインへ出力}\;
        event/phase0.tex が  
        \(\beta_{\text{weekday},i,t}^{(3)}\) を利用。
\end{enumerate}
\end{flushleft}

\subsection*{追加変数・係数}
\begin{flushleft}
\begin{minipage}{0.88\textwidth}
\begin{tabularx}{\textwidth}{@{}>{\hfil$\displaystyle}l<{$\hfil}@{\quad}X@{}}
\toprule
記号 & 定義・役割 \\
\midrule
\tilde\beta_{\text{weekday},i,t} & holiday/phase3 出力係数 \\
\beta_{\text{weekday},i,t}^{(3)} & weekday 系最終係数 (本フェーズ) \\
\bottomrule
\end{tabularx}
\end{minipage}
\end{flushleft}
\bigskip
%===============================================================================
| で  
        \(\tilde\beta_{\text{weekday},i,t}\) を取得。
  \item \textbf{最終 weekday 係数を宣言}\;
        \[
          \boxed{\beta_{\text{weekday},i,t}^{(3)}
          =\tilde\beta_{\text{weekday},i,t}}
        \]
  \item \textbf{イベント係数パイプラインへ出力}\;
        event/phase0.tex が  
        \(\beta_{\text{weekday},i,t}^{(3)}\) を利用。
\end{enumerate}
\end{flushleft}

\subsection*{追加変数・係数}
\begin{flushleft}
\begin{minipage}{0.88\textwidth}
\begin{tabularx}{\textwidth}{@{}>{\hfil$\displaystyle}l<{$\hfil}@{\quad}X@{}}
\toprule
記号 & 定義・役割 \\
\midrule
\tilde\beta_{\text{weekday},i,t} & holiday/phase3 出力係数 \\
\beta_{\text{weekday},i,t}^{(3)} & weekday 系最終係数 (本フェーズ) \\
\bottomrule
\end{tabularx}
\end{minipage}
\end{flushleft}
\bigskip
%===============================================================================
