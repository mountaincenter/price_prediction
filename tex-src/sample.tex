\documentclass[dvipdfmx, openany]{jsbook}
\usepackage{amsmath, amssymb, tabularx, booktabs}
\renewcommand{\arraystretch}{1.2}   % 表の行間

\begin{document}

%────────────────────────────────────
\section*{1. 基本形:日中レンジを伴うモデル}\nopagebreak[4]
%────────────────────────────────────
\begin{align}
\text{ベース値}\quad
  B_{t-1} &=
    \begin{cases}
      Cl_{t-1} & (\text{デイトレ中心})\\[4pt]
      \displaystyle\frac{H_{t-1}+L_{t-1}}{2} & (\text{振れの大きい銘柄})
    \end{cases} \\[10pt]
%
\text{中心シフト量}\quad
  \alpha_t &= \kappa(\sigma_t)\,S_t,\qquad 0\le|S_t|\le1 \\[6pt]
%
\text{日中中心値}\quad
  C_t &= B_{t-1}(1+\alpha_t)\,\beta_{\text{event},t} \\[10pt]
%
\text{半レンジ}\quad
  m_t &= \sigma_t\,\beta_{\text{vol},t} \\[6pt]
%
\text{高値}\quad
  H_t &= C_t + m_t \\[4pt]
\text{安値}\quad
  L_t &= C_t - m_t \\[10pt]
%
\text{始値}\quad
  O_t &= C_t + \gamma_t\sigma_t \\[4pt]
\text{終値}\quad
  Cl_t &= C_t - \gamma_t\sigma_t
\end{align}

\subsection*{主要変数・係数(基本形)}
\noindent\hfill
\begin{minipage}{0.85\textwidth}
\begin{tabularx}{\textwidth}{@{}>{\hfil$\displaystyle}l<{$\hfil}@{\quad}X@{}}
\toprule
記号 & 定義・役割 \\
\midrule
Cl_{t-1} & 前日終値 \\
H_{t-1},L_{t-1} & 前日高値・安値 \\
B_{t-1} & 前日リファレンス値 \\
\sigma_{t} & 当日ボラティリティ推定 \\
\kappa(\sigma) & ボラ依存シフトスケール \\
S_{t} & direction\_score \\
\beta_{\text{event},t} & 曜日・決算などバイアス係数 \\
\beta_{\text{vol},t} & 幅倍率(\(\sigma\) 拡大率) \\
\gamma_{t} & モメンタム偏位係数 \\
\bottomrule
\end{tabularx}
\end{minipage}
\par\bigskip

%────────────────────────────────────
\section*{2. 発射台:EWMA ギャップ補正(Phase 0)}\nopagebreak[4]
%────────────────────────────────────
\begin{equation}
\begin{aligned}
B_{t-1}&=Cl_{t-1}\;\text{or}\;\tfrac{H_{t-1}+L_{t-1}}{2} \\[4pt]
\bar G_t^{(\lambda)}&=\lambda G_t+(1-\lambda)\bar G_{t-1}^{(\lambda)} \\[4pt]
G_t&=O_t-Cl_{t-1},\qquad
O_t=B_{t-1}+\bar G_t^{(\lambda)}
\end{aligned}
\end{equation}

\subsection*{追加変数・係数(発射台)}
\noindent\hfill
\begin{minipage}{0.85\textwidth}
\begin{tabularx}{\textwidth}{@{}>{\hfil$\displaystyle}l<{$\hfil}@{\quad}X@{}}
\toprule
記号 & 定義・役割 \\
\midrule
Cl_{t-1} & 前日終値(基準値の第一候補) \\
H_{t-1},L_{t-1} & 前日高値・安値(中央値で基準値に代用) \\
B_{t-1} & 前日リファレンス値(Cl\(_{t-1}\) または \(\tfrac{H_{t-1}+L_{t-1}}{2}\)) \\
G_t & 当日ギャップ \((O_t-Cl_{t-1})\) \\
\lambda & EWMA 平滑定数(0.1–0.3 が目安) \\
\bar G_t^{(\lambda)} & EWMA 平滑ギャップ \\
O_t & 発射台始値予測(フェーズ0 時点) \\
\bottomrule
\end{tabularx}
\end{minipage}
\par\bigskip

%======================================================================
%                           Phase 1
%======================================================================
\section*{3. Phase 1:ギャップ分布スケーリング}\nopagebreak[4]
%────────────────────────────────────
\subsection*{ステップ / 目的 / 詳細}
\begin{enumerate}
  \item \textbf{直近 63 営業日の IQR を測定}  
        \[
          \mathrm{IQR}_G
            = Q_{75}\!\bigl(G_{t-63\ldots t-1}\bigr)
            - Q_{25}\!\bigl(G_{t-63\ldots t-1}\bigr)
        \]
  \item \textbf{スケーラ \(s_g\) を算出}  
        \[
          s_g=\frac{1}{\mathrm{IQR}_G}
        \]
  \item \textbf{EWMA ギャップをスケール}  
        \[
          \bar G_t'=\bar G_t^{(\lambda)}\,s_g
          \qquad(|\bar G_t'|\le5\sigma\ \text{でクリップ可})
        \]
  \item \textbf{始値を再計算}  
        \(O_t=B_{t-1}+\bar G_t'\)
\end{enumerate}

\subsection*{変数のポイント}
\begin{itemize}
  \item \(\mathrm{IQR}_G\):外れ値に頑健な 50 % 幅。
  \item \(s_g\):銘柄間の“体格差”を均す係数。
\end{itemize}

\subsection*{実装ヒント}
63 d 未満は Phase 0 を使用。  
\(\mathrm{IQR}_G<10^{-4}\) の場合は下限固定でゼロ割り防止。

\subsection*{追加変数・係数(フェーズ1)}
\noindent\hfill
\begin{minipage}{0.85\textwidth}
\begin{tabularx}{\textwidth}{@{}>{\hfil$\displaystyle}l<{$\hfil}@{\quad}X@{}}
\toprule
記号 & 定義・役割 \\
\midrule
Q_{75},Q_{25} & ギャップ四分位点(直近 63 日) \\
\mathrm{IQR}_G & IQR=\(Q_{75}-Q_{25}\) \\
s_g & 分位点スケーラ \(1/\mathrm{IQR}_G\) \\
\bar G_t' & スケール後 EWMA ギャップ \\
O_t & フェーズ1 始値予測 \\
\bottomrule
\end{tabularx}
\end{minipage}
\par\bigskip

%======================================================================
%                           Phase 2
%======================================================================
\section*{4. Phase 2:ボラティリティ連動補正}\nopagebreak[4]
%────────────────────────────────────
\subsection*{ステップ / 目的 / 詳細}
\begin{enumerate}
  \item \textbf{σ 比 \(r_\sigma\) を計算}  
        \[
          r_\sigma=\frac{\sigma_t}{\sigma_{63d}}
        \]
  \item \textbf{ギャップを減衰/増幅}  
        \[
          \bar G_t''=\frac{\bar G_t'}{r_\sigma^{\,\eta}},
          \qquad \eta\approx0.5
        \]
  \item \textbf{始値を更新}  
        \(O_t=B_{t-1}+\bar G_t''\)
\end{enumerate}

\subsection*{変数のポイント}
\begin{itemize}
  \item \(r_\sigma\):当日ボラ / 平常ボラ。0.3–3.0 でクリップ。
  \item \(\eta\):補正感度。0→無補正、1→線形反比例。
\end{itemize}

\subsection*{実装ヒント}
\(\sigma_t\) は前日実現ボラや EWMA(14) で代用可。  
Phase 1→2 追加で MAE が 2–5 % 改善するのが目安。

\subsection*{追加変数・係数(フェーズ2)}
\noindent\hfill
\begin{minipage}{0.85\textwidth}
\begin{tabularx}{\textwidth}{@{}>{\hfil$\displaystyle}l<{$\hfil}@{\quad}X@{}}
\toprule
記号 & 定義・役割 \\
\midrule
\sigma_{63d} & 直近 63 日平均ボラティリティ \\
r_\sigma & σ 比 \(\sigma_t / \sigma_{63d}\) \\
\eta & ボラ補正指数(既定 0.5) \\
\bar G_t'' & ボラ補正後ギャップ \\
O_t & フェーズ2 始値予測 \\
\bottomrule
\end{tabularx}
\end{minipage}
\par\bigskip

%======================================================================
%                           Phase 3
%======================================================================
\section*{5. Phase 3:Proxy Board Gap 補正}\nopagebreak[4]
%────────────────────────────────────
\subsection*{ステップ / 目的 / 詳細}
\begin{enumerate}
  \item \textbf{Proxy Gap を計算}  
        \[
          G_{\text{proxy}}=
          \frac{Cl_{t-1}-\text{5DMA}_{t-1}}{\sigma_{63d}}
        \]
  \item \textbf{出来高比率と重み}  
        \[
          r_v=\frac{\text{Vol}_{t-1}}{\text{AvgVol}_{25}},
          \qquad w_v=\min\!\bigl(1.5,\,r_v\bigr)
        \]
  \item \textbf{最終ギャップを合成}  
        \[
          \bar G_t'''=\bar G_t''+w_v\,G_{\text{proxy}}
        \]
  \item \textbf{始値を決定}  
        \(O_t=B_{t-1}+\bar G_t'''\)
\end{enumerate}

\subsection*{変数のポイント}
\begin{itemize}
  \item 5DMA/25DVMA が無い日は \(w_v=0\)(Proxy 無効化)。
  \item \(w_v\) は 1.5 で飽和し過剰反応を防止。
\end{itemize}

\subsection*{実装ヒント}
終値 vs 当日 VWAP に置換すると滑らかさ向上。  
Phase 2→3 追加で Hit-Rate +1–3 % の改善が典型。

\subsection*{追加変数・係数(フェーズ3)}
\noindent\hfill
\begin{minipage}{0.85\textwidth}
\begin{tabularx}{\textwidth}{@{}>{\hfil$\displaystyle}l<{$\hfil}@{\quad}X@{}}
\toprule
記号 & 定義・役割 \\
\midrule
Cl_{t-1} & 前日終値 \\
\text{5DMA}_{t-1} & 5 日移動平均終値 \\
\sigma_{63d} & 63 日リターン標準偏差 \\
G_{\text{proxy}} & 終値 vs 5DMA の Z-score \\
\text{Vol}_{t-1} & 前日出来高 \\
\text{AvgVol}_{25} & 25 日出来高平均(25DVMA) \\
r_v & 出来高比率 \\
w_v & 出来高重み(上限 1.5) \\
\bar G_t''' & Phase 3 最終ギャップ \\
O_t & Phase 3 始値予測 \\
\bottomrule
\end{tabularx}
\end{minipage}
\par\bigskip

%======================================================================
%                           Phase 4
%======================================================================
\section*{6. Phase 4:自己適応 $\lambda$ 更新}\nopagebreak[4]
%────────────────────────────────────
\subsection*{ステップ / 目的 / 詳細}
\begin{enumerate}
  \item \textbf{直近 30 営業日の誤差を計算}  
        \[
          e_{t-k}=G_{t-k}-\bar G_{t-k}^{(\lambda_{t-1})}
        \]
  \item \textbf{局所 MSE を取る}  
        \[
          \mathrm{MSE}_{t}=\frac1{30}\sum_{k=1}^{30}e_{t-k}^{2}
        \]
  \item \textbf{簡易勾配 \(g_t\) を近似}  
        \[
          g_t\approx-\frac{2}{30}\sum_{k=1}^{30}
          e_{t-k}\,\bar G_{t-k}^{(\lambda_{t-1})}
        \]
  \item \textbf{\(\lambda\) を更新}  
        \[
          \lambda_t=
          \mathrm{clip}\!\bigl(\lambda_{t-1}-\eta\,g_t,\,0.05,\,0.50\bigr)
        \]
  \item \textbf{翌日の EWMA に反映}  
        新しい \(\lambda_t\) で \(\bar G_{t+1}^{(\lambda_t)}\) を算出。
\end{enumerate}

\subsection*{変数のポイント}
\begin{itemize}
  \item \(\lambda\):0.05 → 半減期$\approx$13 d、0.50 → ほぼ当日値。
  \item \(g_t\):\(\lambda\) 増減方向のヒント(標準化・クリップ推奨)。
\end{itemize}

\subsection*{実装ヒント}
\begin{itemize}
  \item \textbf{ウォーミングアップ}:30 d 溜まるまでは \(\lambda=0.20\) 固定。
  \item \textbf{数値安定化}:\(|g_t|\le10\) でサチらせ暴走防止。
  \item \textbf{バックテスト}:Phase 3 vs Phase 4 で MAE と DD を比較。
\end{itemize}

\subsection*{追加変数・係数(フェーズ4)}
\noindent\hfill
\begin{minipage}{0.85\textwidth}
\begin{tabularx}{\textwidth}{@{}>{\hfil$\displaystyle}l<{$\hfil}@{\quad}X@{}}
\toprule
記号 & 定義・役割 \\
\midrule
\lambda_{t-1} & 前日の EWMA 平滑定数 \\
\lambda_{t} & 更新後 EWMA 平滑定数(0.05–0.50) \\
e_{t-k} & 誤差 \(G_{t-k}-\bar G_{t-k}^{(\lambda_{t-1})}\) \\
\mathrm{MSE}_{t} & 過去 30 d 平均二乗誤差 \\
g_t & MSE 勾配近似 \\
\eta & 学習率(0.005–0.02 推奨) \\
\bottomrule
\end{tabularx}
\end{minipage}

\end{document}
