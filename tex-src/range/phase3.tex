%-------------------------------------------------------------------------------
% range/phase3.tex   v1.2  (2025-06-02)
%-------------------------------------------------------------------------------
% CHANGELOG  -- newest -> oldest
% - 2025-06-02  v1.2 : 「変数のポイント」節を復元(出来高補正の趣旨を明示)
% - 2025-05-31  v1.1 : β_vol,t^{(3)} 表記統一
% - 2025-05-31  v1.0 : Phase-3 出来高補正 初版
%-------------------------------------------------------------------------------

%=== Phase-3 : 出来高補正 β_vol,t^{(3)} ========================================
\section*{range / Phase 3}\nopagebreak[4]
%────────────────────────────────────
\subsection*{ステップ・目的}
\begin{flushleft}
\begin{enumerate}
  \item \textbf{出来高比率を計算}\;
        \( r_v = \dfrac{\text{Vol}_{t-1}}{\text{AvgVol}_{25}} \)
  \item \textbf{指数補正}\;
        \( \beta_{\text{vol},t}^{(3)}
           =\beta_{\text{vol},t}^{(2)}\;r_v^{\eta_v},
           \quad \eta_v = 0.4 \)
        (\(0.2 \le \beta_{\text{vol},t}^{(3)} \le 5.0\) でクリップ)
  \item \textbf{半レンジを更新}\;
        \( m_t=\sigma_t^{\text{shift}}\;\beta_{\text{vol},t}^{(3)} \)
\end{enumerate}
\end{flushleft}

\subsection*{変数のポイント}
\begin{flushleft}
\begin{itemize}
  \item 出来高急増時はレンジを拡大、低迷時は縮小。  
  \item \(\eta_v=0\) とすれば出来高補正を無効化し  
        Phase 2 の幅倍率をそのまま使用。
\end{itemize}
\end{flushleft}

\subsection*{追加変数・係数}
\begin{flushleft}
\begin{minipage}{0.88\textwidth}
\begin{tabularx}{\textwidth}{@{}lX@{}}
\toprule
記号 & 定義・役割 \\
\midrule
\(\text{Vol}_{t-1}\) & 前日出来高 \\
\(\text{AvgVol}_{25}\) & 25 日平均出来高 \\
\(r_v\) & 出来高比率 \\
\(\eta_v\) & 出来高補正指数(既定 0.4) \\
\(\beta_{\text{vol},t}^{(2)}\) & Phase 2 幅倍率 \\
\(\beta_{\text{vol},t}^{(3)}\) & Phase 3 幅倍率(出来高反映) \\
\(m_t\) & 半レンジ \\
\bottomrule
\end{tabularx}
\end{minipage}
\end{flushleft}
%===============================================================================
