%-------------------------------------------------------------------------------
% range/phase6.tex   v1.0  (2025-06-11)
%-------------------------------------------------------------------------------
% CHANGELOG  -- newest -> oldest
% - 2025-06-11  v1.0 : 初版 (m_fin クリップ)
%-------------------------------------------------------------------------------

%=== Phase-6 : 騰落キャップ =============================================
\section*{range / Phase 6}\nopagebreak[4]
%────────────────────────────────────
\subsection*{ステップ・目的}
\begin{flushleft}
\begin{enumerate}
  \item 直近63日間の半レンジ \(R_{t-k}\) から
        四分位点 \(Q_{75}, Q_{25}\) を取得
  \item クリップ上限を
        \( m_{\text{cap}} = Q_{75} + 1.5(Q_{75}-Q_{25}) \) とする
  \item 最終半レンジを
        \( m_t^{\text{fin}} = \operatorname{clip}(m_t^{\text{pred}}, 0, m_{\text{cap}}) \)
\end{enumerate}
\end{flushleft}

\subsection*{変数のポイント}
\begin{flushleft}
\begin{itemize}
  \item 急激な騰落はファンダメンタル要因が必要となるため、
        本手法では IQR に基づく上限で抑制する
  \item 上限計算に過去 63 日を用いることで銘柄ごとの体感値を反映
\end{itemize}
\end{flushleft}

\subsection*{追加変数・係数}
\begin{flushleft}
\begin{minipage}{0.88\textwidth}
\begin{tabularx}{\textwidth}{@{}lX@{}}
\toprule
記号 & 定義・役割 \\
\midrule
\(Q_{75},Q_{25}\) & 75/25 パーセンタイル (63~d) \\
\(m_{\text{cap}}\) & クリップ上限 (1.5~IQR) \\
\(m_t^{\text{pred}}\) & Phase 5 出力 \\
\(m_t^{\text{fin}}\) & Phase 6 最終半レンジ \\
\bottomrule
\end{tabularx}
\end{minipage}
\end{flushleft}
%-------------------------------------------------------------------------------
