%-------------------------------------------------------------------------------
% range/phase1.tex   v1.3  (2025-06-02)
%-------------------------------------------------------------------------------
% CHANGELOG  -- newest -> oldest
% - 2025-06-02  v1.3 : β_vol,t^{(1)} 表記統一・変数ポイント復活
% - 2025-05-31  v1.1 : β_vol,t' 表記
% - 2025-05-31  v1.0 : 初版
%-------------------------------------------------------------------------------

%=== Phase 1 : レンジ分布スケーリング ==========================================
\section*{range / Phase 1}\nopagebreak[4]
%────────────────────────────────────
\subsection*{ステップ・目的}
\begin{flushleft}
\begin{enumerate}
  \item 直近 63 営業日の半レンジ  
        \(R_{t-k}=(H_{t-k}-L_{t-k})/2\)
  \item IQR 計算  
        \(\mathrm{IQR}_R=Q_{75}(R)-Q_{25}(R)\)
  \item スケーラ  
        \(s_{\text{range}}=1/\mathrm{IQR}_R\)
  \item 幅倍率更新  
        \(\beta_{\text{vol},t}^{(1)}=\operatorname{clip}(s_{\text{range}},0.20,5.00)\)
  \item 半レンジ再計算  
        \(m_t=\sigma_t^{\text{shift}}\;\beta_{\text{vol},t}^{(1)}\)
\end{enumerate}
\end{flushleft}

\subsection*{変数のポイント}
\begin{flushleft}
\begin{itemize}
  \item \(\mathrm{IQR}_R\)\;外れ値に強い 50 % 幅  
  \item 63 d 未満は Phase 0 値(1.0)を使用  
  \item \(\mathrm{IQR}_R<1.0\times10^{-4}\) なら固定下限で割り算防止
\end{itemize}
\end{flushleft}

\subsection*{追加変数・係数}
\begin{flushleft}
\begin{minipage}{0.88\textwidth}
\begin{tabularx}{\textwidth}{@{}lX@{}}
\toprule
記号 & 定義・役割 \\
\midrule
\(Q_{75},Q_{25}\) & 四分位点(63 d)\\
\(\beta_{\text{vol},t}^{(0)}\) & 前フェーズ幅倍率\\
\(\beta_{\text{vol},t}^{(1)}\) & 本フェーズ幅倍率\\
\(m_t\) & 半レンジ\\
\(\sigma_t^{\text{shift}}\) & 共通ボラ\\
\bottomrule
\end{tabularx}
\end{minipage}
\end{flushleft}
%===============================================================================
