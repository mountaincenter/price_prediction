%-------------------------------------------------------------------------------
% momentum/phase2.tex   v1.1  (2025-06-02)
%-------------------------------------------------------------------------------
% CHANGELOG  -- new entry on top
% - 2025-06-02  v1.1 : 「変数のポイント」節を追加
% - 2025-05-31  v1.0 : RSI14 で gamma_t 強弱 (±0.10)
%-------------------------------------------------------------------------------

%=== Phase-2 : RSI 強弱 ========================================================
\section*{Phase 2:RSI 強弱}\nopagebreak[4]
%────────────────────────────────────
\subsection*{ステップ/目的}
\begin{flushleft}
\begin{enumerate}
  \item \textbf{RSI14 を計算}\;
        \( \text{RSI}_t=\text{RSI}_{14}(Cl) \)
  \item \textbf{強弱スケール}\;
        \( \Delta_{\text{RSI}}=0.05\,\tanh\!\bigl((\text{RSI}_t-50)/20\bigr) \)
  \item \textbf{モメンタム係数を更新}\;
        \( \gamma_t^{(2)}=\operatorname{clip}\!\bigl(
           \gamma_t^{(1)}+\Delta_{\text{RSI}},\,-0.10,\,0.10\bigr) \)
\end{enumerate}
\end{flushleft}

\subsection*{変数のポイント}
\begin{flushleft}
\begin{itemize}
  \item RSI>70 → 買われ過ぎ:上寄り幅を縮小。RSI<30 → 売られ過ぎ:下寄り幅を縮小。  
  \item \(|\Delta_{\text{RSI}}|\le0.05\) に制限し外挿を抑止。
\end{itemize}
\end{flushleft}

\subsection*{追加変数・係数}
\begin{flushleft}
\begin{minipage}{0.88\textwidth}
\begin{tabularx}{\textwidth}{@{}>{\hfil$\displaystyle}l<{$\hfil}@{\quad}X@{}}
\toprule
記号 & 定義・役割 \\
\midrule
\text{RSI}_t & 14 d RSI (0–100) \\
\Delta_{\text{RSI}} & 強弱変位 (±0.05) \\
\gamma_t^{(1)} & Phase 1 入力 \\
\gamma_t^{(2)} & Phase 2 出力 (±0.10) \\
\bottomrule
\end{tabularx}
\end{minipage}
\end{flushleft}
\bigskip
%===============================================================================
