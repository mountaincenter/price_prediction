%-------------------------------------------------------------------------------
% momentum/phase5.tex   v1.1  (2025-06-02)
%-------------------------------------------------------------------------------
% CHANGELOG  -- new entry on top
% - 2025-06-02  v1.1 : 「変数のポイント」節を追加
% - 2025-05-31  v1.0 : ベイズ縮小でセクター平均へ収束
%-------------------------------------------------------------------------------

%=== Phase-5 : ベイズ縮小 ======================================================
\section*{Phase 5:ベイズ縮小}\nopagebreak[4]
%────────────────────────────────────
\subsection*{ステップ/目的}
\begin{flushleft}
\begin{enumerate}
  \item \textbf{サンプル数を取得}\;
        \(n_i=\text{count}(\gamma_{i,\ast})\)
  \item \textbf{セクター平均}\;
        \(\bar\gamma_s = \text{mean}(\gamma_{j,t}^{(4)}\mid j\in s)\)
  \item \textbf{縮小係数}\;
        \(\tau=15\)
  \item \textbf{ベイズ縮小}\;
        \[
          \gamma_{i,t}^{\text{final}}
            =\frac{n_i}{n_i+\tau}\,\gamma_{i,t}^{(4)}
            +\frac{\tau}{n_i+\tau}\,\bar\gamma_s
        \]
\end{enumerate}
\end{flushleft}

\subsection*{変数のポイント}
\begin{flushleft}
\begin{itemize}
  \item サンプル不足銘柄 (\(n_i\) 小) は **セクター平均** に引き寄せ過学習を回避。  
  \item \(\tau\) を大きくすると縮小強度↑、小さいと個別値を優先。
\end{itemize}
\end{flushleft}

\subsection*{追加変数・係数}
\begin{flushleft}
\begin{minipage}{0.90\textwidth}
\begin{tabularx}{\textwidth}{@{}>{\hfil$\displaystyle}l<{$\hfil}@{\quad}X@{}}
\toprule
記号 & 定義・役割 \\
\midrule
n_i & 銘柄 i のサンプル数 \\
\tau & 縮小強度 (15) \\
\bar\gamma_s & セクター平均 γ \\
\gamma_{i,t}^{(4)} & Phase 4 入力 \\
\gamma_{i,t}^{\text{final}} & 最終モメンタム係数 \\
\bottomrule
\end{tabularx}
\end{minipage}
\end{flushleft}
\bigskip
%===============================================================================
