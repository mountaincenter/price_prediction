%-------------------------------------------------------------------------------
% momentum/phase1.tex   v1.1  (2025-06-02)
%-------------------------------------------------------------------------------
% CHANGELOG  -- new entry on top
% - 2025-06-02  v1.1 : 「変数のポイント」節を追加
% - 2025-05-31  v1.0 : EMA5 符号で gamma_t±0.05 を決定
%-------------------------------------------------------------------------------

%=== Phase-1 : EMA5 符号 =======================================================
\section*{Phase 1:EMA5 符号}\nopagebreak[4]
%────────────────────────────────────
\subsection*{ステップ/目的}
\begin{flushleft}
\begin{enumerate}
  \item \textbf{EMA5 リターンを計算}\;
        \(r_{5,t}=\mathrm{EMA}_5(\Delta Cl_t)\)
  \item \textbf{符号を取得}\;
        \(\operatorname{sgn}_t=\operatorname{sign}(r_{5,t})\)
  \item \textbf{モメンタム係数を決定}\;
        \(\gamma_t^{(1)}=0.05\,\operatorname{sgn}_t\)
\end{enumerate}
\end{flushleft}

\subsection*{変数のポイント}
\begin{flushleft}
\begin{itemize}
  \item EMA5 は短期トレンドの最小検出器。  
  \item **正符号**なら始値を +0.05σ 上へ、**負符号**なら -0.05σ 下へシフト。
\end{itemize}
\end{flushleft}

\subsection*{追加変数・係数}
\begin{flushleft}
\begin{minipage}{0.88\textwidth}
\begin{tabularx}{\textwidth}{@{}>{\hfil$\displaystyle}l<{$\hfil}@{\quad}X@{}}
\toprule
記号 & 定義・役割 \\
\midrule
\Delta Cl_t & \(\ln(Cl_t/Cl_{t-1})\) \\
r_{5,t} & 5 d EMA リターン \\
\operatorname{sgn}_t & 符号 (‐1,0,+1) \\
\gamma_t^{(1)} & Phase 1 出力 (±0.05) \\
\bottomrule
\end{tabularx}
\end{minipage}
\end{flushleft}
\bigskip
%===============================================================================
