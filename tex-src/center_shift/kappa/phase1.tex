%-------------------------------------------------------------------------------
% center_shift/kappa/phase1.tex   v1.3  (2025-06-02)
%-------------------------------------------------------------------------------
% CHANGELOG  -- new entry on top
% - 2025-06-02  v1.3 : add section/subsection hierarchy, hints, ASCII-only
%-------------------------------------------------------------------------------

%=== center_shift =============================================================
\section*{center\_shift}\nopagebreak[4]

%--- kappa ---------------------------------------------------------------------
\subsection*{kappa}\nopagebreak[4]

%--- Phase 1 : 段階定数モデル ---------------------------------------------------
\subsubsection*{Phase 1:段階定数モデル}\nopagebreak[4]
%────────────────────────────────────
\paragraph{スケール関数}
\begin{flushleft}
\begin{equation*}
\kappa(\sigma_t^{\text{shift}})=
  \begin{cases}
    0.05 & 0 \le \sigma_t^{\text{shift}} < 0.01\\[4pt]
    0.10 & 0.01 \le \sigma_t^{\text{shift}} < 0.02\\[4pt]
    0.15 & 0.02 \le \sigma_t^{\text{shift}} < 0.04\\[4pt]
    0.20 & \sigma_t^{\text{shift}} \ge 0.04
  \end{cases}
\end{equation*}
\end{flushleft}

\subsubsection*{変数のポイント}
\begin{flushleft}
\begin{itemize}
  \item 4 バケットで十分な滑らかさと解釈性を両立
  \item 高ボラ域では \(\kappa\) を抑え過剰シフトを防止
\end{itemize}
\end{flushleft}

\subsubsection*{実装ヒント}
\begin{flushleft}
週次で分布を点検し、しきい値 \(0.01,0.02,0.04\) をチューニングすると  
Hit-Rate が 0.5–1.0 pt 改善するケースがあります。
\end{flushleft}

\subsubsection*{追加変数・係数}
\begin{flushleft}
\begin{minipage}{0.86\textwidth}
\begin{tabularx}{\textwidth}{@{}>{\hfil$\displaystyle}l<{$\hfil}@{\quad}X@{}}
\toprule
記号 & 定義・備考 \\
\midrule
\sigma_t^{\text{shift}} & 共通ボラ (sigma/phase1.tex) \\
\kappa(\sigma_t^{\text{shift}}) & スケール係数 (本表) \\
\alpha_t & 中心シフト量 \\
\bottomrule
\end{tabularx}
\end{minipage}
\end{flushleft}
\bigskip
%===============================================================================
