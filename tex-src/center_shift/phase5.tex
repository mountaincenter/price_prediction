%-------------------------------------------------------------------------------
% center_shift/phase5.tex   v1.0  (2025-06-13)
%-------------------------------------------------------------------------------
% CHANGELOG  -- new entry on top (latest -> oldest)
% - 2025-06-13  v1.0 : 短期ノイズ平滑化用に5日・10日平均を導入
%-------------------------------------------------------------------------------

%=== center_shift =============================================================
\section*{center\_shift}\nopagebreak[4]

%=== Phase 5 : 短期ノイズ平滑化 ==============================================
\subsection*{Phase 5:短期ノイズ平滑化}\nopagebreak[4]
%────────────────────────────────────
\paragraph{ステップ/目的}
\begin{flushleft}
\begin{enumerate}
  \item \textbf{基準値平滑化}:前日中値 \(B_{t-1}\) を 5 日・10 日移動平均して
        \(\overline{B}_{t-1}\) を得る
  \item \textbf{予想中央値}:\(C_{p,t}=\overline{B}_{t-1}\bigl(1+\alpha_t\sigma_t^{\mathrm{shift}}\bigr)\)
  \item \textbf{誤差正規化}:\(\mathrm{Norm\_err}_t=|C_{p,t}-C_{r,t}|\big/\bigl(\overline{B}_{t-1}\sigma_t^{\mathrm{shift}}\bigr)\)
\end{enumerate}
\end{flushleft}

\subsubsection*{変数のポイント}
\begin{flushleft}
\begin{itemize}
  \item 短期的な値飛びを抑え、1\% 超の外れ値判定を安定化
  \item 5 日平均に加え、参考値として 10 日平均も保持する
\end{itemize}
\end{flushleft}

\bigskip
%==============================================================================
