%-------------------------------------------------------------------------------
% center_shift/phase2.tex   v1.0  (2025-06-05)
%-------------------------------------------------------------------------------
% CHANGELOG  -- new entry on top (latest -> oldest)
% - 2025-06-05  v1.0 : 初版
%-------------------------------------------------------------------------------

%=== center_shift =============================================================
\section*{center\_shift}\nopagebreak[4]

%=== Phase 2 : 高精度 \alpha_t with \lambda 更新 ============================
\subsection*{Phase 2:高精度 $\alpha_t$ 算出}\nopagebreak[4]
%────────────────────────────────────
\paragraph{ステップ/目的}
\begin{flushleft}
\begin{enumerate}
  \item \textbf{共通ボラ計算}\;\(\sigma_t^{\text{shift}}\) を
        {\scriptsize\verb|sigma/phase2.tex|} の自己適応
        $\lambda_{\text{shift}}$ で更新
  \item \textbf{スケール係数}\;\(\kappa_t=\kappa(\sigma_t^{\text{shift}})\) を
        {\scriptsize\verb|kappa/phase1.tex|} の段階定数モデルで取得
  \item \textbf{方向スコア}\;\(S_t=\operatorname{sign}(\Delta Cl_{t-1})\)
  \item \textbf{中心シフト量}\;\(\alpha_t=\kappa_t S_t\)
\end{enumerate}
\end{flushleft}

\subsubsection*{変数のポイント}
\begin{flushleft}
\begin{itemize}
  \item $\lambda_{\text{shift}}$ は MSE 勾配で適応更新
  \item \(\kappa_t\) しきい値は週次で点検し調整可能
\end{itemize}
\end{flushleft}

\subsubsection*{実装ヒント}
\begin{flushleft}
初期 5 日間は $S_t=0$・$\alpha_t=0$ とする。30~d のウォームアップ後に
$\lambda_{\text{shift}}$ の更新を開始する。
\end{flushleft}

\subsubsection*{追加変数・係数}
\begin{flushleft}
\begin{minipage}{0.88\textwidth}
\begin{tabularx}{\textwidth}{@{}>{\hfil$\displaystyle}l<{$\hfil}@{\quad}X@{}}
\toprule
記号 & 定義・役割 \\
\midrule
\sigma_t^{\text{shift}} & 自己適応ボラ (sigma/phase2.tex) \\
\lambda_{\text{shift}} & EWMA 更新係数 \\
\kappa_t & スケール係数 (kappa/phase1.tex) \\
S_t & 前日符号 (\!-1,0,+1) \\
\alpha_t & 中心シフト量 \\
\bottomrule
\end{tabularx}
\end{minipage}
\end{flushleft}
\bigskip
%===============================================================================
