%-------------------------------------------------------------------------------
% center_shift/phase2.tex   v1.1  (2025-06-05)
%-------------------------------------------------------------------------------
% CHANGELOG  -- new entry on top (latest -> oldest)
% - 2025-06-05  v1.1 : HitRate 改善手法を明記
% - 2025-06-05  v1.0 : 初版
%-------------------------------------------------------------------------------

%=== center_shift =============================================================
\section*{center\_shift}\nopagebreak[4]

%=== Phase 2 : HitRate 改善アルゴリズム ============================
\subsection*{Phase 2:HitRate 向上}\nopagebreak[4]
%────────────────────────────────────
\paragraph{ステップ/目的}
\begin{flushleft}
\begin{enumerate}
  \item \textbf{共通ボラ計算}\;\(\sigma_t^{\text{shift}}\) を
        {\scriptsize\verb|sigma/phase2.tex|} で求め、自己適応
        $\lambda_{\text{shift}}$ を反映
  \item \textbf{方向スコア}\;3 日平均リターン
        \(\bar{r}_t=\tfrac{1}{3}\sum_{k=1}^3 \Delta Cl_{t-k}\) の符号
  \item \textbf{閾値処理}\;$|\bar{r}_t|<0.5\,\sigma_t^{\text{shift}}$ なら
        \(S_t=0\)
  \item \textbf{スケール係数}\;\(\kappa_t=\kappa(\sigma_t^{\text{shift}})\) を
        {\scriptsize\verb|kappa/phase1.tex|} の段階定数モデルで取得
  \item \textbf{中心シフト量}\;\(\alpha_t=\kappa_t S_t\)
\end{enumerate}
\end{flushleft}

\subsubsection*{変数のポイント}
\begin{flushleft}
\begin{itemize}
  \item $\lambda_{\text{shift}}$ は MSE 勾配で適応更新
  \item \(\kappa_t\) しきい値は週次で点検し調整可能
  \item 3 日平均リターン \(\bar{r}_t\) で方向ノイズを低減
\end{itemize}
\end{flushleft}

\subsubsection*{実装ヒント}
\begin{flushleft}
初期 5 日間は $S_t=0$・$\alpha_t=0$ とする。30~d のウォームアップ後に
$\lambda_{\text{shift}}$ の更新を開始。小幅な \(\bar{r}_t\) は
\(S_t=0\) として無視する。
\end{flushleft}

\subsubsection*{追加変数・係数}
\begin{flushleft}
\begin{minipage}{0.88\textwidth}
\begin{tabularx}{\textwidth}{@{}>{\hfil$\displaystyle}l<{$\hfil}@{\quad}X@{}}
\toprule
記号 & 定義・役割 \\
\midrule
\sigma_t^{\text{shift}} & 自己適応ボラ (sigma/phase2.tex) \\
\lambda_{\text{shift}} & EWMA 更新係数 \\
\kappa_t & スケール係数 (kappa/phase1.tex) \\
\bar{r}_t & 3 日平均リターン \\
S_t & 前日符号 (\!-1,0,+1) \\
\alpha_t & 中心シフト量 \\
\bottomrule
\end{tabularx}
\end{minipage}
\end{flushleft}
\bigskip
%===============================================================================
