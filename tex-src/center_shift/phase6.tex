%-------------------------------------------------------------------------------
% center_shift/phase6.tex   v1.0  (2025-06-13)
%-------------------------------------------------------------------------------
% CHANGELOG  -- new entry on top (latest -> oldest)
% - 2025-06-13  v1.0 : C_p 改善手順を総括
%-------------------------------------------------------------------------------

%=== center_shift =============================================================
\section*{center\_shift}\nopagebreak[4]

%=== Phase 6 : C_p 改善の総括 ================================================
\subsection*{Phase 6:$C_p$ 改善の総括}\nopagebreak[4]
%────────────────────────────────────
\paragraph{ステップ/目的}
\begin{flushleft}
\begin{enumerate}
  \item \textbf{共通ボラ}:{\scriptsize\verb|sigma/phase1.tex|} や {\scriptsize\verb|sigma/phase2.tex|} に基づき
        EWMA14 ボラ \(\sigma_t^{\mathrm{shift}}\) を算出
  \item \textbf{スケール係数}:{\scriptsize\verb|kappa/phase1.tex|} の段階定数モデルから
        \(\kappa_t=\kappa(\sigma_t^{\mathrm{shift}})\) を取得
  \item \textbf{方向スコア}:3 日平均リターンの符号を取り、閾値未満は 0 とする
  \item \textbf{中心シフト量}:\(\alpha_t=\kappa_t S_t\) としてボラ単位のシフトを計算
  \item \textbf{基準値平滑化}:前日中値を 10 日平均し \(\overline{B}_{10,t-1}\) を得る
  \item \textbf{予想中央値}:\(C_{p,t}=\overline{B}_{10,t-1}\bigl(1+\alpha_t\sigma_t^{\mathrm{shift}}\bigr)\)
  \item \textbf{誤差正規化}:\(\mathrm{Norm\_err}_t=|C_{p,t}-C_{r,t}|\big/\bigl(\overline{B}_{10,t-1}\sigma_t^{\mathrm{shift}}\bigr)\)
  \item \textbf{比率平滑化}:\(\overline{C_\Delta/C_p}=\text{10d MA of }C_\Delta/C_p\)
\end{enumerate}
\end{flushleft}

\subsubsection*{変数のポイント}
\begin{flushleft}
\begin{itemize}
  \item \(\overline{C_\Delta/C_p}\) が \(\pm1\%\) を超える場合のみ外れ値と判定(区分 0--8 は変更なし)
  \item $C_p$ 算出にはボラ水準に応じた \(\kappa_t\) と方向スコア $S_t$ を組み合わせる
\end{itemize}
\end{flushleft}

\bigskip
%==============================================================================

