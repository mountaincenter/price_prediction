%-------------------------------------------------------------------------------
% center_shift/phase3.tex   v1.0  (2025-06-06)
%-------------------------------------------------------------------------------
% CHANGELOG  -- new entry on top (latest -> oldest)
% - 2025-06-06  v1.0 : 初版
%-------------------------------------------------------------------------------

%=== center_shift =============================================================
\section*{center\_shift}\nopagebreak[4]

%=== Phase 3 : \alpha_t の深掘り ==============================================
\subsection*{Phase 3:$\alpha_t$ の深掘り}\nopagebreak[4]
%────────────────────────────────────
\paragraph{ステップ/目的}
\begin{flushleft}
\begin{enumerate}
  \item \textbf{基準値}\;前日中値
        \(B_{t-1}=\tfrac{High_{t-1}+Low_{t-1}}{2}\)
  \item \textbf{予想中央値}\;
        \(C_{p,t}=B_{t-1}\bigl(1+\alpha_t\sigma_t^{\mathrm{shift}}\bigr)\)
  \item \textbf{誤差正規化}\;
        \(\mathrm{Norm\_err}_t
          =|C_{p,t}-C_{r,t}|\big/\bigl(B_{t-1}\sigma_t^{\mathrm{shift}}\bigr)\)
  \item \textbf{シフト解釈}\;\(\alpha_t\) はボラ単位での中心シフト量を示す
\end{enumerate}
\end{flushleft}

\subsubsection*{変数のポイント}
\begin{flushleft}
\begin{itemize}
  \item \(\sigma_t^{\mathrm{shift}}\) は Phase 2 までと同じ EWMA14 系列
  \item \(\alpha_t\sigma_t^{\mathrm{shift}}\) により、銘柄価格スケールへ依存しない
  \item 小幅な値では \(C_{p,t}\approx B_{t-1}\) となり保守的な予測に
\end{itemize}
\end{flushleft}

\subsubsection*{実装ヒント}
\begin{flushleft}
初期 5 日間は \(\alpha_t=0\) としてシフトを無効化し、\par
30~d のウォームアップ後に各係数の更新を開始すると安定します。
\end{flushleft}

\subsubsection*{追加変数・係数}
\begin{flushleft}
\begin{minipage}{0.88\textwidth}
\begin{tabularx}{\textwidth}{@{}>{\hfil$\displaystyle}l<{$\hfil}@{\quad}X@{}}\toprule
記号 & 定義・役割 \\\midrule
B_{t-1} & 前日中値 (High+Low)/2 \\
C_{p,t} & 予想中央値 \\
C_{r,t} & 実際中央値 \\
\mathrm{Norm\_err}_t & 正規化誤差 \\
\alpha_t & 中心シフト量 (ボラ単位) \\\bottomrule
\end{tabularx}
\end{minipage}
\end{flushleft}
\bigskip
%===============================================================================
