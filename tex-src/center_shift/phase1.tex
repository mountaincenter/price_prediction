%-------------------------------------------------------------------------------
% center_shift/phase1.tex   v1.3  (2025-06-06)
%-------------------------------------------------------------------------------
% CHANGELOG  -- new entry on top (latest -> oldest)
% - 2025-06-06  v1.3 : math delimiter fix for \alpha_t heading
% - 2025-06-06  v1.2 : kappa/sigma の位置付けを明記し改善手順を追記
% - 2025-06-05  v1.1 : math delimiter fix
% - 2025-06-05  v1.0 : 初版
%-------------------------------------------------------------------------------

%=== center_shift =============================================================
\section*{center\_shift}\nopagebreak[4]

%=== Phase 1 : 動的 \alpha_t 算出 =============================================
\subsection*{Phase 1:動的 $\alpha_t$ 算出}\nopagebreak[4]
%────────────────────────────────────
\paragraph{ステップ/目的}
\begin{flushleft}
\begin{enumerate}
  \item \textbf{共通ボラ計算}\;{\scriptsize\verb|sigma/phase1.tex|} を参照し、
        EWMA14 で \(\sigma_t^{\text{shift}}\) を求める
  \item \textbf{スケール係数}\;{\scriptsize\verb|kappa/phase1.tex|} の段階定数
        モデルから \(\kappa_t=\kappa(\sigma_t^{\text{shift}})\) を取得
  \item \textbf{方向スコア}\;\(S_t=\operatorname{sign}(\Delta Cl_{t-1})\)
  \item \textbf{中心シフト量}\;\(\alpha_t=\kappa_t S_t\)
\end{enumerate}
\end{flushleft}

\subsubsection*{変数のポイント}
\begin{flushleft}
\begin{itemize}
  \item ボラ低位では \(\kappa_t\) を小さく抑え過剰シフトを防止
  \item 高ボラ域では段階的に \(\kappa_t\) を増加させ方向性を強調
\end{itemize}
\end{flushleft}

\subsubsection*{実装ヒント}
\begin{flushleft}
初期 5 日間は \(S_t=0\)・\(\alpha_t=0\) として無効化します。\par
ボラ推定は {\scriptsize\verb|sigma/phase1.tex|} の勾配チェック、
\(\kappa_t\) は {\scriptsize\verb|kappa/phase1.tex|} の閾値調整を参照し、
週次でバケット境界を点検すると精度向上が見込めます。
\end{flushleft}

\subsubsection*{追加変数・係数}
\begin{flushleft}
\begin{minipage}{0.88\textwidth}
\begin{tabularx}{\textwidth}{@{}>{\hfil$\displaystyle}l<{$\hfil}@{\quad}X@{}}
\toprule
記号 & 定義・役割 \\
\midrule
\sigma_t^{\text{shift}} & EWMA14 ボラ (sigma/phase1.tex) \\
\kappa_t & スケール係数 (kappa/phase1.tex) \\
S_t & 前日符号 (‐1,0,+1) \\
\alpha_t & 中心シフト量 \\
\bottomrule
\end{tabularx}
\end{minipage}
\end{flushleft}
\bigskip
%===============================================================================
