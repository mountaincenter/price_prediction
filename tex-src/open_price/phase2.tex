%-------------------------------------------------------------------------------
% open_price/phase2.tex   v1.6  (2025-06-02)
%-------------------------------------------------------------------------------
% CHANGELOG  -- new entry on top (latest -> oldest)
% - 2025-06-02  v1.6 : add “変数のポイント” & “実装ヒント”, ASCII only, 0.88w
% - 2025-06-02  v1.5 : add section header
% - 2025-05-31  v1.3 : G_t^{(2)} 表記
%-------------------------------------------------------------------------------

%=== open_price ===============================================================
\section*{open\_price}\nopagebreak[4]

%=== Phase 2 : ボラティリティ連動補正 ==========================================
\subsection*{Phase 2:ボラティリティ連動補正}\nopagebreak[4]
%────────────────────────────────────
\begin{flushleft}
\begin{enumerate}
  \item \textbf{ボラ比を計算}\;
        \( r_\sigma = \sigma_t^{\text{open}} \big/ \sigma_{63d}^{\text{open}} \)

  \item \textbf{補正ギャップ}\;
        \( G_t^{(2)} = \dfrac{G_t^{(1)}}{r_\sigma^{\,\eta}},
           \qquad \eta = 0.5 \)

  \item \textbf{始値を更新}\;
        \( O_t = B_{t-1} + G_t^{(2)} \)
\end{enumerate}
\end{flushleft}

\subsection*{変数のポイント}
\begin{flushleft}
\begin{itemize}
  \item \(r_\sigma\):当日ボラと平常ボラの比率 (0.3–3.0 でクリップ)  
  \item \(\eta\):補正式の指数。0 で無補正、1 で線形反比例
\end{itemize}
\end{flushleft}

\subsection*{実装ヒント}
\begin{flushleft}
\begin{itemize}
  \item \(\sigma_t^{\text{open}}\) は center\_shift と同じ EWMA14 値を共有  
  \item Phase 1 → 2 で MAE が 2~to~5\% 改善するのが典型
\end{itemize}
\end{flushleft}

\subsection*{追加変数・係数}
\begin{flushleft}
\begin{minipage}{0.88\textwidth}
\begin{tabularx}{\textwidth}{@{}>{\hfil$\displaystyle}l<{$\hfil}@{\quad}X@{}}
\toprule
記号 & 定義・役割 \\
\midrule
\sigma_t^{\text{open}} & 当日ボラ (center\_shift と共有) \\
\sigma_{63d}^{\text{open}} & 63~d ボラ平均 \\
r_\sigma & ボラ比 \\
\eta & 補正指数 (0.5) \\
G_t^{(1)} & Phase 1 ギャップ \\
G_t^{(2)} & Phase 2 ギャップ \\
O_t & Phase 2 始値予測 \\
\bottomrule
\end{tabularx}
\end{minipage}
\end{flushleft}
\bigskip
%===============================================================================
